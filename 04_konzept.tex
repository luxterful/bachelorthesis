\chapter{Konzeption und Implementierung}

\section{Anforderungsanalyse}

Für die Anforderungsanalyse wird auf die Prozesse des Requirement Engineerings gesetzt. Qualitative Datenerhebung mit einem semistrukturiertem Interview. In diesem werden Fragen gestellt um dem Interview eine Thematische Richtung zu geben, es aber nicht zu sehr einzuschränken. Für dieses Interview wurde ein Leitfaden erstellt. In diesem wird der Ablauf genau definiert und dient während des Gespräches als Orientierung.

Interviewleitfaden

Forschungsfrage: Wie können die Gamification Elemente der Plattform Twitch von Firmen in der Musikbranche für Marketingzwecke eingesetzt werden?

Einstieg

- Begrüßung des Interviewees und Bedanken für die Teilnahme

- Kurze Einführung in das Thema

- Erklären des Interview Leitfadens

- Datenschutzvereinbarung

Einstigsfragen
- Was genau ist ihre aktuelle Profession und wie lange sind Sie dort schon beschäftigt?
- Was gehört hierbei zum Tagesgeschäft?

Schlüsselfragen

Frage 1: Was sind die Revenue Streams eines Unternehmens in der Musik Branche?

Frage: Welche Zielgruppe ist besonders wichtig?

Frage 2: Was sind die aktuellen Wege einen Kunden / Fan zu erreichen?

Frage 3: Was ist wichtig im Umgang mit einem Kunden / Fan in ihrer Branche? zB Umgangston

Rückblick
- Zusammenfassung der Mitschriften
- Danke für die Teilnahme

Ausblick
- Information über Verwendung der Informationen
- Verabschiedung

\subsection{Erhebung durch Experteninterviews}

\subsection{Zusammenfassen der Ergebnisse}
 
Inhaltsanalyse nach Kuckartz

Auf eine Transkription des Interviews wurde hierbei verzichtet.

\subsection{Definition der Anforderungen}

\section{Architekturkonzept}

\section{Prototypische Implementierung}
