\documentclass[12pt,twoside,a4paper,parskip]{scrbook}
\usepackage[utf8]{inputenc}
\usepackage{csquotes}
\usepackage[ngerman]{babel}
\usepackage{floatflt}
\usepackage{subfigure}
\usepackage[pdftex]{graphicx}
\usepackage[hidelinks]{hyperref}
\usepackage{color}
\usepackage{amssymb}
\usepackage{textcomp}
\usepackage{nicefrac}
\usepackage{scrhack}
\usepackage{pdfpages}
\usepackage{float}
\usepackage{pdflscape}
\usepackage{subfigure}
\usepackage{pdfpages}
\usepackage[verbose]{placeins}
\usepackage[markcase=ignoreuppercase,headsepline,plainfootsepline]{scrlayer-scrpage}
\usepackage{listings}
\usepackage{xcolor}
\usepackage{color}
\usepackage{caption}
\usepackage{subfigure}
\usepackage{epstopdf}
\usepackage{longtable}
\usepackage{setspace}
\usepackage{booktabs}
\usepackage[style=apa,backend=biber]{biblatex}
\usepackage{subfiles}
\usepackage{booktabs, multirow} % for borders and merged ranges
\usepackage{soul}% for underlines
%\usepackage[table]{xcolor} % for cell colors
\usepackage{changepage,threeparttable} % for wide tables
\bibliography{referencesmendsync}
 
%%%%%%%%%%%%%%%%%%%
%% definitions
%%%%%%%%%%%%%%%%%%%
\def\BaAuthor{Lukas Bauer}
\def\BaAuthorStudyProgram{Informatik} %% Wirtschaftsinformatik, E-Commerce, Informationssysteme
\def\BaType{Bachelorarbeit} %% Masterarbeit
\def\BaTitle{Gamification auf Twitch als Online-Marketinginstrument für Unternehmen in der Musikbranche}
\def\BaSupervisorOne{Prof.\ Dr.\ Isabel John}
\def\BaSupervisorTwo{Prof.\ Dr.\ Christina Völkl-Wolf}
\def\BaDeadline{\today}

\ifdefined\iswithfullname
  \def\ShowBaAuthor{\BaAuthor}
\else
  \def\ShowBaAuthor{N.~N.}
\fi

\hypersetup{
pdfauthor={\ShowBaAuthor},
pdftitle={\BaTitle},
pdfsubject={Computer Science and Marketing},
pdfkeywords={Gamification;Marketing;Twitch;Livestreaming;Music}
}

%%%%%%%%%%%%%%%%%%%
%% configs to include
%%%%%%%%%%%%%%%%%%%
\colorlet{punct}{red!60!black}
\definecolor{background}{HTML}{EEEEEE}
\definecolor{delim}{RGB}{20,105,176}
\colorlet{numb}{magenta!60!black}

\definecolor{gray}{rgb}{0.4,0.4,0.4}
\definecolor{darkblue}{rgb}{0.0,0.0,0.6}
\definecolor{cyan}{rgb}{0.0,0.6,0.6}

\definecolor{pblue}{rgb}{0.13,0.13,1}
\definecolor{pgreen}{rgb}{0,0.5,0}
\definecolor{pred}{rgb}{0.9,0,0}
\definecolor{pgrey}{rgb}{0.46,0.45,0.48}

\lstset{
  basicstyle=\ttfamily,
  columns=fullflexible,
  showstringspaces=false,
  commentstyle=\color{gray}\upshape
  linewidth=\textwidth
}

\lstdefinelanguage{json}{
    basicstyle=\normalfont\ttfamily,
    numbers=left,
    numberstyle=\scriptsize,
    stepnumber=1,
    numbersep=8pt,
    showstringspaces=false,
    breaklines=true,
    backgroundcolor=\color{background},
    literate=
     *{0}{{{\color{numb}0}}}{1}
      {1}{{{\color{numb}1}}}{1}
      {2}{{{\color{numb}2}}}{1}
      {3}{{{\color{numb}3}}}{1}
      {4}{{{\color{numb}4}}}{1}
      {5}{{{\color{numb}5}}}{1}
      {6}{{{\color{numb}6}}}{1}
      {7}{{{\color{numb}7}}}{1}
      {8}{{{\color{numb}8}}}{1}
      {9}{{{\color{numb}9}}}{1}
      {:}{{{\color{punct}{:}}}}{1}
      {,}{{{\color{punct}{,}}}}{1}
      {\{}{{{\color{delim}{\{}}}}{1}
      {\}}{{{\color{delim}{\}}}}}{1}
      {[}{{{\color{delim}{[}}}}{1}
      {]}{{{\color{delim}{]}}}}{1},
}

\lstset{language=xml,
  morestring=[b]",
  morestring=[s]{>}{<},
  morecomment=[s]{<?}{?>},
  stringstyle=\color{black},
  numbers=left,
  numberstyle=\scriptsize,
  stepnumber=1,
  numbersep=8pt,
  identifierstyle=\color{darkblue},
  keywordstyle=\color{cyan},
  backgroundcolor=\color{background},
  morekeywords={xmlns,version,type}% list your attributes here
}

\lstset{language=Java,
  showspaces=false,
  showtabs=false,
  tabsize=4,
  breaklines=true,
  keepspaces=true,
  numbers=left,
  numberstyle=\scriptsize,
  stepnumber=1,
  numbersep=8pt,
  showstringspaces=false,
  breakatwhitespace=true,
  commentstyle=\color{pgreen},
  keywordstyle=\color{pblue},
  stringstyle=\color{pred},
  basicstyle=\ttfamily,
  backgroundcolor=\color{background},
%  moredelim=[il][\textcolor{pgrey}]{$$},
%  moredelim=[is][\textcolor{pgrey}]{\%\%}{\%\%}
}

\newcommand*{\forcetwosidetitle}[1][1]{%
 \begingroup
   \cleardoubleoddpage
   \KOMAoptions{titlepage=true}% useful e.g. for scrartcl
   \csname @twosidetrue\endcsname
   \maketitle[{#1}]
 \endgroup
}

\newcommand*{\inlineimg}[1]{%
    \raisebox{-.3\baselineskip}{%
        \includegraphics[
        height=\baselineskip,
        width=\baselineskip,
        keepaspectratio,
        ]{#1}%
    }%
}

\DeclareLabeldate{%
  \field{year}
  \literal{nodate}
}

\begin{document}
\frontmatter
\titlehead{%  {\centering Seitenkopf}
  {Hochschule für angewandte Wissenschaften Würzburg-Schweinfurt\\
   Fakultät Informatik und Wirtschaftsinformatik}}
\subject{\BaType}
\title{\BaTitle\\[15mm]}
\subtitle{\normalsize{vorgelegt an der Hochschule f\"{u}r angewandte Wissenschaften W\"{u}rzburg-Schweinfurt in der Fakult\"{a}t Informatik und Wirtschaftsinformatik zum Abschluss eines Studiums im Studiengang \BaAuthorStudyProgram}}
\author{\ShowBaAuthor}
\date{\normalsize{Eingereicht am: \BaDeadline}}
\publishers{
  \normalsize{Erstpr\"{u}fer: \BaSupervisorOne}\\
  \normalsize{Zweitpr\"{u}fer: \BaSupervisorTwo}\\
}
\lowertitleback{
\centering\includegraphics[width=4cm]{qrcode-thesis}
}
\forcetwosidetitle


\section*{Zusammenfassung}

TODO

\section*{Abstract}

TODO

\newpage
\chapter*{Danksagung}

Ich Danke allen die mich bei dieser Arbeit unterstützt haben.

\tableofcontents

\mainmatter
%
%
%
%
%
%
%
%
%
%
%
\chapter{Einleitung}
\section{Ausgangssituation und Problemstellung}
Werbung ist heutzutage im Internet allgegenwärtig. Sie wird immer moderner und raffinierter. In den frühen 90ern als die Marketingbranche begonnen hat zu flukturieren, waren besonders sachliche Werbungen welche die Besonderheiten eines Produktes herausgestellt haben erfolgreich. Je weiter die Zeit vorrangeschritten ist entwickelte sich die Art und Weise mehr von Push zu Pull. Hierbei geht es darum, dass der Verkäufer nicht in erster Linie den Käufer von einem Produkt überzeugen will. Vielmehr geht es darum dem Kunden mit dem Produkt ein gutes Gefühl zu vermitteln um ihn dazu zu bringen es zu wollen. Die Werbebotschaften werden hier viel subtiler und unterschwelliger vermittelt. \parencite{Kopp2013} 

Diese subtile Art der Werbung entwickelte sich besonders rasant mit dem populär werden des Internets. Vor allem im Zuge von Social Media wurde  Content Marketing, Influencer Marketing und Product Placements sehr populär.

Auch Unternehmen in der Musikbranche nutzen mehr und mehr Social Media Plattformen um ihre Produkte und Dienstleistungen an den Konsumenten zu bringen. Hierdurch kann mit geringerem analogen Aufwand, die Kraft der digitalen Werbung effektiv genutzt wird.

Gerade die Corona Pandemie im Jahr 2020 hat gezeigt, wie wichtig es ist als Unternehmen in der Musik und Veranstaltungsbranche online gut aufgestellt zu sein. Da keine Veranstaltungen stattfinden dürfen bricht somit 50\% des Umsatzes einfach weg. Ganz besonders die Verkaufszahlen von Tickets für Live Präsenz Veranstaltungen sind hiervon stark betroffen. \parencite{Stefan2020} % ausbauen, welche bereiche sind nicht betroffen

\begin{figure}[ht]
\caption{Twitch durchschnittliche gleichzeitige Zuschauer Q2'18 bis Q2'20}
\caption*{Quelle: https://blog.streamlabs.com/streamlabs-stream-hatchet-q2-2020-live-streaming-industry-report-44298e0d15bc}
\centering
\includegraphics[width=0.8\textwidth]{twitch_viewers_2020}
\end{figure}

Heutzutage haben wir eine Vielzahl an online Video Content. Netflix, Amazon Prime, YouTube, Hulu und so weiter. Mit 814 Mio. Stunden Live-Content gestreamt ist Twitch gegenüber YouTube auf welcher nur 226 Mio. Stunden Live-Content gestreamt wurde ein echter Geheimtipp für Marketer \parencite{Sturm2019} Allerdings geht dieser Prozess nur schleppend voran und viel Potential wird nicht genügend genutzt. Es fehlt eine Marketingstrategie die empirisch belegbar einen Erfolg bringt.

\section{Forschungsziel und -methode}
Hier kommt Gamification ins Spiel. Durch Gamification werden die Betrachter auf emotionaler Ebene angesprochen. Sie interagieren mit der Werbung wodurch diese stärker in das System eingebunden werden. Eine Social Media Plattform welche Gamification bereits erfolgreich einsetzt ist die oben erwähnte soziale Live-Streaming Plattform Twitch. Jedoch noch nicht aktiv für Marketing.

In Gamification birgt sich viel Potential da der Benutzer Spaß dabei hat mit der Werbung zu interagieren. In dieser Arbeit soll daher folgende Frage analysiert werden: Wie können die Gamification Elemente der Plattform Twitch von Firmen in der Musikbranche für Marketingzwecke eingesetzt werden?

Gamification wird als ein starkes Tool für das Marketing angesehen. \parencite{Zichermann2011}

Requirements Engineering

Experteninterview

Twitch hat eine sehr gut dokumentierte Programmierschnittstelle (API) mit welcher es möglich ist auch Erweiterungen und Apps zu schreiben die auf Gamification Events von Twitch reagieren und diese beeinflussen können.

Da Gamification als Werkzeug dient, die Motivation und das Interesse zu fördern, kann auch Werbung davon profitieren. Im Fokus dieser Bachelorarbeit stehen Unternehmen in der Musikbranche, welche mithilfe der Gamification die Betrachter auf emotionaler Ebene erreichen sollen. Dies hat den Vorteil dass dem Kunden oft gar nicht bewusst ist, dass es sich hierbei um Werbung handelt.

\section{Aufbau der Arbeit}
In dieser Arbeit soll überprüft werden ob die Gamification Elemente der Plattform Twitch auch als Marketing Instrumente gebraucht werden können.

Zu Beginn der Arbeit werden die Grundlagen erklärt. Dazu zählen eine Einführung in die Plattform Twitch, eine Gamification übersicht dieser als auch einen Einblick in das Marketing auf der Platform.

Für die Phase der Durchführung wurde im Prozess der Requirements Engineerings auf ein Experteninterview gesetzt mit welchem die Anforderungen ermittelt wurden. Mit 

Mit den Ergebnissen wird das Ziel der werbekampagne definiert.

Um diese Ziele anzugehen wurde dann eine technisch gestützte Kampagne prototypisch implementiert. Techn. Gestütz bedeutet in diesem Fall der Einsatz der Gamification Elemente und wie diese getriggert werden etc.

Um am Ende meinen Prototypen zu evaluieren wurde mit den Experten vom Anfang nochmal ein Interview machen und meinen Prototypen vorstellen. In dem Interview würde ich sie dann fragen ob sie Anhand ihrer Erfahrung sagen können ob das funktionieren würde oder nicht. Zusätzlich auch eventuelle Verbesserungen vorschlagen.
%
%
%
%
%
%
%
%
%
%
% 
\chapter{Grundlagen}
\section{Twitch}
\label{chap:grund_twitch}

Twitch ist eine Live-Web-Video-Plattform, die von Amazon betrieben wird. Im März 2020 verzeichnete das Portal durchschnittlich 1,44 Millionen Zuschauer gleichzeitig. Im selben Zeitraum streamten im Durchschnitt 56.000 Streamer die konsumierten Inhalte. Im April 2020 erreichte Twitch Platz 33 im Vergleich mit allen anderen Webseiten in puncto Daten Traffic. \parencite{Iqbal2020}

Twitch entstand als Nebenprojekt aus der Plattform Justin.tv, die im Jahr 2007 gegründet wurde. Justin.tv bot jedem Menschen mit Internetzugang die Möglichkeit kostenlos eigene Liveshows ins Internet zu streamen, sogenannte Broadcasts. Aufgrund der hohen Anfrage von Videospiel-Content wurde eine zweite Seite, dediziert für Videospiele, gestartet. Sie ging unter dem heute bekannten Namen Twitch im Jahr 2011 ans Netz. Später wurde auch der Name der Firma in Twitch umbenannt und Justin.tv wurde vom Netz genommen, da sich die Firma nur auf Twitch konzentrieren wollte. Im Jahr 2014 wurde Twitch von Amazon gekauft.

Auf Twitch gibt es die sogenannten Channels (dt. Kanäle). Jeder Twitch-User hat einen eigenen Channel. Findet ein User den Channel eines anderen Users interessant und möchte benachrichtigt werden, sobald dieser Live geht, so kann er diesem folgen. Jeder Channel hat eine eigene Seite und Internetadresse, auf der der Channel-Betreiber in der \glqq About\grqq{}-Sektion Texte und Bilder einbauen kann. Diese Sektion wird in der Regel beispielsweise für Infos über den Streamer als Person,  Infos und Links seiner Werbepartner oder Links zu seinen anderen Auftritten im Internet genutzt. Ist der Streamer gerade live wird der Video-Feed des Livestreams wie in \autoref{img:twitch_channel} groß und zentral auf dem Channel angezeigt. Auf der linken Seite im Bild ist eine Liste mit Channels zu sehen denen der User folgt. Hierbei werden Channels, die gerade live sind, ganz oben in der Liste, zusammen mit ihren aktuellen Zuschauerzahlen, angezeigt.

Die Besonderheit von Twitch ist die Kombination aus High-Fidelity-Video-Content und textbasierter Kommunikation über einen Chat \parencite{Hamilton2014}. Hierbei haben Zuschauer in einem Livestream die Möglichkeit in Echtzeit untereinander und mit dem Streamer zu kommunizieren. Der Chat wird, wie in \autoref{img:twitch_channel} zu sehen, rechts neben dem Video-Feed des Livestreams angezeigt. Die wohl charakteristischsten Elemente im Chat sind die sogenannten Emotes. Dies sind kleine Bildzeichen, die im Live-Chat eingesetzt werden können, um mehr Emotionen im Text auszudrücken. Sie sind Kernbestandteil der Sprache auf Twitch. Die sogenannten Global-Emotes sind für jeden User verfügbar. Es gibt aber auch Emotes die man erst freischalten muss. Das beliebteste Global-Emote ist \glqq Kappa\grqq{} (\inlineimg{kappa}). Es zeigt die schwarz-weiß Abbildung des Gesichts eines ehemaligen Twitch Entwicklers. \parencite{Barbieri2018}

\begin{figure}[ht]
\caption{Twitch Layout \parencite{Twitchf}}
\label{img:twitch_channel}
\centering
\includegraphics[width=\textwidth]{twitch_live_screen}
\end{figure}

Jeder Livestream wird vom Streamer einer Kategorie zugeordnet. Sie spiegelt dessen Inhalt wider. Diese erleichtern das Auffinden von für den User interessanten Inhalten. So werden auf der Startseite dem User Live-Channels angezeigt, die gerade Kategorien streamen, die den User interessierten könnten. Zudem bietet Twitch über die \glqq Browse\grqq{}-Seite die Möglichkeit spezifisch nach Kategorien zu filtern. Der größte Teil der Kategorie sind in der Regel Videospiele. Die laut Statista beliebtesten Spiele aller Zeiten auf Twitch können der \autoref{tab:twitchStat2020} entnommen werden. So ist League of Legends, mit 33,25 Milliarden Aufrufen, das beliebteste Spiel der Plattform. \parencite{Statista2020}

\begin{table}[!htp]\centering
\caption{Statista 2020}\label{tab:twitchStat2020}
\scriptsize
\begin{tabular}{lrr}\toprule
\multicolumn{2}{c}{\textbf{Twitch most popular games by all time viewers 2020}} \\\midrule
\multicolumn{2}{c}{Most popular games on Twitch worldwide as of September 2020, by all time views (in billions)} \\
& \\
League of Legends &33,26 \\
Fortnite &19,33 \\
Counter-Strike: Global Offensive &15,1 \\
DOTA 2 &14,4 \\
Hearthstone &11,2 \\
Grand Theft Auto V &9,02 \\
World of Warcraft &7,46 \\
Overwatch &6,78 \\
Minecraft &4,16 \\
Tom Clancy's Rainbow Six: Siege &2,67 \\
\bottomrule
\end{tabular}
\end{table}

Wie bereits erwähnt, wurde zu Beginn von Twitch lediglich Gaming Content gestreamt. Seit der Übernahme von Amazon ist ein deutlicher Trend zu erkennen, dass auch andere Kategorien eingeführt und beworben werden. Unter anderem Just Chatting, bei dem der Streamer sich mit den Zuschauern über alle möglichen Themen unterhält, gemeinsam Videos auf YouTube anschaut oder auch andere Streamer bewertet. Die Bandbreite reicht bis hin zu Kochen, Musik oder auch Do-It-Yourself. \parencite{Alexander2018}

Die Plattform ist für beide Seiten kostenlos, sowohl für die Seite des Zuschauers als auch für die des Streamers. Für das Konsumieren der Inhalte der Plattform benötigt man keinen Account. Möchte man aber in einer Chat-Diskussion teilnehmen, einem Channel folgen oder selbst streamen muss man sich einen Account erstellen und mit diesem angemeldet sein.

Das Streamen wird von vielen Streamer zu Beginn nur als Hobby angesehen. Damit das Streamen auch finanziell lukrativ wird hat Twitch die Möglichkeit eingebaut, dass Zuschauer einen Streamer auch finanziell unterstützen können. Hierfür müssen Streamer den sogenannten \glqq Affiliate\grqq{}- oder den \glqq Partner\grqq{}-Status erreichen. Diese Status ermöglichen dem Streamer die sogenannten Subscriptions (dt. Abonnements) und Bits zu aktivieren. 

Die Subscriptions sind eine Möglichkeit einen Streamer auf monatlicher Basis finanziell zu unterstützen. Es gibt drei Stufen, die sogenannten "Tiers". Die Preise der jeweiligen Tiers können aus \autoref{img:tierPreise} entnommen werden. Der Streamer erhält hiervon als Affiliate jeweils 50\%, die andere Hälfte behält Twitch. Als Partner ist die Teilung bei 50/50, 60/40 und 70/30 für Tier 1, 2 und 3. Durch die Zugehörigkeit von Twitch zu Amazon haben Amazon Prime-Mitglieder Vorteile. Hat ein Amazon Prime-Mitglied seinen Accounts mit Twitch verknüpft, so hat er die Möglichkeit monatlich eine Tier-1-Subscription ohne Mehrkosten bei einem Streamer seiner Wahl abzuschließen \parencite{Walter2020}. Zudem erhält der User ein sogenanntes Chat-Badge. Dies ist ein kleines Symbol, das im Chat links neben seinem Namen angezeigt wird. Im Fall von Prime-Gaming ist es eine blau weiße Krone.

\begin{figure}[ht]
\caption{Twitch Subscibtions \parencite{Twitchd}}
\label{img:tierPreise}
\centering
\includegraphics[width=0.7\textwidth]{twitch_sub}
\end{figure}

Nach dem Kauf einer Subscription ist diese, zusammen mit ihren Vorteilen, für einen Monat aktiv. Durch das Subscriben kann der Stream ohne Werbeunterbrechungen geschaut werden. Zudem werden die sogenannten Channel-Emotes freigeschalten. Diese kann der Streamer selbst erstellen um sie individuell an seinen Channel anzupassen. Jeder Subscriber schaltet ein Channel-Badge frei. Hat dieser schon mehrmals subscribet so kann sich das Badge je nach Channel und Dauer verändern. Hierdurch wird im Chat direkt sichtbar wer den Streamer unterstützt. Diese Badges sind ebenfalls in jedem Channel individuell angepasst.

Des weiteren kann ein Streamer im Falle eines sehr aktiven und somit unübersichtlichen Chat den \glqq Follower-Only\grqq{} oder \glqq Subscriber-Only\grqq{} Modus aktivieren. In diesen Modi ist es nur Followern bzw. Subscribern möglich zu schreiben. Somit ist ein weiterer Vorteil des Subscribens auch in diesem Modus in den Chat schreiben zu können.

Ein User kann nicht nur sich selbst eine Subscription kaufen sondern auch anderen. Dieser Vorgang wird auch \glqq Giften\grqq{} genannt und macht aus dem Schenkenden einen \glqq Gifter\grqq{}. Beim Schenken besteht die Möglichkeit einer bestimmten Person eine Subscribtion oder eine bestimmte Anzahl an Subscribtions an die Community zu vergeben. Bei letzterem entscheidet Twitch zufällig wer diese Geschenke erhält.

Die bereits vorher erwähnten Bits sind eine digitale Währung der Plattform, die für Echtgeld erworben werden können. Sie können für verschiedene Dinge eingesetzt werden, unter anderem für das sogenannte \glqq Cheeren\grqq{}. Dies kann als \glqq Anfeuern\grqq{} ins Deutsche übersetzt werden und bedeutet, dass eine Nachricht in den Chat geschrieben und mit animierten Cheer-Emotes versehen wird. Den Mindestbetrag für ein Cheer kann der Streamer selbst bestimmen und startet bei einem Bit. Die Preise der Bits können der \autoref{tab:bitsPreise} entnommen werden. Ein Bit entspricht einem US-Cent, den der Streamer erhält. In der Tabelle wird deutlich, dass sich Twitch beim Kauf der Bits einen bestimmten Prozentsatz einbehält.

\begin{table}[!htp]\centering
\caption{Bits Preise Stand 18.10.2020}\label{tab:bitsPreise}
\scriptsize
\begin{tabular}{lrr}\toprule
\textbf{Anzahl der Bits} &\textbf{Preis in Eur} \\\midrule
100 &1,47 \\
500 &7,34 \\
1500 &20,91 \\
5000 &67,51 \\
10000 &132,08 \\
25000 &322,87 \\
\bottomrule
\end{tabular}
\end{table}

Ein weiteres Mittel, das einen User dazu animieren soll einen Streamer zu unterstützen, ist der sogenannte \glqq HypeTrain\grqq{}. Dieser ist ein Event, welches automatisch eintritt, sobald innerhalb von fünf Minuten eine bestimmte Menge an Bits und Anzahl an Subscriptionkäufe getätigt wurden. Die Menge, um das Event zu triggern, kann der Streamer in den Einstellungen selbst festlegen. Dies hat den Vorteil, dass jeder Streamer anhand seiner Zuschauerzahl selbst bestimmen kann, wann und wie oft der HypeTrain aktiviert werden soll. Passiert dies allerdings zu oft, verliert der HypeTrain seine \glqq Besonderheit\grqq{} und wird schnell als störend empfunden. Der Streamer muss ihn deshalb mit Bedacht einsetzen.

Bei Aktivierung des HypeTrains wird oberhalb des Chats ein Fortschrittsbalken mit Prozentzahl und Countdown angezeigt. Das Ziel der Zuschauer ist es nun diesen Balken mit Hilfe von Subscribtions und Cheers zu füllen. Der HypeTrain ist aufgeteilt in aufeinanderfolgende Level. Es können bis zu fünf Level aktiviert werden. Jedes Level muss innerhalb von jeweils fünf Minuten abgeschlossen werden, ansonsten bricht der HypeTrain ab, sobald der Countdown bei Null angekommen ist. Der Streamer kann den Schwierigkeitsgrad des HypeTrains bestimmen. Das bedeutet in diesem Fall, wie viele Bits und Subscribtions benötigt werden damit ein Level abgeschlossen wird. Bei erfolgreichem Abschluss des HypeTrains erhalten die Teilnehmer als  Belohnung je nach Level verschiedene Emotes.


\begin{figure}[ht]
\caption{Hypetrain \parencite{Twitche}}
\centering
\includegraphics[width=0.7\textwidth]{twitch_hypetrain}
\end{figure}

Ein weiteres Event ist der sogenannte \glqq Raid\grqq{}. Der Begriff ist in Online-Rollenspielen entstanden. Man spricht von einem Raid, wenn eine Gruppe von Spielern gemeinsam eine andere Gruppe von Spielern oder einen Endgegner bekämpfen, für den sie alleine zu schwach wären. Auf Twitch ist ein Raid wenn der Streamer am Ende des Streams seine aktuellen Zuschauer auf einen andern Twitch-Channel übergibt. Als Zuschauer hat man nun das Gefühl an einem gemeinsamen Event teilzunehmen. Hierbei wird das Gemeinschaftsgefühl des vorherigen Streams zusätzlich gestärkt. \parencite{Stephen2020}

\section{Gamification}
\label{chap:gamif}
Spiele erfreuen sich seit vielen Jahren großer Beliebtheit. Seien es Brettspiele, sportliche Wettbewerbe oder seit der Entwicklung des Computers auch Videospiele. Aufgrund des großen Erfolges von Spielen wurde begonnen, deren Hintergründe zu erforschen. Das Ziel ist es herauszufinden, weshalb sie so beliebt sind und was sie so faszinierend und fesselnd macht. \parencite{Prensky2001}

Bei diesen Untersuchungen wurde herausgefunden, dass es bestimmte Elemente in Spielen gibt, die besonders stark zur Motivation und Faszination eines Spielers beitragen. Aus dieser Erkenntnis ist die Forschung der Gamification entstanden. Gamification ist ein Begriff, der bereits seit einigen Jahren in vielen Bereichen Anwendung findet. 

Nachdem lange Verwirrung um eine genaue Definition von Gamification bestand, definierte es \cite{Deterding2011} wie folgt: \glqq Gamification is the use of game design elements in non-game contexts\grqq{}. Somit ist Gamification die Anwendung von Elementen aus Spielen in \glqq Nicht-Spiel-Umgebungen\grqq{}. Die eingangs erwähnten Elemente sind unter anderem Punkte, Badges oder Leaderboards, im folgenden kurz \glqq PBLs\grqq{}. \parencite{Deterding2011}

Es wird fälschlicherweise oft angenommen, dass das alleinige Hinzufügen von Punkten bereits Gamification ist. Dabei entsteht eine Verwechslung mit Pointification. Dies ist das reine Anwenden von PBLs auf ein System ohne tieferes Konzept, in der Hoffnung, dass es eine positive Auswirkung auf das Verhalten der Benutzer hat. So werden hierbei für das Durchführen bestimmter Aufgaben Punkte vergeben, die Benutzer anhand ihres Punktestandes in einem Leaderboard verglichen und nach dem Erreichen einer bestimmten Punktzahl ein Badge oder Zertifikat vergeben. \parencite{Kifetew2017}

Gamification ist viel mehr. Es ist das gezielte Ansprechen von Motivationsfaktoren in der menschlichen Psyche. Deshalb wird bei Gamification ein besonderes Augenmerk auf den User, dessen Verhalten, Bedürfnisse, Wünsche und Emotionen gelegt. Dieses Vorgehen wird "Human-Focused Design" genannt, im Folgenden kurz HFD. Im Gegensatz dazu wird beim klassischen "Function-Focused Design" das Augenmerk auf die durchzuführende Arbeit und die hierfür möglichst effiziente funktionale Lösung gelegt. \parencite{Mechelen2017}

\subsection{Spielertypen}

Um das HFD erfolgreich einsetzen zu können muss sich bei der Entwicklung der Gamification Strategie das Entwicklungsteam der Zielgruppe bewusst sein. Zur Erleichterung kann in Gamification auf das Konzept von Spielertypen zurückgegriffen werden. Diese beschreiben die verschiedenen Verhaltensmuster von Menschen und ordnen sie bestimmten Kategorien bzw. Typen zu.

Hierbei können verschiedenen Modelle verwendet werden, die in verschiedenen Bereichen Anwendung finden. Beispielsweise gibt es die Spielertypen nach Bartels, die in vier verschiedene Player aufgeteilt werden. Jedoch ist dieser Ansatz für Rollenspiele entwickelt worden und deshalb hier nicht anwendbar. \parencite{Bartle1996} Besser sind die Spielertypen nach Marczewski, bestehend aus 6 verschiedenen Arten von Usern. \parencite{Marczewski2015}

Die verschiedenen Spielertypen nach Marczewski können der \autoref{img:hexad} entnommen werden. Der äußere, grüne Bereich, stellt die Spielertypen dar, wobei der innere, rote Bereich, die dazugehörigen Motivationsfaktoren des Typen sind. So ist der \glqq Socializer\grqq{} durch die Verbundenheit zu anderen Menschen und der \glqq Free Spirit\grqq{} durch möglichst viel Autonomie und kreativer Freiheit motiviert. Allerdings kann man einen Menschen nicht nur einem Typen zuordnen. Vielmehr hat jeder Mensch auch jeden Typen mit unterschiedlicher Ausprägung in sich. Jeder Typ reagiert auf die verschiedenen PBLs unterschiedlich stark. \parencite{Marczewski2015}

\begin{figure}[ht]
\caption{Gamification User Types Hexad von Marczewski \parencite{Marczewski2016}}
\label{img:hexad}
\centering
\includegraphics[width=0.7\textwidth]{gami_usertypes_hexad}
\end{figure}

\subsection{Motivationsfaktoren}
\label{chap:motfak}

Um auch die Motivationsfaktoren besser zu verstehen wurden hierfür ebenfalls verschiedene Frameworks entwickelt. Eines davon ist das Octalysis Framework von YouKai Chou. Dieses beschreibt die Motivation des Menschen anhand acht verschiedener Teilbereiche. Sie werden in dem Framework auch \glqq Core Drives\grqq{} genannt. Die acht verschiedenen Core Drives können der \autoref{img:core_drives} entnommen werden und werden im Folgenden kurz erklärt.

\begin{figure}[ht]
\caption{Octalysis eight core drives \parencite[p.~23]{Chou2015}}
\label{img:core_drives}
\centering
\includegraphics[width=\textwidth]{gami_oct}
\end{figure}

Der erste Core Drive ist \glqq Epic Meaning \& Calling\grqq{} (dt. Höhere Bedeutung \& Berufung). Hierbei hat der User das Gefühl etwas zu machen, das größer ist als er selbst bzw. dass er zu etwas Größerem einen Beitrag leistet. Ein gutes Beispiel hierfür ist die Seite Wikipedia auf der viele Menschen gemeinsam, freiwillig und unentgeltlich an einem Projekt arbeiten, um das Wissen der Menschheit zu sammeln.

Der zweite Core Drive ist \glqq Development \& Acomplishment\grqq{} (dt. Entwicklung / Fortschritt \& Errungenschaften). Hierbei ist es wichtig einem User eine Herausforderung zu geben. Hat der User diese geschafft, so bekommt er das Gefühl, dass er stolz auf sich sein kann. Das menschliche Gehirn ist darauf ausgelegt belohnt zu werden. Diese Belohnung erhält er in visueller Form, indem ein Fortschritt angezeigt wird oder indirekt in dem er fühlt, dass seine eigene Erfahrung wächst. Dieser Core Drive ist der meist genutzte in der Gamification Industrie, da er sich auch am einfachsten implementieren lässt.

\glqq Empowerment of Creativity \& Feedback\grqq{} (dt. Förderung der Kreativität \& Feedback) ist der dritte Core Drive. Es spricht besondere Eigenschaften des Menschen an, wie zum Beispiel den Drang Neues zu lernen, seinen Horizont zu erweitern oder neue Dinge zu erfinden. Ein Beispiel hierfür ist das Spiel mit den bunten Bausteinen, Lego.

Das vierte Core Drive \glqq Ownership \& Possession\grqq{} (dt. Eigentum \& Besitz) funktioniert nach einem einfachen Prinzip. Hat ein Mensch von etwas Besitz ergriffen, möchte er davon nicht mehr loslassen, sondern vielmehr es verbessern und daraus das Beste herausholen. Hierbei spielen Elemente wie digitale Währungen oder Besitztümer eine große Rolle, da dort eine große Wahrscheinlichkeit besteht, dass der User automatisch versuchen wird einen virtuellen Wohlstand aufzubauen. Auch die Möglichkeit der Individualisierung spielen hier eine große Rolle.

\glqq Social Influence \& Relatedness\grqq{} (dt. sozialer Einfluss \& Verbundenheit) als vierter Core Drive taucht oft im Kontext von Wettbewerben, Neid, Gruppenaufgaben, Allgemeingut und Kameradschaft auf. Dabei wird der Drang des Menschen angesprochen sich mit anderen auszutauschen oder zu messen. Auch bei Dingen zu denen sich ein User emotional verbunden fühlt oder diese mit Kindheitserinnerungen in Verbindung bringt, ist das die treibende Kraft dahinter. 

\glqq Scarcity \& Impatience\grqq{} (dt. Knappheit \& Ungeduld) ist der sechste Core Drive. Dieser baut auf dem natürlichen Drang des Menschen auf, Dinge zu wollen, die unerreichbar scheinen. Auch das Gefühl von Exklusivität spielt hierbei eine große Rolle.

Der CoreDrive sieben, \glqq Unpredictability \& Curiosity\grqq{} (dt. Unvorhersehbarkeit \& Neugierde), ist oft bei Glücksspielen anzutreffen. So ist ein Mensch oft stärker motiviert, wenn es anstelle einer Gewinngarantie nur eine Gewinnwahrscheinlichkeit gibt. Mit dem sicheren Wissen einer Belohnung, ist die Motivation nur so groß wie die Belohnung selbst. Besteht aber nur eine bestimmte Chance, dass er diese Belohung erhält, so ist er durch die Spannung stärker motiviert als durch die Belohnung an sich.

Der achte und letzte \glqq Core Drive, Loss \& Avoidance\grqq{} (dt. Verlust \& Vermeidung), ist durch das Gefühl angeregt etwas Erarbeitetes zu verlieren. Wobei auch das \glqq Sunk Cost\grqq{}-Prinzip eine große Rolle spielt. Dieses Prinzip tritt ein, wenn ein User bereits Bemühungen in die Erreichung eines Ziels investiert hat. Selbst wenn für den Abschluss dieses Ziels noch unrentabel viel Aufwand  zu erbringen ist, werden die bereits investierten Bemühungen als Motivation genommen, trotzdem weiter zu machen. Der User möchte somit das Gefühl vermeiden, die bereits investierten Dinge zu verlieren.

Die acht verschiedenen Core Drives werden in \glqq Left\grqq{}- oder \glqq Right Brain\grqq{} und \glqq White Hat\grqq{} oder \glqq Black Hat\grqq{} Gamification aufgeteilt. Wie in \autoref{img:core_drives_dir} zu sehen, sind hiermit die drei Core Drives auf der linken und die drei auf der rechten Seite bzw. die drei Unteren Core Drives und die drei Oberen gemeint. Die in der Mitte liegenden Core Drives werden für die jeweilige Unterteilung als neutral betrachtet. Die Unterteilung in \glqq Left Brain\grqq{} und \glqq Right Brain\grqq{} kann auch als extrinsisch und intrinsisch bezeichnet werden.

\glqq White Hat\grqq{} Core Drives geben einem User das Gefühl von Stärke, Erfüllung und Befriedigung. Sie geben das Gefühl alles unter Kontrolle zu haben, wohingegen \glqq Black Hat\grqq{} Core Drives genau das Gegenteil bewirken. Diese verbreiten das Gefühl von Angst, Besessenheit und Sucht, wodurch ein User das Gefühl bekommt, die Kontrolle über seine Aktionen verloren zu haben.

\begin{figure}[ht]
\caption{Octalysis Tendenzen \parencite[pp.~29-32]{Chou2015}}
\label{img:core_drives_dir}
\centering
\includegraphics[width=\textwidth]{gamif_oct_left_right_black_white}
\end{figure}

\subsection{Design Framework }

Mit der Kenntnis der Motivationsfaktoren wird nun deren Anwendung näher erläutert. Benötigt wird hierfür ein \glqq Gamification Design Framework\grqq{} (GDF) \parencite{Baldeon2016}. Es gibt beispielsweise das \glqq Gamification Model Canvas\grqq{} (GMC), das in acht Schritte unterteilt ist und das \glqq Octalysis Strategy Dashboads\grqq{} (OSD). In dieser Arbeit wird jedoch nur auf das OSD gesetzt, da dieses ebenfalls von YouKai Chou entwickelt wurde, um auf den acht Core Drives aufzubauen. Zudem ist es auch etwas leichter anzuwenden als das GMC, da hierbei nur sechs Schritte definiert sind.

Das Ziel des OSD ist es herauszufinden, welche Gamification Elemente eingesetzt werden müssen, um einen User zu einem festgelegtem Verhalten zu motivieren. Dabei fokusiert es sich auf die wichtigsten Aspekte und unterteilt diese in einzelne Schritte. Diese Schritte bauen aufeinander auf und sind untereinander abhängig. Das Strategy Dashboard wird in 5 Schritte unterteilt. Diese fünf Schritte können \autoref{img:gami_oct_strat} entnommen werden und werden im Folgenden näher erläutert. 

Zu Beginn werden die genauen Ziele anhand von Business Metriken definiert. Diese Metriken sollten messbare, zählbare oder in einer anderen Weise quantifizierbare Werte sein. Sie sollen sich im Verlauf der Anwendung verbessern und werden im Nachgang mit ihren Ursprungswerten verglichen um die Effektivität der Kampagne zu evaluieren. Bei den Metriken ist ebenfalls darauf zu achten, dass diese nach Priorität sortiert sind um zuerst die wichtigsten Metriken  zu verbessern.

Im nächsten Schritt werden anhand der bereits eingangs erwähnten Spielertypen die User definiert. Hierbei ist darauf zu achten, dass diese durch ihre unterschiedlichen Motivationensfaktoren kategorisiert werden. Anderenfalls besteht die Gefahr, dass Gruppen definiert werden, welche unterschiedlich erscheinen, jedoch auf die gleiche Weise motiviert werden können. Um einen User richtig zu definieren kann auch auf das Konzept der Personas zurückgegriffen werden \parencite{Nielsen2013}. 

Im dritten Schritt werden die \glqq Desired Actions\grqq{} und daraus resultierenden \glqq Win-States\grqq{} definiert. Die \glqq Desired Actions\grqq{} sind kleine Schritte die der User durchführen soll, um die Business Metriken zu verbessern. Solche Schritte können beispielsweise das Besuchen einer Webseite, das Ausfüllen eines Formulars und das anschließende Registrieren, das Anmelden zu einem Newsletter oder das Klicken auf eine Werbeanzeige sein. Der \glqq Win-State\grqq{} beschreibt einen Zustand, in dem die Ziele des Players, als auch die Ziele der Business Metriken erreicht werden.

Der vierte Schritt beschreibt die \glqq Feedback Mechanics\grqq{}. Diese helfen dem User auf dem Weg zum \glqq Win-State\grqq{} seinen aktuellen Fortschritt im Blick zu behalten. Sie sorgen dafür, dass er die \glqq Desired Actions\grqq{} auch durchführt, indem sie ihn motivieren. Dieser Schritt ist der erste, bei dem konkret auf Gamification Elemente eingegangen wird. 

Im fünften und letzten Schritt werden die \glqq Incentives\grqq{} (dt. Belohnungen) definiert, welche im \glqq Win-State\grqq{} enthalten sind. Diese Belohnungen werden dem User nach der Erfüllung der \glqq Desired Actions\grqq{} gegeben, um ihm zu zeigen, dass er eine richtige Aktion durchgeführt hat.


\begin{figure}[ht]
\caption{Octalysis Strategy Dashboard \parencite[p.~467]{Chou2015}}
\label{img:gami_oct_strat}
\centering
\includegraphics[width=\textwidth]{gami_oct_strat_dash}
\end{figure}




%
%
%
%
%
%
%
%
%
%
% 
\chapter{Gamification auf Twitch}

Twitch ist eine Plattform, die ursprünglich speziell für das Live Streamen von Videospielen entwickelt wurde. Somit ist es nicht verwunderlich, dass hier auch aktiv Gamification Anwendung findet. Im folgenden Kapitel werden die Gamification Elemente in drei Kategorien aufgeteilt. Die erste Kategorie ist Nativ in Twitch implementierte Gamification. Diese ist direkt in der Plattform integriert. Die Kategorien zwei und drei sind durch die Verwendung von Third-Party Software hinzugefügte Overlays im Video-Feed des Streams und Chatbots welche in den Twitch Chat eingebunden werden. 

Außerdem werden die einzelnen Gamification Elemente anhand der acht Core Drives aus \autoref{chap:motfak} bewertet. Im Anschluss wird überprüft wie diese Elemente durch die Twitch API oder andere Schnittstellen beeinflusst werden können. Denn nur dann, wenn in den Gamifiaction Prozess der Elemente eingegriffen werden kann, können diese Elemente effektiv für Marketing verwendet werden.

\section{Nativ in Twitch}
Sowohl auf der Seite eines Zuschauers als auch auf der eines Streamers werden Gamification Elemente eingesetzt. Zunächst wird die Seite des Streamers beleuchtet. Dieser wird motiviert weiter zu Streamen um neuen Content für Twitch zu produzieren. Danach wird auf die Seite des Zuschauers eingegangen. Dieser soll motiviert werden den Stream weiter zu konsumieren und den Streamer finanziell zu unterstützen.

\subsection{Achievements \& Meilensteine für Streamer}
Der Weg auf Twitch ein erfolgreicher Streamer zu werden ist lange und mit viel Arbeit verbunden. Damit die Content Creator auf dem Weg dort hin durchgehen motiviert bleiben hat Twitch ein Achievement System in die Plattform eingebaut. Dieses System besteht aus vielen kleinen Aufgaben die es zu erreichen gilt. Solche Aufgaben sind zum Beispiel eine bestimmte Anzahl an Followern zu erreichen, eine Anzahl an Tagen zu Streamen, eine Anzahl an gleichzeitigen Usern im Chat zu haben und vieles mehr.

Die Achievments und deren aktueller Fortschritt können im Creator Dashboards eines Twitch Profils angezeigt werden. Dort ist eine Liste mit allen Achievments mit eingebettetem Fortschrittsbalken an der Unterseite jedes Achievments und die aktuellen Zahlen auf der rechten Seite. Zudem gibt es auch Meilensteine die man erreichen muss um die in \autoref{chap:grund_twitch} beschriebenen Streamer Status Affiliate und Partner zu erreichen. Wie in \autoref{img:milestones} zu sehen gibt es für beide Status den \glqq Path to Affiliate\grqq{} und \glqq Path to Partner\grqq{}. Die Achievements und Meilensteine können beide den Core Drives zwei und vier zugeschrieben werden. Da diese Gamification Elemente aber nur für den Streamer selbst sichtbar sind können sie in einer Marketing Kampagne nicht angewandt werden. Somit werden sie für diese Arbeit als nicht relevant definiert.

\begin{figure}[ht]
\caption{Twitch Achievments \& Meilensteine für Streamer \parencite{Twitchb}}
\label{img:milestones}
\centering
\includegraphics[width=\textwidth]{twitch_gamif_achievements}
\end{figure}

Wie in \autoref{chap:grund_twitch} geschrieben sind einige Twitch Mechaniken auf dem Profil eines Streamers erst mit der Erreichung des Affiliate oder Partner Status freigeschaltet. Zudem kann er mit dem Abschließen einiger Achievements die Quantität von Gamification Elemente erhöhen. So kann beispielsweise durch Erhöhen der Anzahl durchschnittlicher Chatter die Menge an vergebbaren VIP Badges erhöht werden. \parencite{Twitchg}

Twitch ist eine sich sehr schnell entwickelnde Plattform. So testet die Plattform oft neue Elemente, die nach wenigen Monaten wieder entfernt werden, da sie nicht den gewünschten Langzeiteffekt erzielt haben. Ein Beispiel hierfür sind die \glqq Twitch Crates\grqq{} die ein User durch Spielkäufe direkt auf der Plattform erhalten konnte. Da das Kaufen von Spielen jedoch entfernt wurde, entfielen auch die Crates. In diesen Crates waren beispielsweise Badges oder Emotes enthalten die trotzdem heute noch auf der Plattform zu finden sind. Durch fehlende Aktualität und Relevanz werden diese in dieser Arbeit nicht behandelt.

\subsection{Badges}
Wie bereits erwähnt sind Badges kleine Symbole. Sie werden im Chat links neben dem Usernamen angezeigt um dessen Status oder Errungenschaften zu präsentieren. Einzige Ausnahme hierbei ist das \glqq Verified\grqq{} Badged, das neben dem Chat auch auf der Profilseite eines gepartnerten Streamers angezeigt wird. Zur leichteren Orientierung werden die Badges in dieser Arbeit in vier Kategorien aufgeteilt. Rollengebundene Badges, Leaderboard-Badges, Subsciber-Badge und andere Badges.

\begin{figure}[ht]
\caption{Twitch Badges Overview }
\label{img:badges_overview}
\centering
\includegraphics[width=\textwidth]{twitch_badges_overview}
\end{figure}

\textbf{Rollengebundene Badges}

Die in \autoref{img:role_badges} dargestellten Badges sind an die entsprechenden Rollen gebunden. Nimmt eine User eine dieser Rollen ein, erhält er das hierfür vorgesehene Badge. Die Mitarbeiter von Twitch haben das \glqq Twitch Staff\grqq{} oder \glqq Admins\grqq{} Badge. 

Ist man selbst ein Streamer erhält man den Broadcaster Badge. 

Da sich ein Stream leichter verwalten lässt wenn mehrere User mithelfen gibt es die Moderator Rolle. Diese Rolle kann vom Streamer an andere User vergeben werden. Fällt ein User im Chat negativ auf so kann dieser von einem Moderator auf verschiedene Weisen entfernt werden. Ein Moderator erhält das Chat Moderator Badge.


- Staff
- Admin
- Broadcaster
- Mod
- ...

\begin{figure}[ht]
\caption{Rollengebundene Badges}
\label{img:role_badges}
\centering
\includegraphics[width=\textwidth]{x_twitch_badges}
\end{figure}

\textbf{Leaderboard Badges}
\begin{figure}[ht]
\caption{Loaderboard Badges}
\centering
\includegraphics[width=0.6\textwidth]{x_twitch_badges_lead}
\end{figure}

\textbf{Loyalty Badges}
Die sogenannten Loyalty Badges oder Subsciber Badges bekommt ein User welcher eine Subscibtion eines Streamers kauft. Da jeder Streamer einen anderen Stil hat kann auch das Aussehen der Badges vom Streamer selbst definiert werden. So hat ein Streamer der eher Militärsimulationen spielt auch Militärische Abzeichen als Badges. Ein Musikstreamer eher etwas Musikrelevantes wie Schallplatten, Noten, etc.

Das erzeugt ein starkes Gemeinschafts bzw. Community Gefühl bei allen Badge-Besitzern. 

Oder auch besondere insight: wenn der Streamer zum Beispiel großer Fan von Pizza ist und alle seine Zuschauer das kennen können auch die Badges in Form einer Pizza sein.

Hat ein User schon mehrere Monate in Folge eine Subscription abgeschlossen bekommt dieser neues Badges. Diese sollen wie ein Abzeichen den Rang eines Unterstützers darstellen.  Hat man eine Abo eines höheren Tiers so wird dem Badge noch ein zusätzlicher Flair verliehen. Ein Flair ist ein zusätzliches Symbol oder Grafik welche auf das Badge gelegt wird. Tier 2 und Tier 3 haben in der Regel unterschiedliche Flairs.

\begin{figure}[ht]
\caption{Loyalty Badges}
\centering
\includegraphics[width=0.4\textwidth]{twitch_eskei_badges}
\end{figure}

Ein besonderes Badge bekommt ein User der unter den ersten zehn ist welche den Channel subsciben. Das sogenannte Founde Badge.


\textbf{andere Badges}
Die TwitchCon ist eine Messe die von Twitch einmal im Jahr abgehalten. Jeder Teilnehmer der Messe der beim Ticket kauf seinen Twitch Namen angegeben hat erhält daraufhin ein Badge für den Chat der ihn nun als Messe Teilnehmer auszeichnet.

Hype Train Conductor

\begin{figure}[ht]
\caption{Hypetrain Conductor}
\centering
\includegraphics[width=0.4\textwidth]{twitch_hypetrain_conductor}
\end{figure}


Zudem macht Twitch hin und wieder Sonderaktion wie Beispielsweise ein Charity Event. Dieses war unter dem Titel "Holiday Season of Giving" im Jahr 2018 vom 12. - 27. Dezember aktiv. Pro 100 Bits hat Twitch in diesem Zeitraum 20ct an Wohltätige Zwecke gespendet. Jeder der in diesem Zeitraum mindestens 100 Bits mit dem Hashtag \#Charity gespendet hat, hat ein Schneflocken Badge bekommen. \parencite{Twitch2018a}


\subsection{Leaderboard}
Oberhalb des Chats befindet sich eine Liste mit den \glqq Top 10 Sub Gifter\grqq{} sowie den \glqq Top 10 Cheerers\grqq{}. Diese Liste wird auch Leaderboard genannt. Der Streamer kann bestimmen ob die Top 10 des Tages, der Woche, des Monats oder die Top 10 insgesamt angezeigt werden sollen.

Das Leaderboard ist ein einfaches Mittel von Twitch den User dazu zu bewegen den Streamer weiter zu unterstüzden. Durch Verschenken von Subs und das Cheeren kann er sich in der Rangliste nach oben Kämpfen.
Zudem müssen Sie den Streamer weiter unterstüzuen um nicht von anderen überholt zu werden. 

das Leaderboard ist besonders für Philantrophen

Core Drive 2, 5 und 8


\begin{figure}[ht]
\caption{Leaderboards}
\centering
\includegraphics[width=0.5\textwidth]{Twitch_Leaderboards}
\end{figure}


\subsection{Punkte}
\label{chap:tw_gami_punkte}
Twitch hat ein Punkte System in die Streams implementiert um einen User länger in einem Stream zu halten. Ein User sammelt Punkte unter anderem durch Zusehen des Streams oder der Teilnehmen an einem Raid. So erhält ein User wie in \autoref{img:chan_pts} zu sehen alle 5 Minuten 10 Punkte und durch Teilnahme an einem Raid 250 Punkte. Ist ein User Subscriber so multiplizieren sich die Punkte jeweils um eine definierten Faktor.

\begin{figure}[ht]
\caption{Channelpoints verdienen}
\label{img:chan_pts}
\centering
\includegraphics[width=0.6\textwidth]{twitch_channelpoints_earning}
\end{figure}

Diese gesammelten Punkte können eingesetzt werden um bestimmte Aktionen im Stream auszulösen. Hierbei gibt es einige vorgefertigte Aktionen von Twitch wie zum Beispiel das Hervorheben einer Nachricht im Chat oder das freischalten eines Subscriber Emotes. Der Streamer selbst kann aber auch noch andere Aktionen einbauen. Sei es dass er zum Beispiel einen Schluck Wasser trinken muss oder seine Mütze absetzt. Der Streamer kann die Anzahl der für die Aktionen nötigen Punkte selbst festlegen.

Mit spezieller Software kann man auch das Streaming Programm des Streamers beeinflussen und so zum Beispiel Szenen wechseln, um andere Kamera Perspektiven zu sehen. Besonders engagierte Streamer bieten auch die Möglichkeit, Sounds im Stream abzuspielen oder physikalische Elemente im Raum des Streamer zu beeinflussen, wie zum Beispiel Lichter im Hintergrund.

Streamer definieren auch oft Aktionen mit sehr vielen Punkten um einen Zuschauer auf diese Aktion sparen zu lassen.

Core Drive 2, 3 und 7

\subsection{Währung}

Mit den Bits hat Twitch eine eigene \glqq in-Game-Währung\grqq{} etabliert und damit ein eigenes Wirtschaftssystem in ihre Plattform eingebaut. Aus echtem Geld wie Euro oder Dollar werden Bits. Diese sind nach dem Transfer physisch nicht mehr greifbar und nur noch als Zahl einer Fremdwährung erkennbar. Somit ist die Transparenz eingeschränkt. Punkte werden leichter ausgegeben als Echtgeld da ein User oft selbst vergisst wie viel Geld er umgerechnet ausgegeben hat. \parencite{Madigan2019}

Core Drive 4

\subsection{Progressbar}
\textbf{Hype Train}

Countdown

Core Drive 2, 5, 6, 7 und 8


\subsection{Items und Unlockables}

Emotes
- Man gehört zur community und kann die emotes im chat verwenden (wenn der Streamer sagt spammt emote X in den Chat)
- man kann die emotes in anderen channels verwenden und zeichnet sich somit als supporter des anderen channels aus und macht werbung 

Core Drive 2, 4, 5, 6 und 8


Drops sind ein recht neuer Ansatz von Twitch die Zuschauer für längere Zeit in einem Stream zu halten. Hat ein Streamer Drops aktiviert und spielt ein Spiel welches Drops unterstützt so kann ein Zuschauer nur durch das zusehen des Streams ab einer gewissen Zeit Items für das Gestreamte Spiel freischalten.

Core Drive 4 und 5

\subsection{Achievements für Zuschauer}
hat ein Zuschauer eine subscibtion gekauft so wird dies im Chat besonders hervorgehoben. Der Zuschauer hat auch die möglichkeit eine Nachricht zu schreiben die innerhalb dieses hervorgehobenen bereiches angezeigt wird.

Der Streamer sieht dann auch diese Events in seinem Streamer Dashboard und kann im Stream darauf reagieren.

Core Drive 2 und 5

\section{Stream Overlays}
Overlays sind seit Anbeginn von Twitch ein Kernbestandteil von Streams. Um Overlays erklären zu können muss zunächst erklärt werden wie ein Stream produziert wird. Twitch ist eine sogenannte Crowdsourced Content Plattform. Hierbei geht es darum, dass nicht Twitch selbst den Content für ihre Internetseite produziert sondern jeder die Möglichkeit hat einen Stream zu starten. Zum Streamen benötigt man einen Computer auf dem eine spezielle Streaming Software installiert ist. Die bekannteste ist Open Broadcaster Software, im folgenden kurz OBS. \parencite{Zhang2015}

Auch die modernen Videospielkonsolen Playstation 4 und Xbox One bieten mit integrierter Software die Möglichkeit zu Streamen. Diese werden jedoch in dieser Arbeit als nicht relevant definiert.

OBS bietet die Möglichkeit in Echtzeit, Video und Audio Material aus verschiedenen Quellen zu verschmelzen. Diese Quellen können zum Beispiel das Signal von Kameras und Mikrofonen sein. Auch virtuelle Quellen wie Bildschirmaufnahmen, Audio-, Bild- \& Videodateien und Texte können eingebunden werden. Die Quellen werden wie in \autoref{img:obs_overlays} Bildlich dargestellt, wie verschiedene Layer übereinandergelegt und ergeben ein gemeinsames Bild. Das entstandene Bild wird von OBS in ein vorgegebenes Format kodiert. Das fertige Material kann entweder als Video Datei auf der Festplatte abgespeichert werden oder über das Real Time Messaging Protocol, kurz RTMP, an einen Online Service wie Twitch geschickt werden. \parencite{Alam2018}

\begin{figure}[ht]
\caption{Overlay Layers (eigene Darstellung)}
\label{img:obs_overlays}
\centering
\includegraphics[width=\textwidth]{gami_overlay_layers}
\end{figure}

Overlays sind Ebenen welche über das Video Bild des Streams gelegt werden. Daher auch der Name Overlay, vom deutschen \glqq Überlagern\grqq{}. Ein Overlay kann ein Bild sein, das zum Beispiel als Rahmen für die Webcam des Streamers benutzt wird. Auch dynamische Inhalte wie Webseiten können eingebunden werden. Ein großer Vorteil dieser Webseiten ist deren Fähigkeit sich über einen WebSocket mit der Twitch API zu verbinden. Eine WebSocket Verbindung ist ein Kommunikations Kanal zwischen zwei Parteien. Beide Parteien können Pakete senden und empfangen. Somit ist es eine zwei-wege Kommunikationsverbindung. \parencite{Mozilla}  Diese Parteien sind in diesem Fall die Webseite und die Twitch API. Über diesen Kanal kann die Webseite Event-basierte Nachrichten erhalten ohne eine neue Anfrage hierfür zu senden. Ein solches Event kann zum Beispiel das Folgen, Subscriben oder Cheeren eines Users sein.

Beliebte Formen auf diese Events zu reagieren sind Progressbars und Alerts. Progressbars werden dafür verwendet, bestimmte Ziele für die Zuschauer zu definieren. Dies können beispielsweise eine bestimme Anzahl an Followern, Subs, Bits oder direkte Spenden sein. Wie in \autoref{img:goal_overlay} zu sehen wurde als Ziel ein Geldbetrag von 300\$ für einen neuen Gaiming Stuhl definiert. Die Zuschauer haben bereits zusammen 245\$ erreicht. Spendet nun ein weiterer User wird der Fortschittsbalken um den gespendeten Betrag erhöht. 

\begin{figure}[ht]
\caption{Goal Overlay \parencite{Streamlabsa}}
\label{img:goal_overlay}
\centering
\includegraphics[width=\textwidth]{twitch_goals}
\end{figure}

Alerts sind die effektivste Möglichkeit auf Twitch einen User zu einer Spende oder einer Subscribtion zu motivieren. Dies sind kurze Animationen die zusammen mit Hintergrundsound sehr present im Stream angezeigt werden. Zusätzlich wird wie in \autoref{img:alert} zu sehen in der Regel auch der Name des Spenders oder oder in diesem Fall des Subscribers angezeigt.

\begin{figure}[ht]
\caption{Alert Overlay}
\label{img:alert}
\centering
\includegraphics[width=\textwidth]{twitch_alert}
\end{figure}

Der Kreativität sind bei den Overlays keine Grenzen gesetzt. Alles was mit HTML, CSS und JavaScript implementiert werden kann, kann als Overlay dienen. Hier ist darauf zu achten das der Hintergrund der Webseite als transparent definiert ist. Damit sich nicht jeder Streamer selbst seine Overlays programmieren muss ist hierfür ein großer Markt an Dienstleistern entstanden. Die zwei bekanntesten sind StreamLabs und Streamelements.

Auch Anbieter von Software die mit Video Spiel Streaming eng verknüpft ist bieten Overlay Dienste an. So hat Discord, der Anbieter einer Instant Messaging und Voip Plattform, das sogenannte Discord StreamKit. Damit können dynamische Elemente in den Stream eingebunden werden. Sie zeigen zB an, wie viele User gerade auf dem Server angemeldet sind oder wer gerade in einem definierten Voip Channel spricht.


\section{Chat Bots}
Neben den Overlays sind auch Chat Bots eine beliebte Möglichkeit einen Stream interessant zu gestalten. Diese sind kleine Programme die dauerhaft mit dem Twicht Chat eines Channels verbunden sind. Sie können Nachrichten lesen und schreiben. Vorher definierte Schlüsselwörter mit einem Ausrufezeichen davor fungieren als Kommandos. Wird ein solches Kommando von einem User in den Chat geschrieben so führt der Bot eine vorher definierte Aktion durch.

Diese Aktionen können zum Beispiel das Abspielen verschiedener Sounds oder das Einblenden kurzer Gifs  im Stream sein. Aber auch komplexere Dinge sind möglich. So können die bereits in \autoref{chap:tw_gami_punkte} erwähnten Kanalpunkte auch mit Bots umgesetzt werden. Der Bot erkennt alle Spieler die aktuell mit dem Chat verbunden sind und gibt diesen in definierten Abständen Punkte. Den aktuellen Punktestand kann ein User sich zum Beispiel mit dem Kommando !points ausgeben lassen. Punkte einlösen ist ebenfalls mit einem anderen Kommando möglich.

% TODO Paper über Bots Gamification 

Im Gegensatz zu den nativen Kanalpunkten können Bots die Punkte für verschiedene Aktionen an User gutgeschrieben werden. So gibt es auch einige Minispiele wie zum Beispiel Roulette. Hier kann ein User Punkte setzten um diese zu vermehren oder alle zu verlieren. Ein weiteres Beispielt sind Affiliate Links. Hierbei kann durch ein Kommando ein Link generiert werden der bei Klicken durch Dritte dem Linkersteller Punkte gutschreibt. Die Dritten werden über den Link zum Stream geleitet. \parencite{Browne2018}

\section{Twitch API}
Damit die Gamification Element auch in einer Kampagne mit definierten Zielen verwendet werden können, muss es eine Möglichkeit geben diese Elemente zu beeinflussen. 

Dies 

%If the table is too wide, replace \begin{table}[!htp]...\end{table} with
%\begin{adjustwidth}{-2.5 cm}{-2.5 cm}\centering\begin{threeparttable}[!htb]...\end{threeparttable}\end{adjustwidth}
\begin{table}[!htp]\centering
\caption{Gamification auf Twitch}\label{tab: }
\scriptsize
\begin{tabular}{lrrr}\toprule
\textbf{Achievements} &Triggered by &external triggers \\
Alerts im Stream &free &x \\
Mention vom Bot &free &x \\
Mention vom Streamer &free &x \\
\textbf{} & & \\
\textbf{Badges} & & \\
Staff &role staff & \\
Admin &role admin & \\
Broadcaster &role broadcaster & \\
Mod &role mod & \\
Verified &role verified & \\
VIP &role vip &x \\
Twitch Turbo User &role turbo user & \\
Prime User &role prime user & \\
1st Sub &sub & \\
Subgifter (1-1K) &sub gift & \\
Top Gifter (1-3) &sub gift & \\
Bits (1-1M) &bits & \\
Top Bits (1-3) &bits & \\
Special Events (TwitchCon) &special event & \\
& & \\
\textbf{Progressbars} & & \\
Hypetrain &sub, bits & \\
Goals im Stream &sub, bits, free &x \\
\textbf{} & & \\
\textbf{Leaderboards} & & \\
Top 3 Bits & Subgifter über Chat &bits / subs & \\
Top 1 Bitter / Subber im Stream & & \\
Last Bitter / Subber im Stream & & \\
\textbf{} & & \\
\textbf{Ingame Währung} & & \\
Bits &purchase & \\
\textbf{} & & \\
\textbf{Points} & & \\
Channelpoints &watchtime & \\
& & \\
\textbf{Unlockables} & & \\
Emotes &subs & \\
Drops &watchtime &x nicht API \\
& & \\
\textbf{Gifting / Sharing} & & \\
Gifted Subs &subs & \\
\bottomrule
\end{tabular}
\end{table}


%
%
%
%
%
%
%
%
%
%
% 
\chapter{Marketing auf Twitch}
Online Marketing ist ein sehr großes Feld und umfasst viele verschiedene Strategien. Diese können einzeln angewandt oder miteinander kombiniert werden. Eine beispielhafte Übersicht über die aktuelle Landschaft der Marketing Strategien kann \autoref{img:mark_sek} entnommen werden.

\begin{figure}[ht]
\caption{Online Marketing Kompakt im Sektorenmodell \parencite{Volkl-wolf2020}}
\label{img:mark_sek}
\centering
\includegraphics[width=0.8\textwidth]{mark_sek_mod}
\end{figure}

Aufgrund des Aufbaus und der Art und Weise wie Twitch funktioniert werden nur bestimmte Formen des Online-Marketings dort angewandt. Diese sind Display-Marketing, Video-Marketing, Branded-Content-Marketing, Affiliate-Marketing und Influencer-Marketing. Im folgenden werden die verschiedenen Formen anhand von Beispielen erklärt. Danach wird bewertet und überprüft, wie diese mit Gamification kombiniert werden können.

\section{Display-Marketing}
Eine der ältesten Marketing Methoden im digitalen Bereich ist das Display-Marketing. Sie wird alternativ auch Banner-Werbung genannt. Hierbei werden auf einer Webseite verschiedene Banner eingeblendet, die über Produkte oder Firmen informieren. Diese Banner können in verschiedenen Formen und Größen auf einer Webseite eingebunden werden. Beispielsweise gibt es sogenannte Skyscraper, welche am rechten bzw. linken Bildschirmrand platziert sind. Oft werden Banner auch zwischen normalen Content einer Webseite angezeigt, um die Aufmerksamkeit eines Betrachters in einem konzentrierten Moment auf ein Produkt zu lenken. Banner können auch unerwartet bildschirmfüllend erscheinen, in manchen Fällen sogar als eigenes Fenster oder Tab. Diese Art der Banner wird Pop-Ups genannt.\parencite{Xovi2019}

Firmen können auf Twitch Banner-Werbung schalten, um Produkte zu bewerben oder ihren Livestream hervorzuheben. Diese Banner werden in vier verschiedenen Bereichen der Plattform angezeigt. Beginnend von links nach rechts in \autoref{img:banner_ads} ist die erste Variante, die als Medium Rectangle bezeichnet wird, ein mittelgroßer Banner für Produkte, welcher in der Explore Seite von Twitch zwischen den Kategorie-Kacheln angezeigt wird. Das Super Leaderboard wird ebenfalls für Produkte auf der Explore Seite angezeigt, jedoch am oberen Browser-Rand als langer horizontaler Balken. Auf der Startseite von Twitch wird als erstes Element das sogenannte Homepage Carousel angezeigt. In diesem werden Livestreams hervorgehoben. Die Kanäle die dort angezeigt werden, haben sich für die dritte Variante der Banner-Werbung entschieden. Die letzte Variante ist ein großer horizontaler Banner, welcher hinter dem Carousel angezeigt wird, um Produkte zu bewerben. \parencite{Twitch2020}

\begin{figure}[ht]
\caption{Banner platzierung auf Twitch \parencite{Twitcha}}
\label{img:banner_ads}
\centering
\includegraphics[width=\textwidth]{twitch_ads_banner}
\end{figure}

\section{Video-Marketing}
Video-Marketing ist der Einsatz von audio-visuellen Inhalten in einer Marketing Kampagne. Einfach betrachtet ist Video-Marketing das Erstellen und Platzieren eines Videos. Die Ziele dieser Videos können vielfältig sein und reichen vom bewerben der Firma, anregen von Verkäufen, erhöhen der Bekanntheit einer Marke oder Dienstleistung bis hin zum Stärken der Kundenbindung.\parencite{Stringfellow2017}

Twitch als reine Video-Plattform hat Video-Marketing  als einen Kernbestandteil. Der Plattform wurde im Jahr 2019 der \glqq Best Video Marketing and Advertising Platform\grqq{} Award verliehen\parencite{DigidayAwards2019}. Hierbei wird klar ersichtlich wie viel Potential in Video Marketing auf Twitch steckt.

Auf Twitch wird Video-Marketing zum einen in Form von kurzen Werbeclips im Videofeed des Streams eingesetzt. Hier werden diese zu verschiedenen Zeiten des Streams eingeblendet. Es wird zwischen pre-roll, mid-roll und post-roll Ads unterschieden. Hierbei handelt es sich um klassische Werbevideos wie sie auch im Fernsehen zu sehen sind. Diese werde vor, während oder nach einem Livestream angezeigt. Klickt ein User beispielsweise auf einen Live-Channel um dessen Stream beizutreten, wird automatisch ein Werbeclip gestartet. Wie in \autoref{img:preroll} zu sehen wird oben rechts in der Ecke die noch verbleibende Dauer angezeigt und der Clip kann nicht übersprungen werden. 

\begin{figure}[ht]
\caption{Twitch Pre-Roll Ad; Screenshot von \parencite{LaraLoft2020}}
\label{img:preroll}
\centering
\includegraphics[width=\textwidth]{twitch_video_preroll_edit}
\end{figure}

Bei den Mid-Roll Ads haben die Streamer die Möglichkeit den Zeitpunkt und die Dauer der Einblendung selbst zu bestimmen. Hierbei können sie zwischen 30, 60, 90, 120, 150 und 180 Sekunden wählen. Diese nutzen sie beispielsweise, um in möglichst spannenden Szenen des Streams die Aufmerksamkeit auf die Werbung zu lenken oder eine kurze Pause einzulegen. Diese Werbeclips werden im sogenannten Picture-By-Picture Modus angezeigt. Hierbei wird bei Aktivieren des Clips der Livestream verkleinert und im rechten Bildrand angezeigt, während die Werbung den ursprünglichen Platz des Streams einnimmt. \parencite{Twitch}

\begin{figure}[ht]
\caption{Twitch Video Marketing Varianten \parencite{Twitcha}}
\label{img:twitch_vid_ads}
\centering
\includegraphics[width=\textwidth]{twitch_ads_video}
\end{figure}

Twitch bietet vier verschiedene Varianten an, mit denen Werbevideos an den Konsumenten ausgeliefert werden. Beginnend von links nach rechts in \autoref{img:twitch_vid_ads} ist die erste Variante die Ausspielung in der Smartphone-App von Twitch sowie in der Browser Version. Variante zwei und drei betrachtet die Ausspielung auf Mobile und in der App als separate Kanäle. So kann spezifischer auf eine Zielgruppe eingegangen werden. \parencite{Twitcha}

Jedoch werden die Ads nicht angezeigt falls man einen AdBlocker (dt. Werbeblocker) im Browser benutzt. Das ist eine Browser Erweiterung welche das Darstellen von Werbung verhindert. Ist man Subscriber eines Channels oder hat für Twitch Turbo bezahlt, werden die Werbeclips auch übersprungen. Variante vier kann das Problem mit dem AdBlocker allerdings umgehen. Es handelt sich um eine neue Technologie mit dem Namen SureStream von Twitch, welche es der Plattform ermöglicht die Werbeclips direkt in den Video-Feed des Livestreams zu integrieren. Dadurch hat ein AdBlocker nicht mehr die Möglichkeit das Abspielen einer Werbung zu unterbinden. \parencite{Twitch2016}

Zudem wird Video-Marketing auch ganz allgemein auf Twitch eingesetzt. Wie bereits eingangs beschrieben umfasst diese Marketing Strategie jegliche Art von audio-visuellem Content. Somit können auch die im folgenden erklärten Marketing Strategien in der Art und Weise wie sie auf Twitch Anwendung finden, unter dem Video Marketing angesiedelt werden.

\section{Branded Content-Marketing}
Bei Branded Content-Marketing geht es darum relevante Inhalte zu produzieren und diese direkt mit einer Marke zu verknüpfen. Die Inhalte sollen ein positives Gefühl mit der Marke assoziieren und die Gefühle der Zuschauer ansprechen. Sie haben oft einen unterhaltsamen Charakter und es wird in der Regel Storytelling angewandt.  Die Inhalte werden oft auch in Kooperation mehrerer Firmen oder Influencern produziert. Firmen, welche bekannt dafür sind diese Strategien besonders gut einsetzen, sind Red Bull und Coca Cola. \parencite{Cardona2020}

Auf Twitch startete beispielsweise die Firma Old Spice, ein Hersteller von Pflegeprodukten für Herren, im Jahr 2015 eine interaktive Kampagne auf Twitch. Unter dem Titel \glqq Old Spice Nature Adventure\grqq{} wurde ein Live Stream erstellt, in welchem ein Mann, mit einer Kamera auf dem Rücken gespannt, durch einen Wald läuft. Das Videosignal dieser Kamera wurde im Stream übertragen. Jeder Zuschauer konnte Vorschläge in den Twitch Chat schreiben, welche Aktion der Mann als nächstes durchführen soll. Das Produktionsteam suchte aus diesen Vorschlägen etwa alle 10 - 30 Sekunden den Besten aus und der Mann führte diesen durch. Der nächste durchzuführende bzw. aktuelle Vorschlag wurde unten links im Bild (vgl. \autoref{img:oldspice}) zusammen mit dem Namen des Vorschlaggebers eingeblendet.

\begin{figure}[ht]
\caption{Old Spice Nature Adventure; Screenshot von \parencite{OldSpice2015}}
\label{img:oldspice}
\centering
\includegraphics[width=\textwidth]{twitch_marketing_old_spice}
\end{figure}

Die Kampagne war wie ein virtuelles Rollenspiel aufgebaut. Der Mann war der Spieler-Charakter, welcher durch die Zuschauer gesteuert wurde. Im Wald, in dem sich der Mann bewegte, waren verschiedene Szenen aufgebaut. Zudem liefen auch Nicht-Spieler-Charaktere (NPC) umher, mit denen der Mann interagieren konnte. Einige dieser NPC waren als Tiere verkleidet, wodurch einige Szenen entstanden sind, in denen der Mann gegen einen Bären kämpfen musste (vgl. \autoref{img:oldspice}).

Insgesamt wurde die Werbekampagne von Old Spice als sehr erfolgreich eingestuft. Diese erreichte 2,56 Millionen Aufrufe und 29 Jahre an insgesamt angesehenem Video Material. Die exklusiv für diesen Stream freigeschaltenen Chat Emotes wurden über 105.000 Mal verwendet.  \parencite{Workman2015}

\section{Affiliate-Marketing}
Affiliate-Marketing ist ein provisionsbasierter Ansatz, um ein Produkt direkt an die Konsumenten zu bringen. Hierbei können sich Content Creator sogenannte \glqq Affiliate-Links\grqq{} generieren, welche direkt auf die Seite eines Produktes verweisen. In diesem Link ist ein Parameter gesetzt, welcher den Linkersteller eindeutig als solchen identifiziert, sozusagen eine Referenz zu diesem hergestellt wird. Die Links werden deshalb auch \glqq Ref\grqq{}-Links bezeichnet. Sie werden in der Regel bei Produktreviews, Inventarlisten oder in Artikeln über die bestimmte Produktkategorie als Empfehlung eingesetzt. Klickt ein Kunde nun auf einen solchen Link gelangt er direkt auf die Seite des Produktes. Kauft der Kunde sich dieses Produkt zahlt der Verkäufer eine Provision an den Linkersteller aus, für den Käufer entstehen jedoch keine Mehrkosten. \parencite{Gallaugher2001}

\begin{figure}[ht]
\caption{Ref Links in der \glqq About\grqq{} -Sektion auf Twitch \parencite{LeosMind}}
\label{img:aff_about}
\centering
\includegraphics[width=\textwidth]{twitch_affiliate_reflinks}
\end{figure}

Twitch bietet die Möglichkeit in der \glqq About\grqq{} -Sektion eines Profils verschiedene Banner und Texte einzublenden. Ebenso können auch Links eingebunden werden. Dieser Bereich wird von vielen Streamern dazu genutzt Affiliate-Links zu integrieren. Die \glqq About\grqq{} -Sektion wird auch während eines Streams unterhalb des Video-Feeds angezeigt. Wie in \autoref{img:aff_about} zu sehen, hat der Streamer (1) eine Liste mit den dazugehörigen Links seiner verwendeten Hardware in seiner \glqq About\grqq{} -Sektion integriert. Jeder dieser Links ist ein \glqq Ref\grqq{}-Link zu Amazon. Auch ein (2) Banner, welcher zur Seite eines Drittanbieters führt, wurde mit einem \glqq Ref\grqq{}-Link versehen. Zudem kann der Streamer über einen Bot auch regelmäßig \glqq Ref\grqq{}-Links in den Chat schicken lassen. Bots sind kleine Programme welche sich mit dem Twitch Chat verbinden um Nachrichten zu lesen und zu schreiben.

\section{Influencer-Marketing}

Um zu verstehen was Influencer-Marketing bedeutet muss zunächst der Begriff Influencer erklärt werden. Ein Influencer ist eine Person welche aufgrund ihres Status und ihrer Reichweite in einer Nische einen signifikanten Einfluss auf die Kaufentscheidungen anderer Personen hat \parencite{Brown2008, InfluencerMarketingHub2017}. Solch ein Influencer wird von Firmen bezahlt um deren Produkte zu bewerben.

Die Bewerbung der Produkte kann er auf verschiedene Arten machen. Dies sind gesponserte Inhalte, Empfehlungen oder Produktplatzierungen. So können auf YouTube beispielsweise Firmen einem Influencer verschiedene Produkte kostenlos zur Verfügung stellen der diese in seinen Videos einsetzt, präsentiert oder als besonders gut darstellen. \parencite{Geyser2016}

\begin{figure}[ht]
\caption{Alien Promotion von Lara Loft}
\label{img:laraloftalien}
\centering
\includegraphics[width=\textwidth]{twitch_laraloft_alien_promo_edit}
\end{figure}

Zum Start von \glqq Alien: Convenant\grqq{} im Jahr 2017 führte 20th Century Fox eine Kampagne auf Twitch durch. Diese bestand jedoch nicht aus dem Besprechen des Films oder dem gemeinsamen Schauen des Trailers im Stream, sondern aus der Inszinierung einer Alien-Invasion kurz vor Ende des Streams. Es wurde mit flackernden Lichtern, Bildstörungen oder auch wie in \autoref{img:laraloftalien} zu sehen mit CGI-Effekten wie (1) deutlichen Kratzern an der Wand und (2) einem Alienschwanz, der am Rande des Bildes vorbeistreift zuerst eine gruselige Atmosphäre geschaffen, um Spannung aufzubauen. Nachdem das Licht komplett ausfiel und die Streamerin aus der Kamera trat, um dieses wieder einzuschalten, sprang ein Alien direkt in die Kamera um die Zuschauer zu erschrecken. Daraufhin startet der Trailer zum neuen Film. Dieser erreichte somit mehr als eine halbe Millionen Aufrufe und die Kampagne war ein Erfolg. \parencite{T3N2018}

\section{Bewertung der einzelnen Marketingmethoden}

Für die Bewertung der einzelnen Marketing Methoden auf Twitch wurden diese im Detail über mehrere Monate beobachtet. 

Hierbei wurde festgestellt, dass die Banner des Display Marketings offensichtlich kaum noch eine Rolle spielen. Bei der Aktivierung eines AdBlockers werden diese zuverlässig ausgeblendet. War der AdBlocker jedoch deaktiviert, so war die Wahrscheinlichkeit, dass ein Banner angezeigt wurde, sehr gering. Während des Schreibens dieser Arbeit wurde gezielt auf die Benutzung eines AdBlockers verzichtet. Jedoch wurde im gesamten Zeitraum von drei Monaten nur einmal ein Banner im Header der Startseite angezeigt.  Deshalb werden Banner für diese Arbeit als nicht relevant definiert.

Da bei Affiliate Marketing durch einen Link auf die Seite eines Drittanbieters weitergeleitet wird, können die dortigen technischen Freiheiten genutzt werden. So könnte beim Kauf eines Produktes ein Event ausgelöst werden, welches die Overlays in einem Twitch Video-Feed beeinflusst. Somit bietet diese Marketing Strategie eine Möglichkeit in dieser Arbeit. Hierzu mehr in \autoref{kap:gami_konzept}.

Video-Marketing ist für Unternehmen eine einfache und lukrative Methode. Die neue Technologie der SureStreams kann, wie beworben, auch von einem Ad Blocker nicht umgangen werden. Die Gamification Elemente der Plattform finden jedoch erst ihren Einsatz im Video-Feed des Streamers. Somit können die Video Ads auch als nicht relevant definiert werden. Allerdings ist Video Marketing auf Twitch wie bereits beschrieben auch in Branded Content Marketing und Influencer Marketing implementiert. 

Als Resultat können als Marketing Strategie dieser Arbeit drei Möglichkeiten definiert werden. Erstens der Live Stream einer Firma die eigenen Content über ihren eigenen Channel produziert. Zweitens wenn Firmen einen Influencer bezahlen um Content zu prodzieren welcher auf dem Channel des Influencers ausgestrahlt wird. Drittens das Affiliate Marketing mit den entsprechenden Links. 
%
%
%
%
%
%
%
%
%
%
% 
\chapter{Konzeption und Implementierung}

\section{Anforderungsanalyse}
Für die Anforderungsanalyse wird ein Ansatz mit einer qualitativen Datenerhebung gefahren. In diesem werden semistrukturierte Interviews durchgeführt. Diese Art der Interviews wird auch Leitfadeninterviews genannt. In den Interviews werden Fragen gestellt um dem Interview eine thematische Richtung zu geben, es aber nicht zu sehr einzuschränken. \parencite{Wessel2010} Für dieses Interview wurde ein Leitfaden erstellt. In diesem wird der Ablauf genau definiert und dient während des Gespräches als Orientierung.

Begonnen wird der Leitfaden mit dem Start des Interviews. Der Interviewee wird begrüßt und die Rahmenbedingungen für das Interview werden geklärt.

\begin{itemize}
\item Begrüßung des Interviewees und Bedanken für die Teilnahme
\item Kurze Einführung in das Thema
\item Erklären des Interview Leitfadens
\item Datenschutzvereinbarung
\end{itemize}

Daraufhin wird mit einleitenden Fragen zum Thema hingeführt

\begin{itemize}
\item Wissen Sie was Gamification ist?
\item Was genau ist ihre Beziehung zur Musikindustrie und Twitch? Wie lange besteht diese Beziehung schon?
\item Wie und wo sind ihre täglichen Kontaktpunkte und Tätigkeiten im Bezug auf die Musikindustrie?
\item Wie viele Zuschauer haben Sie im Durchschnitt auf Twitch?
\end{itemize}

Im Hauptteil des Interviews werden dann die Schlüsselfragen gestellt. Hierfür müssen diese Interviewfragen systematisch aus der Forschungsfrage der Arbeit abgeleitet werden. Dabei wurde sich am Prozess von \cite{Kaiser2014} orientiert. In diesem werden zunächst aus der Forschungsfrage die Analysedimensionen definiert. Hieraus im Anschluss Fragenkomplexe, aus welchen dann wiederum konkrete Fragen für die Interviews entstehen. 

Die Forschungsfrage in dieser Arbeit lautet \textbf{"Wie können die Gamification Elemente der Plattform Twitch von Firmen in der Musikbranche für Marketingzwecke eingesetzt werden?"} Bei der Betrachtung der Forschungsfrage fallen direkt drei Schlagworte auf:

\begin{itemize}
\item Musikbranche
\item Marketingzwecke
\item Gamification
\end{itemize}

Diese sind die drei Kernbestandteile der Frage, Ziele dieser Forschung und können somit auch als Analysedimensionen verwendet werden.

In der ersten Dimension "Musikbranche" ist herauszufinden wie die Industrie funktioniert bzw. womit Firmen Geld verdienen. Daraus leitet sich direkt die Interviewfrage 1 ab.

Die zweite Dimension "Marketingzwecke" dreht sich um die Frage was genau die zu implementierende Kampagne erreichen soll. Für eine erfolgreiche Kampagne muss man mehrere Faktoren berücksicktigen. Auf der einen Seite muss man sich seiner Zielgruppe im klaren sein. \parencite{Knauer2010} Daher sollte in den Interviews im Fragekomplex "Zielgruppe" auch herausgefunden werden wie genau die Zielgruppe aussieht. Also konkrete Interviewfrage leitet sich hier die Frage nach der wichtigsten Zielgruppe ab. Zudem ist auch in Social Media Marketing die Tonalit bzw. der Umgangston mit dem Kunden sehr wichtig. \parencite{Ballouli2016} Hierraus leitet sich die andere Frage in diesem Fragekomplex ab.
- Kommunikationskanäle


Als weiter Fragenkomplex unter den Marketingzwecken ist die Kampagnenziele angesiedelt, da ohne Ziele eine Kampagne keinen Sinn macht.
- Allgemeine Ziele
- Neue Ziele


Die letzte Dimension ist "relevante Gamification Elemente". Diese benötigen wir um herauszufinden welche Elemente wir von der Plattform Twitch einsetzen können. In einem vorherigen Kapitel wurden bereits die auf Twitch vorhanden Elemente gesucht und aufgelistet. In einem zweiten Schritt wurde analysiert welche Elemente beeinflusst werden können. Somit existiert bereits eine Liste von Elementen welche für eine Kampagne eingesetzt werden. Wie bereits in \autoref{chap:gamif} angesprochen, leiten sich die zu verwendenden Elemente aus den verfolgten Zielen ab. Somit können später die Ergebnisse von Frage 5 verwendet werden.

\textbf{Schlüsselfragen}

\begin{enumerate}
\item Frage: Wie funktioniert die Musikindistrie bzw. wie verdienen Unternehmens in der Musik Branche ihr Geld? Wie sind diese mit Twitch verknüpft?
\item Frage: Welche Zielgruppe ist besonders wichtig? Wie ist die demografische Verteilung? Was ist wichtige, Quantität oder Qualität?
\item Frage: Was sind die aktuellen Wege einen Kunden / Fan zu erreichen? Über welche Kanäle kommunizieren Sie bereits?
\item Frage: Was ist wichtig im Umgang mit einem Kunden / Fan in ihrer Branche? Welcher Umgangston ist zu verwenden, eher professionell distanziert oder nah relateable?
\item Frage: Aktuelle Ziele von Marketing Kampagnen? Mögliche Ziele einer Marketing Kampagne auf Twitch?
\item Frage: Wie könnte man für eine Kampagne auf Twitch auch Gamification Elemente einsetzen?
\end{enumerate}

Zum Abrunden des Gespräches
\begin{itemize}
\item Zusammenfassung der Mitschriften
\item Information über Verwendung der Informationen
\item Danke für die Teilnahme
\item Verabschiedung
\end{itemize}

\subsection{Erhebung durch Experteninterviews}
Für die Erhebung der Daten wurden mehrere Experteninterviews durchgeführt. Diese wurden digital in der online Meeting Software Zoom abgehalten. Durch die integrierte recording Funktion wurden die Interviews aufgezeichnet.

Es wurden insgesamt drei verschiedene Interviews abgehalten. 

\begin{itemize}
\item (Newcomer) DJ und aktiver Twitch Nutzer mit weniger als 100 Followern auf Twitch (Affiliate)
\item (Fortgeschritten) Musiker \& Produzent mit 5000 Tausend Followern auf Twitch (Affiliate)
\item (Profi) DJ \& Produzent mit 6000 Followern auf Twitch (Partner)
\end{itemize}

\subsection{Zusammenfassen der Ergebnisse}

Um die Ergebnisse der drei Interviews analysieren zu können müssen diese zunächste in eine einheitliche Form gebracht werden. Hierzu wird die Inhaltsanalyse nach \cite{Kuckartz2018} angewendet. Diese ist ein mehrstufiger Prozess in welchem die transkribierten Fließtexte der Interviews in verschiedene Kategorien zusammengefasst werden. Da in dieser Arbeit  die Antworten der Experten in stichpunktartiger Form erhoben wurden, wurde auf den ersten Schritt initiierende Textarbeit verzichtet. Der zweite Schritt ist das Entwickeln von Hauptkategorien. Diese orientieren sich in diesem Fall an den Interviewfragen.

Auf eine Transkription des Interviews wurde in dieser Arbeit verzichtet und die Informationen wurden direkt in tabellarischer Form und stichpunktartig zu den entstandenen Kategorien zugeordnet.

Die Interviewees konnten zunächst mit dem Begriff Gamification wenig anfangen bzw. hatten eine falsche Vorstellung was Gamification ist. Jedoch war ihnen die Mechaniken dahinter sehr wohl bewusst. So ist allen bereits aufgefallen dass Twitch gezielt Elemente verwendet um User zu motivieren und an die Platform zu binden.


\subsection{Definition der Anforderungen}

Als Marketingziele wird definiert:
- Merschandise Käufe erhöhen
- Anzahl der Discord Nutzer erhöhen

\subsection{Auswahl der passenden Gamification Elemente}

Die bereits im Kapitel

\subsection{Gamification Konzept}
\label{kap:gami_konzept}

Progressbar Merch Soll Anzahl -> Merchandise Kauf -> Alert im Stream
Progressbar Discord User -> Discord Join -> Alert im Stream

Unter allen Discord Usern und Merch Käufern wird ein 1 Monat VIP Badge verlost

Da die Gamification Elemente auf Twitch bereits seit Jahren erfolgreich einsetzt kann davon ausgegangen werden dass diese auf die Zuschauerschaft abgestimmt sind. Da wir in dieser Arbeit auf die gleichen Elemente zurückgreifen ist eine separate Betrachtung der Spielertypen nicht notwendig.

\section{Architekturkonzept}
Im Architekturkonzept gibt es einen zentralen Server welcher das Herzstück der Anwendung bildet. Auf diesem laufen sämtliche Informationen zusammen und werden gebündelt über die Overlays ausgegeben. 

Der Server hat ein Frontend welches die Grundlage der Overlays bildet. Dieses kann über den Browser abgerufen werden. Es wird mit HTML, CSS und JavaScript entwickelt. Zudem stellt dieses Overlay eine Web Socket Verbindung zum Backend des Servers her. 



Das erste Ziel der Kampagne ist das erhöhen der Nutzerzahlen auf dem Discord Server. Da Discord eine Chat Plattform ist bietet es genauso wie Twitch die Möglichkeit Chat Bots zu integrieren. Diese Bots können zum einen Unterhaltungen mitlesen, auf Bestimmte events wie das beitreten eines Users reagieren und in den chat schreiben. 

Für das Lesen der Nutzerzahlen und das Beitreten eines Users wird somit ein Bot entwickelt der auf diese Events reagieren kann. Sobald ein User dem Discord Server beitritt bekommt der Bot dieses Event mitgeteilt. Der Bot benachrichtigt daraufhin den Server.



Twitch API
Twitch IRC Chat Interface

Discord Bot API -> Join Alert

Stream overlay WebSocket

Overlay Server Controller

\begin{figure}[ht]
\caption{Architekturkonzept}
\centering
\includegraphics[width=\textwidth]{Architekturkonzept}
\end{figure}

%
%
%
%
%
%
%
%
%
%
% 
\chapter{Diskussion}

\section{Kritische Betrachtung}

Da in dieser Arbeit zur Evalation des Ergebnisses nur Experteninterviews durchgeführt wurden ist dieses Ergebnis eher kritisch zu betrachten. Dieser eher subjektiven Ergebnisse kann nicht zu hundert prozent vertraut werden. Hierfür wäre eine Feldtest Studie empehlenswert um Statistisch beweisbare Ergebnisse zu erzeugen.

In den Experteninterviews wurden lediglich Personen befragt die bereits auf Twitch aktiv sind mit einer Zuschauerschaft von 10 - 100 durchschnittlichen Zuschauern.

Twitch sollte Kanalpunkte dynamischer gestalten damit diese mehr Möglichkeiten Bieten.
- punkte vergeben können
- definieren wie viel ein User für die aktionen bekommt

\section{Empfehlung für zukünftige Forschungen}
“Neid Faktor” der Achievements im Stream
Spielertypen auf Twitch
%
%
%
%
%
%
%
%
%
%
% ###################################################################### Ausblick

\chapter{Fazit und Ausblick}

Ausblick

%
%
%
%
%
%
%
%
%
%
% ###################################################################### ENDE
\backmatter

\listoffigures
\addcontentsline{toc}{chapter}{Verzeichnisse}

\listoftables

%% create listings list
%  \lstlistoflistings
%  \addcontentsline{toc}{chapter}{Listings}

\cleardoublepage
\phantomsection
\addcontentsline{toc}{chapter}{Literatur}
\printbibliography

\addchap{Eidesstattliche Erklärung}

Hiermit versichere ich, dass ich die vorgelegte Bachelorarbeit selbstständig verfasst und noch nicht anderweitig zu Prüfungszwecken vorgelegt habe. Alle benutzten Quellen und Hilfsmittel sind angegeben, wörtliche und sinngemäße Zitate wurden als solche gekennzeichnet.

\vspace{20pt}
\begin{flushright}
$\overline{~~~~~~~~~~~~~~~~~\mbox{\ShowBaAuthor, am \today}~~~~~~~~~~~~~~~~~}$
\end{flushright}

\addchap{Zustimmung zur Plagiatsüberprüfung}

Hiermit willige ich ein, dass zum Zwecke der Überprüfung auf Plagiate meine vorgelegte Arbeit in digitaler Form an PlagScan (www.plagscan.com) übermittelt und diese vorübergehend (max. 5~Jahre) in der von PlagScan geführten Datenbank gespeichert wird sowie persönliche Daten, die Teil dieser Arbeit sind, dort hinterlegt werden.

\begin{small}
Die Einwilligung ist freiwillig. Ohne diese Einwilligung kann unter Entfernung aller persönlichen Angaben und Wahrung der urheberrechtlichen Vorgaben die Plagiatsüberprüfung nicht verhindert werden. Die Einwilligung zur Speicherung und Verwendung der persönlichen Daten kann jederzeit durch Erklärung gegenüber der Fakultät widerrufen werden.
\end{small}

\vspace{20pt}
\begin{flushright}
$\overline{~~~~~~~~~~~~~~~~~\mbox{\ShowBaAuthor, am \today}~~~~~~~~~~~~~~~~~}$
\end{flushright}

\end{document}
