\chapter{Twitch}
\section{Übersicht}
Twitch ist eine WebVideo Platform welche von Amazon betrieben wird. Sie wird hauptsächlich zum Streamen von Videspielen eingesetzt. Sie wird täglich von X Zuschauern besucht welche Y Streamer zuschauen.

Sie war zum Start der Platform noch unter dem namen Justin.tv bekannt. Aufgrund der hohen Anfrage von Videospiel Content auf dieser Seite wurde eine zweite Seite dediziert für Videospiele mit dem heute bekannten Namen Twitch gestartet welche im Jahr 2011 ans Netz ging. Später wurde auch der Name der Firma in Twitch umbenannt. Im JAhr 2014 wurde Twitch von Amazon gekauft und ist seitdem der Betreiber.


Jeder Stream wird einer Kategorie zugeordnet. Die beliebtesten Kategorie auf Twitch sind
-
-
-

Neben Videosielen werde auch viele andere Kategorien gestreamt
Just Chatting
Kochen
Musik

Monetarisiert 
Subs
Bits
Direkte Spenden

\section{Marketing auf Twitch}

Auf Twitch werden verschiedene Formen des Online Marketings angewandt. Im folgenden werden die verschiedenen Formen anhand von Beispielen erklärt.

Dass Twitch eine lukrative Platform ist merkte auch Amazon worufhin diese Twitch kauften.

\subsection{Video Marketing}
Das Video Marketing sind die altbekannten Werbespots welche auch in älteren Medien wie dem Fernseher zum einsatz kommen.

Auf Twitch wird diese Form im Videofeed des Streams eingesetzt. Hier werden Werbeclips zu verschiedenen Zeiten des Streams eingeblendet. Klickt ein Besucher auf einen Live Kanale wird noch bevor der Video Feed des Streamers gestartet wird ein Werbeclip eingeblendet.

Auch währden des Streams können Werbeclips eingeblendet werden. Hierüber hat der Streamer auch die Kontrolle wann, wie viele und wie lange die Clips eingeblendet werden. Nachdem ein Clip gestartet wurde setzt ein Cooldown ein welcher einen weiteren Werbeclip verhindert.

pre-roll, mid-roll, and post-roll ads
Hierbei handelt es sich um klassische Werbe Videos wie sie auch im Fernsehen zu sehen sind. Diese werde vor, während oder nach einem Livestream angezeigt. Klickt man beispielsweise auf einen Live Channel um dessen Stream beizutreten wird automatisch ein Werbeclip gestartet. Streamer haben auch die möglichkeit während dem Stream einen Werbeclip einzuplenden. Hierbei können Sie zwischen 15 sekunden und 3 Minuten wählen. Diese nutzen Sie beispielsweise um in möglichst spannenden Szenen des Streams die Aufmerksamkeit auf die Werbung zu lenken oder eine Pause um zB auf die Toilette zu gehen mit Werbung zu füllen.

Jedoch werden die Ads nicht angezeigt falls man einen Ad Blocker im Browser benutzt. Ist man Subscriber einer Channels oder hat für Twitch Turbo bezahlt werden die Werbeclips auch übersprungen.

\subsection{Display Marketing}
Eine der ältesten Marketing Methoden im digitalen Bereich ist vermutlich das Display Marketing. Hierbei werden auf einer Webseite verschiedene Banner eingeblendet die über Produkte oder Firmen informieren. Hierzu zählen auch Popups.

Oft werden diese auch zwischen normalen Content einer Webseite angezeigt um in der großen Konzentration eines Betrachters dessen Aufmerksamkeit auf ein Produkt zu lenken.

Banner Werbung
Firmen können auf Twitch Banner Werbung schalten. Diese wird in verschiedenen Bereichen der Platform angezeigt. 

\subsection{Content Marketing}
Bei Content Marketing geht es darum relevanten Content zu produzieren welcher die Aufgabe hat den Kunden zu informieren und die Marke in einem besonders guten Licht dastehen zu lassen.

Branded Content (Original Content)
- "Old Spice" livestream in dem der Chat entscheiden konnte was der Protagonist tun soll
- Duracel

\subsection{Affiliate marketing}
Affiliate Marketing ist ein provisionsbasierter Anzatz um ein produkt direkt an den Mann zu bringen. Hierbei werden können sich Content ersteller sogenannte Affiliate Links generieren welche direkt auf die Kaufsseite eines Produktes verweisen. Diesem Link ist ein Flag mitgegeben welches auf den Linkersteller zurückführt. Der Link wird in der Regel bei Produktreviews, Inventarlisten oder in Artikeln über die bestimmte Produktkategorie als “Empfehlung” eingesetzt.

Klickt ein Kunde nun auf einen solchen Link gelangt er direkt auf die Kaufsseite des Produktes. Kauft der Kunde sich dieses Produkt so zahlt der Verkäuft eine Provision an den Linkersteller aus. 

Manche Plattformen wie Amazon vergüten auch die reine referenzierung auf ihre Platform. Hat ein Kunde auf einen Affiliate Link geklickt und befindet sich auf der Produktseite, kauft jedoch das verlinkte Produkt nicht, sondern ein anderes auf der Platform so bekommt der Verlinker dennoch eine Provieion für die Weiterleitung auf die Platform.

\subsection{Influencer marketing}
Product Placements

Zum Start von ­­­„Alien: Convenant“ im Jahr 2017 führte 20th Century Fox eine Kampagne auf Twitch durch. Diese sollte eine junge Zielgruppe für den neuen Teil der beliebten Filmreihe erreichen. Da der letzte Film der Reihe bereits im Jahr 1997

Twitch mit einer Kampagne, die eine junge Zielgruppe für das Prequel der Kultfilmreihe begeistern sollte. Die Filmreihe wurde 1997 mit „Alien – Die Wiedergeburt“­ abgeschlossen – zu einer Zeit also, zu der durchschnittliche Twitch-User noch zu jung waren, um zur Zielgruppe zu gehören. Um sie also als neues Publikum zu erschließen, musste das Horror­gefühl der frühen Alienfilme mit den Ansprüchen an dynamischen Live-Content verbunden werden. Vier bekannte Streamer aus Großbritannien, Deutschland, Frankreich und Russland konnten für eine Kooperation gewonnen werden. Allerdings bestand die Kampagne nicht daraus, dass die Streamer den neuen Film besprachen oder sich selbst beim Anschauen des Trailers filmten, wie man es von Plattformen wie Youtube kennt. Stattdessen fand in den regulären Live-Streams der ­Partner eine Alien-Invasion statt. Flackerndes Licht, Bildstörungen, lange Schleimfäden, die sich von der Decke ziehen. Traditionelle ­Horrorfilm-Techniken mit echten CGI-Alien-Aufnahmen aus dem Film kombiniert und die Zuschauer waren live dabei. Die Live-Alien-Attacke erzielte 38.888 Unique Views mit einer Gesamtzeit von 3.591 Stunden. Der platzierte Filmtrailer erreichte 549.645 Views und 35.255 Engagements.
