\chapter{Gamification}

\section{Grundlagen}

Gamification ist ein Begriff  der bereits seit einigen Jahren in vielen Bereichen anwendung findet. Hierbei handelt es sich um die verwendung von Elementen aus Spielen in nicht-spiel Umgebungen bzw. Kontexten. 
Die Hintergründe des ganzen sind dass Spiele eine Starke Motivation im Benutzer bzw Spieler erzeugen. Diese Motivation wird spezifischen Elementen aus Spielen zugeschrieben. Diese sind im Allgemeinen Punkte, Badges oder Leaderboards (PBLs)

Jedoch besteht auch die verwirrung dass angenommen wird dass das alleinige hinzufügen dieser Elemente bereits Gamification ist. Jedoch ist Gamification viel mehr. Es ist das gezielte ansprechen von Motivationsfaktoren in der Menschlichen Psyche

Pointification ist das pure “draufklatschen” von PBLs auf ein System, in der Hoffnung dass es den Benutzer motiviert.

Hierzu kann auf verschiedene Frameworks zurückgegriffen werden. Diese sind Beispielsweise das Gamifcation Model Canvas (GMC) welches vom Business Model Canvas abgeleitet wurde. 

Zum anderen aber auch das Octalysis Framework von YouKaiChou welches die Motivation des Menschen in 8 Teilbereiche unterteilt. Daher auch “Octal”-ysis. In diesem Framework haben wir das Octalysis Strategy Dashboads. Dieses wird dazu eingezetzt ganz spezifisch herauszufinden welche Gamification Elemente eingesetzt werden um vordefinierstes Verhalten auszulösen.

\section{Gamification auf Twitch}

Da Twtich eine Platform ist welche Ursprücnglich nur für das Live Streamen von Videospielen gabaut wurde ist es nicht verwunderlich dass diese sich auch die Kraft der Gamification zu nutze macht. Sowohl die Seite der Zuschauer als auch die Streamer Seite ist mit Elementen aus Spielen übersäht. 

\subsection{Streamer Seite}
Channel Analytics
Stream Summary
Achievements


\subsection{Zuschauer Seite}
\begin{markdown}


*Achievements*

* Alerts im Stream
* Popup im Chat & Mention vom Bot
* Mention vom Streamer

*Badges* 

* Sub Badges
    * 1st
    * Top Leaderboard 1-3
* Bits Badges
    * 1-1M
    * Top Leaderboard 1-3
* VIP
* Mod
* Emotes?!


*Progressbars*
Hypetrain
Goals im Stream
Leaderboards
Top 3 Bits & Subgifter über Chat
Top 1 im Stream
Last im Stream
Points
Channelpoints
Früher Chat Points über 3rd Party
Ingame Währung
Bits


Hier gibt es zum einen die Subscribtions welche aufgeteilt in drei "Tiers". Drei Stufen mit den Preisen 4.99, 9.99 und 24.99.

Subs verschenken
Leaderboards

Vorteile:
Werbefrei schauen
Community Gefühl!!!!!
emotes
Badge im Chat
recuring bekommen coolere badges
höhere tiers bekommen zusätzlichen flair

Channelpoints
* Vorgefertigte
  * Nachricht hervorheben
  * Unlock zufälliges Sub emote
  * Nachricht im Sub Only mode
  * Unlock spezifisches Sub emotes
  * ein emote bearbeiten

Channelpoints verdienen
* Watch for 5 minutes +10
* Claim special bonuses +50
* Participate in a Raid +250
* Follow this channel +300
* Monthly 1st Cheer +350
* Monthly 1st Gift a Sub +500
* Grow a watch streak up to +450
* Tier 1 sub 1.2x multiplier
* Tier 2 sub 1.4x multiplier
* Tier 3 sub 2x multiplier

Bits

HypeTrain
Der HypeTrain ist ein Event welches eintritt sobald eine Bits Spende in einer bestimmten Höhe oder eine bestimmte anzahl an Subscriptions

Drops

Mit steigender Beliebtheit der Platform steigt auch die Anzahl der täglichen Nutzer rapide an.


\end{markdown}

