\documentclass[12pt,twoside,a4paper,parskip]{scrbook}
\usepackage[utf8]{inputenc}
\usepackage{csquotes}
\usepackage[ngerman]{babel}
\usepackage{floatflt}
\usepackage{subfigure}
\usepackage[pdftex]{graphicx}
\usepackage[hidelinks]{hyperref}
\usepackage{color}
\usepackage{amssymb}
\usepackage{textcomp}
\usepackage{nicefrac}
\usepackage{scrhack}
\usepackage{pdfpages}
\usepackage{float}
\usepackage{pdflscape}
\usepackage{subfigure}
\usepackage{pdfpages}
\usepackage[verbose]{placeins}
\usepackage[markcase=ignoreuppercase,headsepline,plainfootsepline]{scrlayer-scrpage}
\usepackage{listings}
\usepackage{xcolor}
\usepackage{color}
\usepackage{caption}
\usepackage{subfigure}
\usepackage{epstopdf}
\usepackage{longtable}
\usepackage{setspace}
\usepackage{booktabs}
\usepackage[style=numeric,backend=biber]{biblatex}
\usepackage[hashEnumerators,smartEllipses]{markdown}
\usepackage{subfiles}
\bibliography{literatur}

 
%%%%%%%%%%%%%%%%%%%
%% definitions
%%%%%%%%%%%%%%%%%%%
\def\BaAuthor{Lukas Bauer}
\def\BaAuthorStudyProgram{Informatik} %% Wirtschaftsinformatik, E-Commerce, Informationssysteme
\def\BaType{Bachelorarbeit} %% Masterarbeit
\def\BaTitle{Gamification auf Twitch als Online-Marketinginstrument für Unternehmen in der Musikbranche}
\def\BaSupervisorOne{Prof.\ Dr.\ Isabel John}
\def\BaSupervisorTwo{Prof.\ Dr.\ Christina Völkl-Wolf}
\def\BaDeadline{\today}

\ifdefined\iswithfullname
  \def\ShowBaAuthor{\BaAuthor}
\else
  \def\ShowBaAuthor{N.~N.}
\fi

\hypersetup{
pdfauthor={\ShowBaAuthor},
pdftitle={\BaTitle},
pdfsubject={Subject},
pdfkeywords={Keywords}
}

%%%%%%%%%%%%%%%%%%%
%% configs to include
%%%%%%%%%%%%%%%%%%%
\colorlet{punct}{red!60!black}
\definecolor{background}{HTML}{EEEEEE}
\definecolor{delim}{RGB}{20,105,176}
\colorlet{numb}{magenta!60!black}

\definecolor{gray}{rgb}{0.4,0.4,0.4}
\definecolor{darkblue}{rgb}{0.0,0.0,0.6}
\definecolor{cyan}{rgb}{0.0,0.6,0.6}

\definecolor{pblue}{rgb}{0.13,0.13,1}
\definecolor{pgreen}{rgb}{0,0.5,0}
\definecolor{pred}{rgb}{0.9,0,0}
\definecolor{pgrey}{rgb}{0.46,0.45,0.48}

\lstset{
  basicstyle=\ttfamily,
  columns=fullflexible,
  showstringspaces=false,
  commentstyle=\color{gray}\upshape
  linewidth=\textwidth
}

\lstdefinelanguage{json}{
    basicstyle=\normalfont\ttfamily,
    numbers=left,
    numberstyle=\scriptsize,
    stepnumber=1,
    numbersep=8pt,
    showstringspaces=false,
    breaklines=true,
    backgroundcolor=\color{background},
    literate=
     *{0}{{{\color{numb}0}}}{1}
      {1}{{{\color{numb}1}}}{1}
      {2}{{{\color{numb}2}}}{1}
      {3}{{{\color{numb}3}}}{1}
      {4}{{{\color{numb}4}}}{1}
      {5}{{{\color{numb}5}}}{1}
      {6}{{{\color{numb}6}}}{1}
      {7}{{{\color{numb}7}}}{1}
      {8}{{{\color{numb}8}}}{1}
      {9}{{{\color{numb}9}}}{1}
      {:}{{{\color{punct}{:}}}}{1}
      {,}{{{\color{punct}{,}}}}{1}
      {\{}{{{\color{delim}{\{}}}}{1}
      {\}}{{{\color{delim}{\}}}}}{1}
      {[}{{{\color{delim}{[}}}}{1}
      {]}{{{\color{delim}{]}}}}{1},
}

\lstset{language=xml,
  morestring=[b]",
  morestring=[s]{>}{<},
  morecomment=[s]{<?}{?>},
  stringstyle=\color{black},
  numbers=left,
  numberstyle=\scriptsize,
  stepnumber=1,
  numbersep=8pt,
  identifierstyle=\color{darkblue},
  keywordstyle=\color{cyan},
  backgroundcolor=\color{background},
  morekeywords={xmlns,version,type}% list your attributes here
}

\lstset{language=Java,
  showspaces=false,
  showtabs=false,
  tabsize=4,
  breaklines=true,
  keepspaces=true,
  numbers=left,
  numberstyle=\scriptsize,
  stepnumber=1,
  numbersep=8pt,
  showstringspaces=false,
  breakatwhitespace=true,
  commentstyle=\color{pgreen},
  keywordstyle=\color{pblue},
  stringstyle=\color{pred},
  basicstyle=\ttfamily,
  backgroundcolor=\color{background},
%  moredelim=[il][\textcolor{pgrey}]{$$},
%  moredelim=[is][\textcolor{pgrey}]{\%\%}{\%\%}
}

\newcommand*{\forcetwosidetitle}[1][1]{%
 \begingroup
   \cleardoubleoddpage
   \KOMAoptions{titlepage=true}% useful e.g. for scrartcl
   \csname @twosidetrue\endcsname
   \maketitle[{#1}]
 \endgroup
}


\begin{document}


%%%%%%%%%%%%%%%%%%%
%% Titelseite
%%%%%%%%%%%%%%%%%%%


\frontmatter
\titlehead{%  {\centering Seitenkopf}
  {Hochschule für angewandte Wissenschaften Würzburg-Schweinfurt\\
   Fakultät Informatik und Wirtschaftsinformatik}}
\subject{\BaType}
\title{\BaTitle\\[15mm]}
\subtitle{\normalsize{vorgelegt an der Hochschule f\"{u}r angewandte Wissenschaften W\"{u}rzburg-Schweinfurt in der Fakult\"{a}t Informatik und Wirtschaftsinformatik zum Abschluss eines Studiums im Studiengang \BaAuthorStudyProgram}}
\author{\ShowBaAuthor}
\date{\normalsize{Eingereicht am: \BaDeadline}}
\publishers{
  \normalsize{Erstpr\"{u}fer: \BaSupervisorOne}\\
  \normalsize{Zweitpr\"{u}fer: \BaSupervisorTwo}\\
}
\lowertitleback{
\centering\includegraphics[width=4cm]{qrcode-thesis}

}
\forcetwosidetitle


%%%%%%%%%%%%%%%%%%%
%% abstract
%%%%%%%%%%%%%%%%%%%

\section*{Zusammenfassung}

TODO

\section*{Abstract}

TODO

\newpage
\chapter*{Danksagung}



%%%%%%%%%%%%%%%%%%%
%% Inhaltsverzeichnis
%%%%%%%%%%%%%%%%%%%
\tableofcontents



%%%%%%%%%%%%%%%%%%%
%% Main part of the thesis
%%%%%%%%%%%%%%%%%%%
\mainmatter
\subfile{01_einleitung}
\subfile{02_twitch}
\subfile{03_gamification}  
\subfile{04_konzept}
\subfile{05_diskussion}
\subfile{06_ausblick}  
  
\backmatter
%%%%%%%%%%%%%%%%%%%
%% create figure list
%%%%%%%%%%%%%%%%%%%

\listoffigures
\addcontentsline{toc}{chapter}{Verzeichnisse}

%%%%%%%%%%%%%%%%%%%
%% create tables list
%%%%%%%%%%%%%%%%%%%
\listoftables

%%%%%%%%%%%%%%%%%%%
%% create listings list
%%%%%%%%%%%%%%%%%%%
%\lstlistoflistings
%\addcontentsline{toc}{chapter}{Listings}

\cleardoublepage
\phantomsection
\addcontentsline{toc}{chapter}{Literatur}
\printbibliography

%%%%%%%%%%%%%%%%%%%
%% declaration on oath
%%%%%%%%%%%%%%%%%%%

\addchap{Eidesstattliche Erklärung}

Hiermit versichere ich, dass ich die vorgelegte Bachelorarbeit selbstständig verfasst und noch nicht anderweitig zu Prüfungszwecken vorgelegt habe. Alle benutzten Quellen und Hilfsmittel sind angegeben, wörtliche und sinngemäße Zitate wurden als solche gekennzeichnet.

\vspace{20pt}
\begin{flushright}
$\overline{~~~~~~~~~~~~~~~~~\mbox{\ShowBaAuthor, am \today}~~~~~~~~~~~~~~~~~}$
\end{flushright}

\addchap{Zustimmung zur Plagiatsüberprüfung}

Hiermit willige ich ein, dass zum Zwecke der Überprüfung auf Plagiate meine vorgelegte Arbeit in digitaler Form an PlagScan (www.plagscan.com) übermittelt und diese vorübergehend (max. 5~Jahre) in der von PlagScan geführten Datenbank gespeichert wird sowie persönliche Daten, die Teil dieser Arbeit sind, dort hinterlegt werden.

\begin{small}
Die Einwilligung ist freiwillig. Ohne diese Einwilligung kann unter Entfernung aller persönlichen Angaben und Wahrung der urheberrechtlichen Vorgaben die Plagiatsüberprüfung nicht verhindert werden. Die Einwilligung zur Speicherung und Verwendung der persönlichen Daten kann jederzeit durch Erklärung gegenüber der Fakultät widerrufen werden.
\end{small}

\vspace{20pt}
\begin{flushright}
$\overline{~~~~~~~~~~~~~~~~~\mbox{\ShowBaAuthor, am \today}~~~~~~~~~~~~~~~~~}$
\end{flushright}

\end{document}
