\documentclass[12pt,twoside,a4paper,parskip]{scrbook}
\usepackage[utf8]{inputenc}
\usepackage{csquotes}
\usepackage[ngerman]{babel}
\usepackage{floatflt}
\usepackage{subfigure}
\usepackage[pdftex]{graphicx}
\usepackage[hidelinks]{hyperref}
\usepackage{color}
\usepackage{amssymb}
\usepackage{textcomp}
\usepackage{nicefrac}
\usepackage{scrhack}
\usepackage{pdfpages}
\usepackage{float}
\usepackage{pdflscape}
\usepackage{subfigure}
\usepackage{pdfpages}
\usepackage[verbose]{placeins}
\usepackage[markcase=ignoreuppercase,headsepline,plainfootsepline]{scrlayer-scrpage}
\usepackage{listings}
\usepackage{xcolor}
\usepackage{color}
\usepackage{caption}
\usepackage{subfigure}
\usepackage{epstopdf}
\usepackage{longtable}
\usepackage{setspace}
\usepackage{booktabs}
\usepackage[style=numeric,backend=biber]{biblatex}
\usepackage{subfiles}
\bibliography{library}
 
%%%%%%%%%%%%%%%%%%%
%% definitions
%%%%%%%%%%%%%%%%%%%
\def\BaAuthor{Lukas Bauer}
\def\BaAuthorStudyProgram{Informatik} %% Wirtschaftsinformatik, E-Commerce, Informationssysteme
\def\BaType{Bachelorarbeit} %% Masterarbeit
\def\BaTitle{Gamification auf Twitch als Online-Marketinginstrument für Unternehmen in der Musikbranche}
\def\BaSupervisorOne{Prof.\ Dr.\ Isabel John}
\def\BaSupervisorTwo{Prof.\ Dr.\ Christina Völkl-Wolf}
\def\BaDeadline{\today}

\ifdefined\iswithfullname
  \def\ShowBaAuthor{\BaAuthor}
\else
  \def\ShowBaAuthor{N.~N.}
\fi

\hypersetup{
pdfauthor={\ShowBaAuthor},
pdftitle={\BaTitle},
pdfsubject={Computer Science and Marketing},
pdfkeywords={Gamification;Marketing;Twitch;Livestreaming;Music}
}

%%%%%%%%%%%%%%%%%%%
%% configs to include
%%%%%%%%%%%%%%%%%%%
\colorlet{punct}{red!60!black}
\definecolor{background}{HTML}{EEEEEE}
\definecolor{delim}{RGB}{20,105,176}
\colorlet{numb}{magenta!60!black}

\definecolor{gray}{rgb}{0.4,0.4,0.4}
\definecolor{darkblue}{rgb}{0.0,0.0,0.6}
\definecolor{cyan}{rgb}{0.0,0.6,0.6}

\definecolor{pblue}{rgb}{0.13,0.13,1}
\definecolor{pgreen}{rgb}{0,0.5,0}
\definecolor{pred}{rgb}{0.9,0,0}
\definecolor{pgrey}{rgb}{0.46,0.45,0.48}

\lstset{
  basicstyle=\ttfamily,
  columns=fullflexible,
  showstringspaces=false,
  commentstyle=\color{gray}\upshape
  linewidth=\textwidth
}

\lstdefinelanguage{json}{
    basicstyle=\normalfont\ttfamily,
    numbers=left,
    numberstyle=\scriptsize,
    stepnumber=1,
    numbersep=8pt,
    showstringspaces=false,
    breaklines=true,
    backgroundcolor=\color{background},
    literate=
     *{0}{{{\color{numb}0}}}{1}
      {1}{{{\color{numb}1}}}{1}
      {2}{{{\color{numb}2}}}{1}
      {3}{{{\color{numb}3}}}{1}
      {4}{{{\color{numb}4}}}{1}
      {5}{{{\color{numb}5}}}{1}
      {6}{{{\color{numb}6}}}{1}
      {7}{{{\color{numb}7}}}{1}
      {8}{{{\color{numb}8}}}{1}
      {9}{{{\color{numb}9}}}{1}
      {:}{{{\color{punct}{:}}}}{1}
      {,}{{{\color{punct}{,}}}}{1}
      {\{}{{{\color{delim}{\{}}}}{1}
      {\}}{{{\color{delim}{\}}}}}{1}
      {[}{{{\color{delim}{[}}}}{1}
      {]}{{{\color{delim}{]}}}}{1},
}

\lstset{language=xml,
  morestring=[b]",
  morestring=[s]{>}{<},
  morecomment=[s]{<?}{?>},
  stringstyle=\color{black},
  numbers=left,
  numberstyle=\scriptsize,
  stepnumber=1,
  numbersep=8pt,
  identifierstyle=\color{darkblue},
  keywordstyle=\color{cyan},
  backgroundcolor=\color{background},
  morekeywords={xmlns,version,type}% list your attributes here
}

\lstset{language=Java,
  showspaces=false,
  showtabs=false,
  tabsize=4,
  breaklines=true,
  keepspaces=true,
  numbers=left,
  numberstyle=\scriptsize,
  stepnumber=1,
  numbersep=8pt,
  showstringspaces=false,
  breakatwhitespace=true,
  commentstyle=\color{pgreen},
  keywordstyle=\color{pblue},
  stringstyle=\color{pred},
  basicstyle=\ttfamily,
  backgroundcolor=\color{background},
%  moredelim=[il][\textcolor{pgrey}]{$$},
%  moredelim=[is][\textcolor{pgrey}]{\%\%}{\%\%}
}

\newcommand*{\forcetwosidetitle}[1][1]{%
 \begingroup
   \cleardoubleoddpage
   \KOMAoptions{titlepage=true}% useful e.g. for scrartcl
   \csname @twosidetrue\endcsname
   \maketitle[{#1}]
 \endgroup
}


\begin{document}

\frontmatter
\titlehead{%  {\centering Seitenkopf}
  {Hochschule für angewandte Wissenschaften Würzburg-Schweinfurt\\
   Fakultät Informatik und Wirtschaftsinformatik}}
\subject{\BaType}
\title{\BaTitle\\[15mm]}
\subtitle{\normalsize{vorgelegt an der Hochschule f\"{u}r angewandte Wissenschaften W\"{u}rzburg-Schweinfurt in der Fakult\"{a}t Informatik und Wirtschaftsinformatik zum Abschluss eines Studiums im Studiengang \BaAuthorStudyProgram}}
\author{\ShowBaAuthor}
\date{\normalsize{Eingereicht am: \BaDeadline}}
\publishers{
  \normalsize{Erstpr\"{u}fer: \BaSupervisorOne}\\
  \normalsize{Zweitpr\"{u}fer: \BaSupervisorTwo}\\
}
\lowertitleback{
\centering\includegraphics[width=4cm]{qrcode-thesis}

}
\forcetwosidetitle


\section*{Zusammenfassung}

TODO

\section*{Abstract}

TODO

\newpage
\chapter*{Danksagung}

Ich Danke allen die mich bei dieser Arbeit unterstütz haben.

\tableofcontents

\mainmatter










% ###################################################################### Einleitung
\chapter{Einleitung}

\section{Ausgangssituation und Problemstellung}

Twitch als “Geheimtipp”

Besonders in Zeiten von Pandemien wie der Corona Krise im Jahr 2020 wird deutlich wie wichtig es ist als Unternehmen in der Musik und Veranstaltungsbranche online gut aufgestellt zu sein. Ansonsten sind mit starken Umsatzeinbrüchen und am Ende sogar mit entlassungen zu rechnen.


Heutzutage haben wir eine Vielzahl an online Video Content. Netflix, Amazon Prime, YouTube, Hulu und so weiter.

Werbung ist heutzutage im Internet allgegenwärtig. Sie wird immer moderner und raffinierter. Jedoch gewöhnen sich die Menschen recht schnell an neue Werbetechniken. Zudem sind durch die konstante technologische Entwicklung auch die Ansprüche der Menschen gestiegen. Neben den Produkten steigt auch der Anspruch an die Werbung. Um diesem Anspruch gerecht zu werden, hat sich eine ganze Branche um die Vermarktung von Produkten entwickelt. Die Online-Marketing Welt umfasst viele verschiedene Bereiche.

Unternehmen in der Musikbranche (Musikinstrumente, Technik, Clubs, Veranstalter und Künstler) nutzen beispielsweise mehr und mehr Social Media Plattformen um ihre Produkte und Dienstleistungen an den Konsumenten zu bringen. Hierdurch kann mit geringerem analogen Aufwand, die Kraft der digitalen Werbung effektiv genutzt wird.

Allerdings geht dieser Prozess nur schleppend voran und viel Potential wird nicht genutzt. Es fehlt eine Marketingstrategie die empirisch belegbar einen Erfolg bringt.

\section{Forschungsziel und -methode}

Hier kommt Gamification ins Spiel. Durch Gamification werden die Betrachtenden “direkter” angesprochen. Sie interagieren mit der Werbung wodurch eine Immersion erzeugt werden kann. Eine Social Media Plattform welche Gamification bereits erfolgreich einsetzt ist die soziale Live-Streaming Plattform Twitch. Jedoch noch nicht aktiv für Marketing.

In dieser Arbeit soll daher folgende Frage analysiert werden: Wie können die Gamification Elemente der Plattform Twitch von Firmen in der Musikbranche für Marketingzwecke eingesetzt werden?

Twitch hat eine sehr gut dokumentierte API mit welcher es möglich ist auch Erweiterungen und Apps zu schreiben die auf Gamification Events von Twitch reagieren und diese beeinflussen können.

Da Gamification als Werkzeug dient, die Motivation und das Interesse zu fördern, kann auch Werbung davon profitieren. Im Fokus dieser Bachelorarbeit stehen Unternehmen in der Musikbranche, welche mithilfe der Gamification die Betrachter auf emotionaler Ebene erreichen sollen. Dies hat den Vorteil dass dem Kunden oft gar nicht bewusst ist, dass es sich hierbei um Werbung handelt.

Seit 2011 gibt es auch Twitch.

\section{Aufbau der Arbeit}

In dieser Arbeit soll überprüft werden ob die Gamification Elemente der Platform Twitch auch als Marketing Instrumente gebraucht werden können.

Zu Beginn der Arbeit werden die Grundlagen erklärt. Dazu zählen eine Einführung in die Platform Twitch, eine Gamification übersicht dieser als auch einen Einblick in das Marketing auf der Platform.

Für die Phase der Durchführung wurde im Prozess der Requirements Engineerings auf ein Experteninterview gesetzt mit welchem die Anforderungen ermittelt wurden. Mit 

Mit den Ergebnissen wurde das Ziel der werbekampagne definiert.

Um diese Ziele anzugehen wurde dann eine technisch gestützte Kampagne prototypisch implementiert. Techn. Gestütz bedeutet in diesem Fall der Einsatz der Gamification Elemente und wie diese getriggert werden etc.

Um am Ende meinen Prototypen zu evaluieren wurde mit den Experten vom Anfang nochmal ein Interview machen und meinen Prototypen vorstellen. In dem Interview würde ich sie dann fragen ob sie Anhand ihrer Erfahrung sagen können ob das funktionieren würde oder nicht. Zusätzlich auch eventuelle Verbesserungen vorschlagen.










% ###################################################################### Twitch
\chapter{Twitch}
\section{Übersicht}
Twitch ist eine WebVideo Platform welche von Amazon betrieben wird. Sie wird hauptsächlich zum Streamen von Videspielen eingesetzt. Sie wird täglich von X Zuschauern besucht welche Y Streamer zuschauen.

Sie war zum Start der Platform noch unter dem namen Justin.tv bekannt. Aufgrund der hohen Anfrage von Videospiel Content auf dieser Seite wurde eine zweite Seite dediziert für Videospiele mit dem heute bekannten Namen Twitch gestartet welche im Jahr 2011 ans Netz ging. Später wurde auch der Name der Firma in Twitch umbenannt. Im JAhr 2014 wurde Twitch von Amazon gekauft und ist seitdem der Betreiber.


Jeder Stream wird einer Kategorie zugeordnet. Die beliebtesten Kategorie auf Twitch sind
-
-
-

Neben Videosielen werde auch viele andere Kategorien gestreamt
Just Chatting
Kochen
Musik

Monetarisiert 
Subs
Bits
Direkte Spenden

\section{Marketing auf Twitch}

Auf Twitch werden verschiedene Formen des Online Marketings angewandt. Im folgenden werden die verschiedenen Formen anhand von Beispielen erklärt.

Dass Twitch eine lukrative Platform ist merkte auch Amazon worufhin diese Twitch kauften.

\subsection{Video Marketing}
Das Video Marketing sind die altbekannten Werbespots welche auch in älteren Medien wie dem Fernseher zum einsatz kommen.

Auf Twitch wird diese Form im Videofeed des Streams eingesetzt. Hier werden Werbeclips zu verschiedenen Zeiten des Streams eingeblendet. Klickt ein Besucher auf einen Live Kanale wird noch bevor der Video Feed des Streamers gestartet wird ein Werbeclip eingeblendet.

Auch währden des Streams können Werbeclips eingeblendet werden. Hierüber hat der Streamer auch die Kontrolle wann, wie viele und wie lange die Clips eingeblendet werden. Nachdem ein Clip gestartet wurde setzt ein Cooldown ein welcher einen weiteren Werbeclip verhindert.

pre-roll, mid-roll, and post-roll ads
Hierbei handelt es sich um klassische Werbe Videos wie sie auch im Fernsehen zu sehen sind. Diese werde vor, während oder nach einem Livestream angezeigt. Klickt man beispielsweise auf einen Live Channel um dessen Stream beizutreten wird automatisch ein Werbeclip gestartet. Streamer haben auch die möglichkeit während dem Stream einen Werbeclip einzuplenden. Hierbei können Sie zwischen 15 sekunden und 3 Minuten wählen. Diese nutzen Sie beispielsweise um in möglichst spannenden Szenen des Streams die Aufmerksamkeit auf die Werbung zu lenken oder eine Pause um zB auf die Toilette zu gehen mit Werbung zu füllen.

Jedoch werden die Ads nicht angezeigt falls man einen Ad Blocker im Browser benutzt. Ist man Subscriber einer Channels oder hat für Twitch Turbo bezahlt werden die Werbeclips auch übersprungen.

\subsection{Display Marketing}
Eine der ältesten Marketing Methoden im digitalen Bereich ist vermutlich das Display Marketing. Hierbei werden auf einer Webseite verschiedene Banner eingeblendet die über Produkte oder Firmen informieren. Hierzu zählen auch Popups.

Oft werden diese auch zwischen normalen Content einer Webseite angezeigt um in der großen Konzentration eines Betrachters dessen Aufmerksamkeit auf ein Produkt zu lenken.

Banner Werbung
Firmen können auf Twitch Banner Werbung schalten. Diese wird in verschiedenen Bereichen der Platform angezeigt. 

\subsection{Content Marketing}
Bei Content Marketing geht es darum relevanten Content zu produzieren welcher die Aufgabe hat den Kunden zu informieren und die Marke in einem besonders guten Licht dastehen zu lassen.

Branded Content (Original Content)
- "Old Spice" livestream in dem der Chat entscheiden konnte was der Protagonist tun soll
- Duracel

\subsection{Affiliate marketing}
Affiliate Marketing ist ein provisionsbasierter Anzatz um ein produkt direkt an den Mann zu bringen. Hierbei werden können sich Content ersteller sogenannte Affiliate Links generieren welche direkt auf die Kaufsseite eines Produktes verweisen. Diesem Link ist ein Flag mitgegeben welches auf den Linkersteller zurückführt. Der Link wird in der Regel bei Produktreviews, Inventarlisten oder in Artikeln über die bestimmte Produktkategorie als “Empfehlung” eingesetzt.

Klickt ein Kunde nun auf einen solchen Link gelangt er direkt auf die Kaufsseite des Produktes. Kauft der Kunde sich dieses Produkt so zahlt der Verkäuft eine Provision an den Linkersteller aus. 

Manche Plattformen wie Amazon vergüten auch die reine referenzierung auf ihre Platform. Hat ein Kunde auf einen Affiliate Link geklickt und befindet sich auf der Produktseite, kauft jedoch das verlinkte Produkt nicht, sondern ein anderes auf der Platform so bekommt der Verlinker dennoch eine Provieion für die Weiterleitung auf die Platform.

\subsection{Influencer marketing}
Product Placements

Zum Start von ­­­„Alien: Convenant“ im Jahr 2017 führte 20th Century Fox eine Kampagne auf Twitch durch. Diese sollte eine junge Zielgruppe für den neuen Teil der beliebten Filmreihe erreichen. Da der letzte Film der Reihe bereits im Jahr 1997

Twitch mit einer Kampagne, die eine junge Zielgruppe für das Prequel der Kultfilmreihe begeistern sollte. Die Filmreihe wurde 1997 mit „Alien – Die Wiedergeburt“­ abgeschlossen – zu einer Zeit also, zu der durchschnittliche Twitch-User noch zu jung waren, um zur Zielgruppe zu gehören. Um sie also als neues Publikum zu erschließen, musste das Horror­gefühl der frühen Alienfilme mit den Ansprüchen an dynamischen Live-Content verbunden werden. Vier bekannte Streamer aus Großbritannien, Deutschland, Frankreich und Russland konnten für eine Kooperation gewonnen werden. Allerdings bestand die Kampagne nicht daraus, dass die Streamer den neuen Film besprachen oder sich selbst beim Anschauen des Trailers filmten, wie man es von Plattformen wie Youtube kennt. Stattdessen fand in den regulären Live-Streams der ­Partner eine Alien-Invasion statt. Flackerndes Licht, Bildstörungen, lange Schleimfäden, die sich von der Decke ziehen. Traditionelle ­Horrorfilm-Techniken mit echten CGI-Alien-Aufnahmen aus dem Film kombiniert und die Zuschauer waren live dabei. Die Live-Alien-Attacke erzielte 38.888 Unique Views mit einer Gesamtzeit von 3.591 Stunden. Der platzierte Filmtrailer erreichte 549.645 Views und 35.255 Engagements.










% ###################################################################### Gamification
\chapter{Gamification}

\section{Grundlagen}

Gamification ist ein Begriff  der bereits seit einigen Jahren in vielen Bereichen anwendung findet. Nachdem lange Verwirrung um eine genaue Definition vom Gamification bestand definierte Deterding es wir folgt:

“Gamification is the use of game design elements in non-game contexts" \cite{Deterding2011}

Hierbei handelt es sich um die Verwendung von Elementen aus Spielen in nicht-spiel Umgebungen bzw. Kontexten. 

Die Hintergründe des ganzen sind dass Spiele eine Starke Motivation im Benutzer bzw Spieler erzeugen. Diese Motivation wird spezifischen Elementen aus Spielen zugeschrieben. Diese sind im Allgemeinen Punkte, Badges oder Leaderboards (PBLs)

Jedoch besteht auch die verwirrung dass angenommen wird dass das alleinige hinzufügen dieser Elemente bereits Gamification ist. Jedoch ist Gamification viel mehr. Es ist das gezielte ansprechen von Motivationsfaktoren in der Menschlichen Psyche

Pointification ist das pure “draufklatschen” von PBLs auf ein System, in der Hoffnung dass es den Benutzer motiviert.

Hierzu kann auf verschiedene Frameworks zurückgegriffen werden. Diese sind Beispielsweise das Gamifcation Model Canvas (GMC) welches vom Business Model Canvas abgeleitet wurde. 

Zum anderen aber auch das Octalysis Framework von YouKaiChou welches die Motivation des Menschen in 8 Teilbereiche unterteilt. Daher auch “Octal”-ysis. In diesem Framework haben wir das Octalysis Strategy Dashboads. Dieses wird dazu eingezetzt ganz spezifisch herauszufinden welche Gamification Elemente eingesetzt werden um vordefinierstes Verhalten auszulösen.

\section{Gamification auf Twitch}

Da Twtich eine Platform ist welche Ursprücnglich nur für das Live Streamen von Videospielen gabaut wurde ist es nicht verwunderlich dass diese sich auch die Kraft der Gamification zu nutze macht. Sowohl die Seite der Zuschauer als auch die Streamer Seite ist mit Elementen aus Spielen übersäht. 

\subsection{Streamer Seite}
Channel Analytics
Stream Summary
Achievements


\subsection{Zuschauer Seite}


*Achievements*

* Alerts im Stream
* Popup im Chat & Mention vom Bot
* Mention vom Streamer

*Badges* 

* Sub Badges
    * 1st
    * Top Leaderboard 1-3
* Bits Badges
    * 1-1M
    * Top Leaderboard 1-3
* VIP
* Mod
* Emotes?!


*Progressbars*
Hypetrain
Goals im Stream
Leaderboards
Top 3 Bits & Subgifter über Chat
Top 1 im Stream
Last im Stream
Points
Channelpoints
Früher Chat Points über 3rd Party
Ingame Währung
Bits


Hier gibt es zum einen die Subscribtions welche aufgeteilt in drei "Tiers". Drei Stufen mit den Preisen 4.99, 9.99 und 24.99.

Subs verschenken
Leaderboards

Vorteile:
Werbefrei schauen
Community Gefühl!!!!!
emotes
Badge im Chat
recuring bekommen coolere badges
höhere tiers bekommen zusätzlichen flair

Channelpoints
* Vorgefertigte
  * Nachricht hervorheben
  * Unlock zufälliges Sub emote
  * Nachricht im Sub Only mode
  * Unlock spezifisches Sub emotes
  * ein emote bearbeiten

Channelpoints verdienen
* Watch for 5 minutes +10
* Claim special bonuses +50
* Participate in a Raid +250
* Follow this channel +300
* Monthly 1st Cheer +350
* Monthly 1st Gift a Sub +500
* Grow a watch streak up to +450
* Tier 1 sub 1.2x multiplier
* Tier 2 sub 1.4x multiplier
* Tier 3 sub 2x multiplier

Bits

HypeTrain
Der HypeTrain ist ein Event welches eintritt sobald eine Bits Spende in einer bestimmten Höhe oder eine bestimmte anzahl an Subscriptions

Drops

Mit steigender Beliebtheit der Platform steigt auch die Anzahl der täglichen Nutzer rapide an.











% ###################################################################### Konzept
\chapter{Konzeption und Implementierung}

\section{Anforderungsanalyse}

Für die Anforderungsanalyse wird auf die Prozesse des Requirement Engineerings gesetzt. Qualitative Datenerhebung mit einem semistrukturiertem Interview. In diesem werden Fragen gestellt um dem Interview eine Thematische Richtung zu geben, es aber nicht zu sehr einzuschränken. Für dieses Interview wurde ein Leitfaden erstellt. In diesem wird der Ablauf genau definiert und dient während des Gespräches als Orientierung.

Interviewleitfaden

Forschungsfrage: Wie können die Gamification Elemente der Plattform Twitch von Firmen in der Musikbranche für Marketingzwecke eingesetzt werden?

Einstieg

- Begrüßung des Interviewees und Bedanken für die Teilnahme

- Kurze Einführung in das Thema

- Erklären des Interview Leitfadens

- Datenschutzvereinbarung

Einstigsfragen
- Was genau ist ihre aktuelle Profession und wie lange sind Sie dort schon beschäftigt?
- Was gehört hierbei zum Tagesgeschäft?

Schlüsselfragen

Frage 1: Was sind die Revenue Streams eines Unternehmens in der Musik Branche?

Frage: Welche Zielgruppe ist besonders wichtig?

Frage 2: Was sind die aktuellen Wege einen Kunden / Fan zu erreichen?

Frage 3: Was ist wichtig im Umgang mit einem Kunden / Fan in ihrer Branche? zB Umgangston

Rückblick
- Zusammenfassung der Mitschriften
- Danke für die Teilnahme

Ausblick
- Information über Verwendung der Informationen
- Verabschiedung

\subsection{Erhebung durch Experteninterviews}

\subsection{Zusammenfassen der Ergebnisse}
 
Inhaltsanalyse nach Kuckartz

Auf eine Transkription des Interviews wurde hierbei verzichtet.

\subsection{Definition der Anforderungen}

\section{Architekturkonzept}

\section{Prototypische Implementierung}










% ###################################################################### Diskussion
\chapter{Diskussion}

\section{Evaluation}

Qualitative evaluation mit einem offenen Interview

\section{Kritische Betrachtung}

Da in dieser Arbeit zur Evalation des Ergebnisses nur Experteninterviews durchgeführt wurden ist dieses Ergebnis eher kritisch zu betrachten. Dieser eher subjektiven Ergebnisse kann nicht zu hundert prozent vertraut werden. Hierfür wäre eine Feldtest Studie empehlenswert um Statistisch beweisbare Ergebnisse zu erzeugen.

\section{Empfehlung für zukünftige Forschungen}
“Neid Faktor” der Achievements im Stream










% ###################################################################### Ausblick


\chapter{Fazit und Ausblick}

Ausblick







% ###################################################################### ENDE
\backmatter

\listoffigures
\addcontentsline{toc}{chapter}{Verzeichnisse}

\listoftables

%% create listings list
%  \lstlistoflistings
%  \addcontentsline{toc}{chapter}{Listings}

\cleardoublepage
\phantomsection
\addcontentsline{toc}{chapter}{Literatur}
\printbibliography

\addchap{Eidesstattliche Erklärung}

Hiermit versichere ich, dass ich die vorgelegte Bachelorarbeit selbstständig verfasst und noch nicht anderweitig zu Prüfungszwecken vorgelegt habe. Alle benutzten Quellen und Hilfsmittel sind angegeben, wörtliche und sinngemäße Zitate wurden als solche gekennzeichnet.

\vspace{20pt}
\begin{flushright}
$\overline{~~~~~~~~~~~~~~~~~\mbox{\ShowBaAuthor, am \today}~~~~~~~~~~~~~~~~~}$
\end{flushright}

\addchap{Zustimmung zur Plagiatsüberprüfung}

Hiermit willige ich ein, dass zum Zwecke der Überprüfung auf Plagiate meine vorgelegte Arbeit in digitaler Form an PlagScan (www.plagscan.com) übermittelt und diese vorübergehend (max. 5~Jahre) in der von PlagScan geführten Datenbank gespeichert wird sowie persönliche Daten, die Teil dieser Arbeit sind, dort hinterlegt werden.

\begin{small}
Die Einwilligung ist freiwillig. Ohne diese Einwilligung kann unter Entfernung aller persönlichen Angaben und Wahrung der urheberrechtlichen Vorgaben die Plagiatsüberprüfung nicht verhindert werden. Die Einwilligung zur Speicherung und Verwendung der persönlichen Daten kann jederzeit durch Erklärung gegenüber der Fakultät widerrufen werden.
\end{small}

\vspace{20pt}
\begin{flushright}
$\overline{~~~~~~~~~~~~~~~~~\mbox{\ShowBaAuthor, am \today}~~~~~~~~~~~~~~~~~}$
\end{flushright}

\end{document}
