\chapter{Einleitung}

\section{Ausgangssituation und Problemstellung}

Twitch als “Geheimtipp”

Besonders in Zeiten von Pandemien wie der Corona Krise im Jahr 2020 wird deutlich wie wichtig es ist als Unternehmen in der Musik und Veranstaltungsbranche online gut aufgestellt zu sein. Ansonsten sind mit starken Umsatzeinbrüchen und am Ende sogar mit entlassungen zu rechnen.


Heutzutage haben wir eine Vielzahl an online Video Content. Netflix, Amazon Prime, YouTube, Hulu und so weiter.

Werbung ist heutzutage im Internet allgegenwärtig. Sie wird immer moderner und raffinierter. Jedoch gewöhnen sich die Menschen recht schnell an neue Werbetechniken. Zudem sind durch die konstante technologische Entwicklung auch die Ansprüche der Menschen gestiegen. Neben den Produkten steigt auch der Anspruch an die Werbung. Um diesem Anspruch gerecht zu werden, hat sich eine ganze Branche um die Vermarktung von Produkten entwickelt. Die Online-Marketing Welt umfasst viele verschiedene Bereiche.

Unternehmen in der Musikbranche (Musikinstrumente, Technik, Clubs, Veranstalter und Künstler) nutzen beispielsweise mehr und mehr Social Media Plattformen um ihre Produkte und Dienstleistungen an den Konsumenten zu bringen. Hierdurch kann mit geringerem analogen Aufwand, die Kraft der digitalen Werbung effektiv genutzt wird.

Allerdings geht dieser Prozess nur schleppend voran und viel Potential wird nicht genutzt. Es fehlt eine Marketingstrategie die empirisch belegbar einen Erfolg bringt.

\section{Forschungsziel und -methode}

Hier kommt Gamification ins Spiel. Durch Gamification werden die Betrachtenden “direkter” angesprochen. Sie interagieren mit der Werbung wodurch eine Immersion erzeugt werden kann. Eine Social Media Plattform welche Gamification bereits erfolgreich einsetzt ist die soziale Live-Streaming Plattform Twitch. Jedoch noch nicht aktiv für Marketing.

In dieser Arbeit soll daher folgende Frage analysiert werden: Wie können die Gamification Elemente der Plattform Twitch von Firmen in der Musikbranche für Marketingzwecke eingesetzt werden?

Twitch hat eine sehr gut dokumentierte API mit welcher es möglich ist auch Erweiterungen und Apps zu schreiben die auf Gamification Events von Twitch reagieren und diese beeinflussen können.

Da Gamification als Werkzeug dient, die Motivation und das Interesse zu fördern, kann auch Werbung davon profitieren. Im Fokus dieser Bachelorarbeit stehen Unternehmen in der Musikbranche, welche mithilfe der Gamification die Betrachter auf emotionaler Ebene erreichen sollen. Dies hat den Vorteil dass dem Kunden oft gar nicht bewusst ist, dass es sich hierbei um Werbung handelt.

Seit 2011 gibt es auch Twitch.

\section{Aufbau der Arbeit}



In dieser Arbeit soll überprüft werden ob die Gamification Elemente der Platform Twitch auch als Marketing Instrumente gebraucht werden können.
