\documentclass[12pt,twoside,a4paper,parskip]{scrbook}
\usepackage[utf8]{inputenc}
\usepackage{csquotes}
\usepackage[ngerman]{babel}
\usepackage{floatflt}
\usepackage{subfigure}
\usepackage[pdftex]{graphicx}
\usepackage[hidelinks]{hyperref}
\usepackage{color}
\usepackage{amssymb}
\usepackage{textcomp}
\usepackage{nicefrac}
\usepackage{scrhack}
\usepackage{pdfpages}
\usepackage{float}
\usepackage{pdflscape}
\usepackage{subfigure}
\usepackage{pdfpages}
\usepackage[verbose]{placeins}
\usepackage[markcase=ignoreuppercase,headsepline,plainfootsepline]{scrlayer-scrpage}
\usepackage{listings}
\usepackage{xcolor}
\usepackage{color}
\usepackage{caption}
\usepackage{subfigure}
\usepackage{epstopdf}
\usepackage{longtable}
\usepackage{setspace}
\usepackage{booktabs}
\usepackage[style=apa,backend=biber]{biblatex}
\usepackage{subfiles}
\usepackage{booktabs, multirow} % for borders and merged ranges
\usepackage{soul}% for underlines
%\usepackage[table]{xcolor} % for cell colors
\usepackage{changepage,threeparttable} % for wide tables
\usepackage[acronym]{glossaries} % AK Verzeichnis
\bibliography{referencesmendsync}
 
%%%%%%%%%%%%%%%%%%%
%% definitions
%%%%%%%%%%%%%%%%%%%
\def\BaAuthor{Lukas Bauer}
\def\BaAuthorStudyProgram{Informatik} %% Wirtschaftsinformatik, E-Commerce, Informationssysteme
\def\BaType{Bachelorarbeit} %% Masterarbeit
\def\BaTitle{Gamification auf Twitch als Online-Marketinginstrument für Unternehmen in der Musikbranche}
\def\BaSupervisorOne{Prof.\ Dr.\ Isabel John}
\def\BaSupervisorTwo{Prof.\ Dr.\ Christina Völkl-Wolf}
\def\BaDeadline{\today}

\ifdefined\iswithfullname
  \def\ShowBaAuthor{\BaAuthor}
\else
  \def\ShowBaAuthor{N.~N.}
\fi

\hypersetup{
pdfauthor={\ShowBaAuthor},
pdftitle={\BaTitle},
pdfsubject={Computer Science and Marketing},
pdfkeywords={Gamification;Marketing;Twitch;Livestreaming;Music}
}

%%%%%%%%%%%%%%%%%%%
%% configs to include
%%%%%%%%%%%%%%%%%%%
\colorlet{punct}{red!60!black}
\definecolor{background}{HTML}{EEEEEE}
\definecolor{delim}{RGB}{20,105,176}
\colorlet{numb}{magenta!60!black}

\definecolor{gray}{rgb}{0.4,0.4,0.4}
\definecolor{darkblue}{rgb}{0.0,0.0,0.6}
\definecolor{cyan}{rgb}{0.0,0.6,0.6}

\definecolor{pblue}{rgb}{0.13,0.13,1}
\definecolor{pgreen}{rgb}{0,0.5,0}
\definecolor{pred}{rgb}{0.9,0,0}
\definecolor{pgrey}{rgb}{0.46,0.45,0.48}

\lstset{
  basicstyle=\ttfamily,
  columns=fullflexible,
  showstringspaces=false,
  commentstyle=\color{gray}\upshape
  linewidth=\textwidth
}

\lstdefinelanguage{json}{
    basicstyle=\normalfont\ttfamily,
    numbers=left,
    numberstyle=\scriptsize,
    stepnumber=1,
    numbersep=8pt,
    showstringspaces=false,
    breaklines=true,
    backgroundcolor=\color{background},
    literate=
     *{0}{{{\color{numb}0}}}{1}
      {1}{{{\color{numb}1}}}{1}
      {2}{{{\color{numb}2}}}{1}
      {3}{{{\color{numb}3}}}{1}
      {4}{{{\color{numb}4}}}{1}
      {5}{{{\color{numb}5}}}{1}
      {6}{{{\color{numb}6}}}{1}
      {7}{{{\color{numb}7}}}{1}
      {8}{{{\color{numb}8}}}{1}
      {9}{{{\color{numb}9}}}{1}
      {:}{{{\color{punct}{:}}}}{1}
      {,}{{{\color{punct}{,}}}}{1}
      {\{}{{{\color{delim}{\{}}}}{1}
      {\}}{{{\color{delim}{\}}}}}{1}
      {[}{{{\color{delim}{[}}}}{1}
      {]}{{{\color{delim}{]}}}}{1},
}

\lstset{language=xml,
  morestring=[b]",
  morestring=[s]{>}{<},
  morecomment=[s]{<?}{?>},
  stringstyle=\color{black},
  numbers=left,
  numberstyle=\scriptsize,
  stepnumber=1,
  numbersep=8pt,
  identifierstyle=\color{darkblue},
  keywordstyle=\color{cyan},
  backgroundcolor=\color{background},
  morekeywords={xmlns,version,type}% list your attributes here
}

\lstset{language=Java,
  showspaces=false,
  showtabs=false,
  tabsize=4,
  breaklines=true,
  keepspaces=true,
  numbers=left,
  numberstyle=\scriptsize,
  stepnumber=1,
  numbersep=8pt,
  showstringspaces=false,
  breakatwhitespace=true,
  commentstyle=\color{pgreen},
  keywordstyle=\color{pblue},
  stringstyle=\color{pred},
  basicstyle=\ttfamily,
  backgroundcolor=\color{background},
%  moredelim=[il][\textcolor{pgrey}]{$$},
%  moredelim=[is][\textcolor{pgrey}]{\%\%}{\%\%}
}

\newcommand*{\forcetwosidetitle}[1][1]{%
 \begingroup
   \cleardoubleoddpage
   \KOMAoptions{titlepage=true}% useful e.g. for scrartcl
   \csname @twosidetrue\endcsname
   \maketitle[{#1}]
 \endgroup
}

\newcommand*{\inlineimg}[1]{%
    \raisebox{-.3\baselineskip}{%
        \includegraphics[
        height=\baselineskip,
        width=\baselineskip,
        keepaspectratio,
        ]{#1}%
    }%
}

\DeclareLabeldate{%
  \field{year}
  \literal{nodate}
}

\makeglossaries
\newacronym{hfd}{HFD}{Human-Focused  Design}
\newacronym{pbls}{PBLs}{Punkte, Badges oder Leaderboards}
\newacronym{obs}{OBS}{Open Broadcaster Software}
\newacronym{rtmp}{RTMP}{Real Time Messaging Protocol}
\newacronym{gdf}{GDF}{Gamification Design Framework}
\newacronym{gmc}{GMC}{Gamification Model Canvas}
\newacronym{osd}{OSD}{Octalysis Strategy Dashboads}
\newacronym{npc}{NPC}{Non-player character}

\begin{document}
\frontmatter
\titlehead{%  {\centering Seitenkopf}
  {Hochschule für angewandte Wissenschaften Würzburg-Schweinfurt\\
   Fakultät Informatik und Wirtschaftsinformatik}}
\subject{\BaType}
\title{\BaTitle\\[15mm]}
\subtitle{\normalsize{vorgelegt an der Hochschule f\"{u}r angewandte Wissenschaften W\"{u}rzburg-Schweinfurt in der Fakult\"{a}t Informatik und Wirtschaftsinformatik zum Abschluss eines Studiums im Studiengang \BaAuthorStudyProgram}}
\author{\ShowBaAuthor}
\date{\normalsize{Eingereicht am: \BaDeadline}}
\publishers{
  \normalsize{Erstpr\"{u}ferin: \BaSupervisorOne}\\
  \normalsize{Zweitpr\"{u}ferin: \BaSupervisorTwo}\\
}
\lowertitleback{
\centering\includegraphics[width=4cm]{qrcode-thesis}
}
\forcetwosidetitle


\section*{Zusammenfassung}

Die Marketing Methoden haben sich vor allem durch die Etablierung des Internets in den vergangenen Jahren von einer sehr sachlichen hin zu einer unterschwelligen und subtilen Art entwickelt. Hierbei spielt Gamification eine große Rolle. Eine Plattform die Gamification bereits erfolgreich einsetzt ist die soziale Live-Streaming-Plattform Twitch. Durch die Corona-Pandemie wurde deutlich, dass Unternehmen in der Musikbranche von dieser Plattform profitieren können. Daher soll in dieser Arbeit untersucht werden, wie diese Firmen die Gamification-Elemente der Plattform Twitch für Marketingzwecke einsetzen können. Um diese Frage zu beantworten wird zunächst anhand von Literatur eine grobe Übersicht über das Themenfeld erstellt. Mit dieser Grundlage wurden Experteninterviews durchgeführt um anhand von möglichen Beispielen die Einsetzbarkeit von Gamification-Elementen zu überprüfen. Die Gamification-Elemente der Plattform Twitch können im Branded Content-Marketing, Affiliate-Marketing und Influencer-Marketing eingesetzt werden. Von den nativ in Twitch implementierten Gamification-Elementen können lediglich die VIP-Badges verwendet werden. Da Overlays und Bots kreative Freiheit bieten, können auch diese Anwendung finden. Aufgrund der limitierten Zeit und Möglichkeiten in dieser Arbeit konnten die Ergebnisse noch nicht überprüft werden. Dies sollte in einer weiterführenden Studie nachgeholt werden.

\section*{Abstract}

Marketing methods have developed from a very objective to a subliminal and subtle way, mainly due to the establishment of the Internet in recent years. Gamification plays a major role here. One platform that Gamification already uses successfully is the social live streaming platform Twitch. The Corona Pandemic made it clear that companies in the music industry can benefit from this platform. Therefore, this thesis should investigate how these companies can use the gamification elements of the Twitch platform for marketing purposes. In order to answer this question, a rough overview of the topic area is first created using literature. Based on this, expert interviews were conducted to check the applicability of gamification elements on the basis of possible examples. The gamification elements of the Twitch platform can be used in branded content marketing, affiliate marketing and influencer marketing. Of the gamification elements natively implemented in Twitch, only the VIP badges can be used. Since overlays and bots offer creative freedom, these can also be used. Due to the limited time and possibilities in this work, the results could not yet be verified. This should be made up for in further studies.

\tableofcontents

\mainmatter
%
%
%
%
%
%
%
%
%
%
%
\chapter{Einleitung}
\section{Ausgangssituation und Problemstellung}
\label{chap:ausg_propl}

Werbung ist heutzutage im Internet allgegenwärtig. Sie wird immer moderner und raffinierter. In den frühen 90ern als die Marketingbranche begonnen hat zu flukturieren, waren besonders sachliche Werbungen erfolgreich. Sie stellten lediglich die Besonderheiten eines Produktes heraus um einen Kaufreiz zu erzeugen. Im Verlauf der Zeit entwickelte sich die Art und Weise der Werbetechniken mehr von \glqq Push\grqq{} zu \glqq Pull\grqq{}.  Hierbei geht es darum, dass Handeltreibende nicht in erster Linie die Kundschaft von einem Produkt überzeugen will. Vielmehr geht es darum bei den Kaufenden einen emotionalen Bezug zum Produkt aufzubauen. Die Produkte werden nicht aus rationalen Gründen gekauft, sondern aus emotionalen. Die Werbebotschaften werden hier viel subtiler und unterschwelliger vermittelt. Diese subtile Art der Werbung entwickelte sich besonders rasant mit der steigenden Popularität des Internets. Vor allem im Zuge von Social Media wurden  Content-Marketing, Influencer-Marketing und Product-Placements immer gefragter. \parencite{Kopp2013}  

Auch Unternehmen in der Musikbranche nutzen immer mehr Social Media Plattformen um ihre Produkte und Dienstleistungen an die Konsumierenden zu bringen. Mit einem geringerem Aufwand an analogen Mitteln, kann die Kraft der digitalen Werbung effektiv genutzt werden. \parencite{SocialFactor2020} 

Gerade die Corona Pandemie im Jahr 2020 hat gezeigt, wie wichtig es ist als Unternehmen in der Musik und Veranstaltungsbranche online gut aufgestellt zu sein. Da keine Veranstaltungen stattfinden dürfen bricht somit 50\% des Umsatzes weg. Ganz besonders die Verkaufszahlen von Tickets für Live-Präsenz-Veranstaltungen sind hiervon stark betroffen. \parencite{Stefan2020} 

Hierbei bietet die Live-Streaming-Plattform Twitch eine gute Möglichkeit um die fehlenden Veranstaltungen digital stattfinden zu lassen. So verlegten viele Musizierende ihre Auftritte auf diese Plattform. \parencite{Wedekind2020} Wie die \autoref{img:twitch_q20} zeigt, sind die Zuschauerzahlen von Twitch parallel zum Corona Ausbruch rasant angestiegen. Mit 814 Mio. gestreamten Stunden Live-Content ist Twitch gegenüber YouTube, auf welcher nur 226 Mio. Stunden Live-Content gestreamt wurde, ein echter Geheimtipp für Marketer \parencite{Sturm2019}.

\begin{figure}[ht]
\caption{Twitch durchschnittliche Zuschauer Q2'18 bis Q2'20 \parencite{May2020}}
\label{img:twitch_q20}
\centering
\includegraphics[width=\textwidth]{twitch_viewers_2020}
\end{figure}

\section{Forschungsziel und -methode}
Twitch ist für viele Musizierende eine komplett neue Erfahrung und der Markt für Musik auf dieser Plattform noch nahezu unberührt. Die Musizierenden können die Plattform nutzen um sich einer komplett neuen Zielgruppe zu offenbaren, aber auch ihren bisherigen Fans eine komplett neue Erfahrung zu bieten. Eine Besonderheit von Twitch ist die mit Gamification-Elementen gestaltete Monetarisierung. \parencite{Allen2019}

Twitch setzt auch in vielen weiteren Bereichen ihrer Plattform auf Gamification-Elemente. Gamification wird als ein starkes Tool für das Marketing angesehen \parencite{Zichermann2011}. Durch Gamification werden die Betrachtenden auf emotionaler Ebene angesprochen. Sie interagieren mit der Werbung, wodurch diese stärker in das System eingebunden werden. 

In Gamification birgt sich viel Potential, da die benutzenden Personen Spaß dabei haben mit der Werbung zu interagieren. In dieser Arbeit soll daher folgende Frage analysiert werden: \textbf{Wie können die Gamification-Elemente der Plattform Twitch von Firmen in der Musikbranche für Marketingzwecke eingesetzt werden?}

Twitch hat eine sehr gut dokumentierte Programmierschnittstelle (API). Mit dieser ist es möglich Erweiterungen und Anwendungen zu entwickeln, die auf Gamification-Events von Twitch reagieren und diese gegebenenfalls beeinflussen können.

Da Gamification als Werkzeug dient, die Motivation und das Interesse zu fördern, kann auch das Marketing davon profitieren. Im Fokus dieser Bachelorarbeit stehen Unternehmen in der Musikbranche, welche mithilfe der Gamification die Betrachtenden auf emotionaler Ebene erreichen sollen. Dies hat den Vorteil, dass der Kundschaft oft gar nicht bewusst ist, dass es sich hierbei um Marketingmethoden handelt.

\section{Aufbau der Arbeit}
In dieser Arbeit soll überprüft werden, ob und wie die Gamification-Elemente der Plattform Twitch als Marketing-Instrumente eingesetzt werden können. Hierfür müssen zu Beginn der Arbeit die Grundlagen analysiert werden. 

Zunächst geht es darum, die Eigenheiten der Plattform Twitch und die dort vorherrschende Community-Kultur näher zu analysieren. Im weiteren geht es um die Monetarisierung auf Twitch und die besonderen Mechaniken wie diese begünstigt wird. Danach folgt eine allgemeine Einführung in Gamification. Hier wird kurz auf die Spielertypen eingegangen, sowie die \glqq Core Drives\grqq{} und das \glqq Design Framework\grqq{} von YouKai Chou näher erklärt.

Im nächsten Punkt wird speziell die Gamification auf Twitch untersucht. Hierbei wurde die Plattform über mehrere Wochen konsumiert, um möglichst viele Gamification-Elemente auf der Plattform zu finden. Die Einordnung als Gamification Element geschieht nach eigener Einschätzung. Außerdem wird aufgezeigt, welche Marketing-Methoden auf Twitch Anwendung finden und in welchen Gamification verwendet werden kann.

Nach der Klärung der Grundlagen wird eine beispielhafte Gamification-Kampagne konzipiert. Anhand von verschiedenen Experteninterviews sollten hierfür zunächst die Anforderungen einer solchen Kampagne herausgefunden werden. Für diese wurde ein Interviewleitfaden erstellt, der den Befragungen eine grobe Richtung vorgab. Es wurden drei Musik-Streamer für diese Interviews ausgewählt. Auch wurden in den Experteinterviews die Hintergründe und Funktionsweise der Musikindustrie erfragt.

Damit die Befragungen auch wissenschaftlich verwendet werden können, wurde eine Auswertung nach Kuckartz durchgeführt. Mit den Ergebnissen wird das Ziel und die Rahmenbedingungen der Werbekampagne definiert. Zum Schluss wurde eine beispielhafte Kampagne konzipiert.

%
%
%
%
%
%
%
%
%
%
% 
\chapter{Grundlagen}
\section{Twitch}
\label{chap:grund_twitch}

Twitch ist eine Live-Web-Video-Plattform, die von Amazon betrieben wird. Im März 2020 verzeichnete das Portal durchschnittlich 1,44 Millionen aktive Zuschauer gleichzeitig. Im selben Zeitraum streamten im Durchschnitt 56.000 Streamerinnen und Streamer die konsumierten Inhalte. Im April 2020 erreichte Twitch Platz 33 im Vergleich mit allen anderen Webseiten in puncto Datenverkehr. \parencite{Iqbal2020}

Twitch entstand als Nebenprojekt aus der Plattform Justin.tv, die im Jahr 2007 gegründet wurde. Justin.tv bot jedem Menschen mit Internetzugang die Möglichkeit kostenlos eigene Liveshows ins Internet zu streamen, sogenannte \glqq Broadcasts\grqq{}. Aufgrund der hohen Anfrage von Videospiel-Content wurde eine zweite Seite, dediziert für Videospiele, gestartet. Sie ging unter dem heute bekannten Namen Twitch im Jahr 2011 ans Netz. Später wurde auch der Name der Firma in Twitch umbenannt und Justin.tv wurde vom Netz genommen, da sich die Firma nur auf Twitch konzentrieren wollte. Im Jahr 2014 wurde Twitch von Amazon gekauft. \parencite{Ionos2018}

Auf Twitch gibt es die sogenannten \glqq Channels\grqq{}. Jedes Twitch-Mitglied hat einen eigenen Channel über den es seine Livestreams anderen Mitgliedern zur Verfügung stellen kann. Eine Person die einen solchen Livestream bereitstellt wird auch \glqq Streamerin und Streamer\grqq{}, \glqq Broadcasterin und Broadcaster\grqq{} oder \glqq Content-Creator\grqq{} genannt. Findet ein Person einen anderen Channel interessant und möchte benachrichtigt werden, sobald dieser Live geht, so kann sie diesem folgen. Jeder Channel hat eine eigene Seite und Internetadresse, auf der der Channel-Betreiber in der \glqq About\grqq{}-Sektion Texte und Bilder einbauen kann. Diese Sektion wird für Infos über die Streamerinnen und Streamer als Personen, Infos und Links von Werbepartnern oder Links zu anderen Auftritten im Internet genutzt. Wird über den Channel gerade ein Livestream übertragen, so wird das Videobild des Livestreams wie in \autoref{img:twitch_channel} groß und zentral auf dem Channel angezeigt. Auf der linken Seite im Bild ist eine Liste mit Channels zu sehen denen der gerade angemeldete Account folgt. Hierbei werden Channels, die gerade live sind, ganz oben in der Liste, zusammen mit ihren aktuellen Zuschauerzahlen, angezeigt.

Die Besonderheit von Twitch ist die Kombination aus High-Fidelity-Video-Content und textbasierter Kommunikation über einen Chat \parencite{Hamilton2014}. Hierbei haben Zuschauerinnen und Zuschauer in einem Livestream die Möglichkeit in Echtzeit untereinander und mit den Streamenden zu kommunizieren. Der Chat wird, wie in \autoref{img:twitch_channel} zu sehen, rechts neben dem Videobild des Livestreams angezeigt. 

Die wohl charakteristischsten Elemente im Chat sind die sogenannten \glqq Emotes\grqq{}. Dies sind kleine Bildzeichen, die im Live-Chat eingesetzt werden können, um mehr Emotionen im Text auszudrücken. Sie sind Kernbestandteil der Sprache auf Twitch. Die sogenannten \glqq Global-Emotes\grqq{} sind für alle Mitglieder gleichermaßen verfügbar. Es gibt aber auch Emotes die erst freischalten werden muss. Das beliebteste Global-Emote ist \glqq Kappa\grqq{} (\inlineimg{kappa}). Es zeigt die schwarz-weiß Abbildung des Gesichts eines ehemaligen Twitch Entwicklers. \parencite{Barbieri2018}

\begin{figure}[ht]
\caption{Screenshot eines Twitch Livestreams von \parencite{Twitchf}}
\label{img:twitch_channel}
\centering
\includegraphics[width=\textwidth]{twitch_live_screen}
\end{figure}

Jeder Livestream wird von den Streamerinnen und Streamern einer Kategorie zugeordnet. Sie spiegelt dessen Inhalt wider. Diese erleichtern das Auffinden von interessanten Inhalten für die Konsumierenden. So werden auf der Startseite Live-Channels angezeigt, die gerade Kategorien streamen, die den Interessen des angemeldeten Accounts entsprechen. Zudem bietet Twitch über die \glqq Browse\grqq{}-Seite die Möglichkeit spezifisch nach Kategorien zu filtern. 

Wie bereits erwähnt, wurde zu Beginn von Twitch lediglich Gaming-Content gestreamt. Es ist aber ein deutlicher Trend zu erkennen, dass auch Nicht-Gaming-Content immer mehr an Beliebtheit gewinnt. So wurde 2016 die Kategorie \glqq in real life\grqq{} (IRL) eingeführt und diese im Jahr 2018 in viele neue Unterkategorien aufgeteilt. Eine dieser Unterkategorien ist \glqq Just Chatting\grqq{}. In dieser unterhalten sich die Streamenden mit den Zuschauerinnen und Zuschauern über alle möglichen Themen, schauen gemeinsam Videos auf YouTube oder bewerten auch andere Streamerinnen und Streamer um nur ein paar Beispiele zu nennen. \parencite{DAnastasio2020} Sie erreichte im Dezember 2019 das erste mal den Platz eins in der Liste der meistgeschauten Kategorien auf Twitch. Mit 80,9 Mio Stunden war sie 6 Mio Stunden vor \glqq League of Legends \grqq{} und 22 Mio Stunden vor \glqq Fortnite\grqq{} den bis dato meistgestreamten Kategorien auf Twitch \parencite{Yosilewitz2020}. Auch Kategorien wie \glqq Music \& Performing Arts\grqq{}, \glqq Art\grqq{} oder \glqq Food \& Drink\grqq{} sind aus IRL hervorgegangen. \parencite{Alexander2018}

Die Plattform ist für beide Seiten kostenlos, sowohl für die Konsumierenden als auch für die Streamenden. Twitch finanziert sich unter anderem durch Werbung und einem optionalen Premium-Abonnement namens \glqq Twitch Turbo\grqq{}. \parencite{Iqbal2020} Für das Konsumieren der Inhalte der Plattform wird keine Account benötigt. Möchte eine Person aber an einer Chat-Diskussion teilnehmen, einem Channel folgen oder selbst streamen muss diese sich einen Account erstellen und mit diesem angemeldet sein.

Das Streamen wird von den meisten Streamerinnen und Streamern zu Beginn nur als Hobby angesehen. Damit es auch finanziell lukrativ wird, hat Twitch die Möglichkeit eingebaut, dass das Publikum die Streamenden auch finanziell unterstützen kann. Hierfür müssen die Streamenden den sogenannten \glqq Affiliate\grqq{}- oder den \glqq Partner\grqq{}-Status erreichen. Diese Status ermöglichen den Streamenden die sogenannten \glqq Subscriptions\grqq{} und \glqq Bits\grqq{} zu aktivieren. 

Die Subscriptions sind eine Möglichkeit eine Streamerin oder einen Streamer auf monatlicher Basis finanziell zu unterstützen. Es gibt drei Stufen, die sogenannten \glqq Tiers\grqq{}. Die Preise der jeweiligen Tiers können aus \autoref{img:tierPreise} entnommen werden. Die Streamenden erhält hiervon als Affiliate jeweils 50\%, die andere Hälfte behält Twitch. Als Partner kann diese Teilung auch zugunsten des Content-Creators verändert werden, sodass manche Streamer sogar bis zu 100\% des Preises ausgezahlt bekommen. \parencite{Stephenson2020} Durch die Zugehörigkeit von Twitch zu Amazon haben Amazon Prime-Mitglieder Vorteile. Hat ein Amazon Prime-Mitglied seinen Account mit Twitch verknüpft, so hat es die Möglichkeit monatlich eine Tier-1-Subscription ohne Mehrkosten bei einer Streamerin oder einem Streamer seiner Wahl abzuschließen. Amazon nennt diese Aktion \glqq Prime-Gaming\grqq{} \parencite{Walter2020}. Nimmt ein Mitglied daran teil so erhält es ein sogenanntes \glqq Chat-Badge\grqq{}. Dies ist ein kleines Symbol, das im Chat links neben seinem Namen angezeigt wird. Im Fall von Prime-Gaming ist es eine blau-weiße Krone.

\begin{figure}[ht]
\caption{Twitch Subscibtions \parencite{Twitchd}}
\label{img:tierPreise}
\centering
\includegraphics[width=0.7\textwidth]{twitch_sub}
\end{figure}

Nach dem Kauf einer Subscription ist diese, zusammen mit ihren Vorteilen, für einen Monat aktiv. Durch das Subscriben kann der Stream ohne Werbeunterbrechungen geschaut werden. Zudem werden die sogenannten \glqq Channel-Emotes\grqq{} freigeschalten. Diese werden von den Streamenden selbst festgelegt um sie individuell an den Channel anzupassen. Jede Subscriberin und jeder Subscriber schaltet das Channel-Badge frei. Haben diese schon mehre Monate eine Subscription abgeschlossen, so kann sich das Badge je nach Channel und Dauer verändern. Hierdurch wird im Chat direkt sichtbar wer den Channel unterstützt. Diese Badges sind ebenfalls in jedem Channel individuell angepasst.

Des weiteren kann im Fall eines sehr aktiven und somit unübersichtlichen Chats der \glqq Follower-Only\grqq{} oder \glqq Subscriber-Only\grqq{} Modus aktiviert werden. In diesen Modi ist es nur folgenden bzw. subscribenden Mitgliedern möglich in den Chat zu schreiben. Somit ist ein weiterer Vorteil des Subscribens auch in diesem Modus in den Chat schreiben zu können.

Ein Twitch-Mitglied kann nicht nur sich selbst eine Subscription kaufen sondern auch anderen. Dieser Vorgang wird auch \glqq Giften\grqq{} genannt und macht aus dem Schenkenden einen \glqq Gifter\grqq{}. Beim Schenken besteht die Möglichkeit einer bestimmten Person eine Subscription oder eine bestimmte Anzahl an Subscriptions an die Community zu vergeben. Bei letzterem entscheidet Twitch zufällig wer diese Geschenke erhält.

Die bereits vorher erwähnten Bits sind eine digitale Währung der Plattform, die für Echtgeld erworben werden können. Sie können für verschiedene Dinge eingesetzt werden, unter anderem für das sogenannte \glqq Cheeren\grqq{}. Dies kann als \glqq Anfeuern\grqq{} ins Deutsche übersetzt werden und bedeutet, dass eine Nachricht in den Chat geschrieben und mit animierten Cheer-Emotes versehen wird. Den Mindestbetrag für ein Cheer können die Streamenden selbst bestimmen. Die Preise der Bits können der \autoref{tab:bitsPreise} entnommen werden. Ein Bit entspricht einem US-Cent, den der Streamer erhält. In der Tabelle wird deutlich, dass sich Twitch beim Kauf der Bits einen bestimmten Prozentsatz einbehält.

\begin{table}[!htp]\centering
\caption{Bits Preise Stand 18.10.2020 \parencite{Twitch2020a}}\label{tab:bitsPreise}
\scriptsize
\begin{tabular}{lrr}\toprule
\textbf{Anzahl der Bits} &\textbf{Preis in Eur} \\\midrule
100 &1,47 \\
500 &7,34 \\
1500 &20,91 \\
5000 &67,51 \\
10000 &132,08 \\
25000 &322,87 \\
\bottomrule
\end{tabular}
\end{table}

Ein weiteres Mittel, das ein Twitch-Mitglied dazu animieren soll eine Streamerin oder einen Streamer zu unterstützen, ist der sogenannte \glqq HypeTrain\grqq{}. Dieser ist ein Event, welches automatisch eintritt, sobald innerhalb von fünf Minuten eine bestimmte Menge an Bits und Anzahl an Subscriptionkäufe getätigt wurden. Die Menge, um das Event zu triggern, wird durch die Streamenden in den Einstellungen selbst festlegen. Dies hat den Vorteil, dass anhand der Zuschauerzahlen festgelegt werden kann, wann und wie oft der HypeTrain aktiviert werden soll. Passiert dies allerdings zu oft, verliert der HypeTrain seine \glqq Besonderheit\grqq{} und wird schnell als störend empfunden. Die Streamenden sollten ihn deshalb mit Bedacht einsetzen. \parencite{Twitche}

Bei Aktivierung des HypeTrains wird oberhalb des Chats ein Fortschrittsbalken mit Prozentzahl und Countdown angezeigt. Das Ziel des Publikums ist es nun diesen Balken mit Hilfe von Subscriptions und Cheers zu füllen. Der HypeTrain ist aufgeteilt in aufeinander folgende Level. Es können bis zu fünf Level aktiviert werden. Jedes Level muss innerhalb von jeweils fünf Minuten abgeschlossen werden, ansonsten bricht der HypeTrain ab, sobald der Countdown bei Null angekommen ist. Die Streamenden können den Schwierigkeitsgrad des HypeTrains selbst bestimmen. Das bedeutet in diesem Fall, wie viele Bits und Subscriptions benötigt werden damit ein Level abgeschlossen wird. Bei erfolgreichem Abschluss des HypeTrains erhalten die Teilnehmerinnen und Teilnehmer als Belohnung je nach Level ein unterschiedliches Emote.


\begin{figure}[ht]
\caption{Aktivierter HypeTrain in einem Channel\parencite{Twitche}}
\centering
\includegraphics[width=0.7\textwidth]{twitch_hypetrain}
\end{figure}

Ein weiteres Event ist der sogenannte \glqq Raid\grqq{}. Der Begriff ist in Online-Rollenspielen entstanden. Es wird von einem Raid gesprochen, wenn eine Gruppe von Spielerinnen und Spielern gemeinsam ein Level bestreiten, für das sie alleine nicht stark genug wären. 

Auf Twitch ist ein Raid, wenn die Streamenden am Ende eines Streams das verbliebene Publikum an einen andern Twitch-Channel übergeben. Die Zuschauerinnen und Zuschauer haben nun das Gefühl an einem gemeinsamen Event teilzunehmen. Raids werden von Streamerinnen und Streamern genutzt, um eine Community aufzubauen oder andere, in der Regel kleinere Streamer, zu unterstützen. \parencite{Stephen2020}

\section{Gamification}
\label{chap:gamif}
Gamification ist ein Begriff der im Verlauf der letzten Jahre immer mehr an Aufmerksamkeit gewonnen hat. Der Begriff existiert seit 2002 und wurde durch Nick Pelling das erste mal verwendet \parencite{Pelling2011}. In diesem Konzept geht es darum eine Aufgabe die für einen Benutzer eher uninteressant und langweilig ist, in eine interessante Aufgabe zu verändern. Da Spiele die Spielenden schon von Natur aus motivieren, wird mit Gamification versucht, die motivierenden Spiel-Elemente aus den Spielen zu extrahieren und die uninteressanten Aufgaben damit motivierend zu gestalten. \parencite{Marczewski2013}

Nachdem lange Verwirrung um eine genaue Definition von Gamification bestand, definierte es \cite{Deterding2011} wie folgt: \glqq Gamification is the use of game design elements in non-game contexts\grqq{}. Somit ist Gamification die Anwendung von Elementen aus Spielen in \glqq Nicht-Spiel-Umgebungen\grqq{}. Beispiele für Elemente sind unter anderem \acrfull{pbls}. \parencite{Deterding2011}

Es wird fälschlicherweise oft angenommen, dass das alleinige Hinzufügen von Punkten bereits Gamification ist. Dabei entsteht eine Verwechslung mit \glqq Pointsification\grqq{}. Dies ist das reine Anwenden von \acrshort{pbls} auf ein System ohne tieferes Konzept, in der Hoffnung, dass es eine positive Auswirkung auf das Verhalten der Zielgruppe hat. So werden bei Pointsification beispielsweise für das Durchführen von Aufgaben Punkte vergeben, die Benutzerinnen und Benutzer anhand ihres Punktestandes in einem Leaderboard verglichen. Nach dem Erreichen einer bestimmten Punktzahl wird ihnen ein Zertifikat vergeben. \parencite{Kifetew2017}

Gamification ist viel mehr. Es ist das gezielte Ansprechen von Motivationsfaktoren in der menschlichen Psyche. Deshalb wird bei Gamification ein besonderes Augenmerk auf den Menschen, dessen Verhalten, Bedürfnisse, Wünsche und Emotionen gelegt. Dieses Vorgehen wird \acrfull{hfd} genannt. Im Gegensatz dazu wird beim klassischen \glqq Function-Focused Design\grqq{} das Augenmerk auf die durchzuführende Arbeit und die hierfür möglichst effiziente funktionale Lösung gelegt. \parencite{Mechelen2017} In den nachfolgenden Unterkapiteln werden hilfreiche Werkzeuge vorgestellt, mit denen die Umsetzung eines \acrshort{hfd}-Prozesses erleichtert wird. Diese Werkzeuge sind das Konzept von Spielertypen, um die Zielgruppe besser zu verstehen, Motivationsfaktoren der menschlichen Psyche, um zu verstehen was Menschen motiviert und ein Design Framework, um passende Gamification-Elemente zu finden.

\subsection{Spielertypen}
\label{chap:playertypes}
Um das \acrshort{hfd} erfolgreich einsetzen zu können muss sich bei der Entwicklung der Gamification-Strategie das Entwicklungsteam der Zielgruppe bewusst sein. Zur Erleichterung kann in Gamification auf das Konzept von Spielertypen zurückgegriffen werden. Diese beschreiben die verschiedenen Verhaltensmuster von Menschen und ordnen sie bestimmten Kategorien bzw. Typen zu. \parencite{Chou2017,Bartle1996}

Hierbei können verschiedenen Modelle verwendet werden, die in verschiedenen Bereichen Anwendung finden. Beispielsweise gibt es die Spielertypen nach Bartels, die in vier verschiedene Typen aufgeteilt werden. \parencite{Bartle1996}  Jedoch ist dieser Ansatz für Rollenspiele entwickelt worden und deshalb hier nicht anwendbar. Nach eigener Aussage sind die Spielertypen nach Marczewski besser. Sie bestehen aus 6 verschiedenen Arten von Spielern. \parencite{Marczewski2015}

Die verschiedenen Spielertypen nach Marczewski können der \autoref{img:hexad} entnommen werden. Der äußere, grüne Bereich, stellt die Spielertypen dar, wobei der innere, rote Bereich, die dazugehörigen Motivationsfaktoren des Typen sind. So ist der \glqq Socializer\grqq{} durch die Verbundenheit zu anderen Menschen und der \glqq Free Spirit\grqq{} durch möglichst viel Autonomie und kreativer Freiheit motiviert. Allerdings kann man einen Menschen nicht nur einem Typen zuordnen. Vielmehr hat jeder Mensch auch jeden Typen mit unterschiedlicher Ausprägung in sich. Jeder Typ reagiert auf die verschiedenen \acrshort{pbls} unterschiedlich stark. \parencite{Marczewski2015}

\begin{figure}[ht]
\caption{Gamification User Types Hexad von Marczewski \parencite{Marczewski2016}}
\label{img:hexad}
\centering
\includegraphics[width=0.7\textwidth]{gami_usertypes_hexad}
\end{figure}

\subsection{Motivationsfaktoren}
\label{chap:motfak}

Um auch die Motivationsfaktoren besser zu verstehen wurden hierfür ebenfalls verschiedene Frameworks entwickelt. Eines davon ist das \glqq Octalysis Framework\grqq{} von YouKai Chou. Dieses beschreibt die Motivation des Menschen anhand acht verschiedener Teilbereiche. Sie werden in dem Framework auch \glqq Core Drives\grqq{} genannt. Die acht verschiedenen Core Drives können der \autoref{img:core_drives} entnommen werden.

\begin{figure}[ht]
\caption{Octalysis eight core drives \parencite[p.~23]{Chou2015}}
\label{img:core_drives}
\centering
\includegraphics[width=\textwidth]{gami_oct}
\end{figure}

Der erste Core Drive ist \glqq Epic Meaning \& Calling\grqq{} (dt. Höhere Bedeutung \& Berufung). Hierbei haben Menschen das Gefühl etwas zu machen, das größer ist als sie selbst bzw. dass sie zu etwas Größerem einen Beitrag leisten. Ein gutes Beispiel hierfür ist die Seite Wikipedia auf der viele Menschen gemeinsam, freiwillig und unentgeltlich an einem Projekt arbeiten, um das Wissen der Menschheit zu sammeln.\parencite[65\psqq]{Chou2015}

Der zweite Core Drive ist \glqq Development \& Acomplishment\grqq{} (dt. Entwicklung / Fortschritt \& Errungenschaften). Hierbei ist es wichtig Menschen vor eine Herausforderung zu stellen. Werden diese geschafft, so bekommt er das Gefühl, dass er stolz auf sich sein kann. Das menschliche Gehirn ist darauf ausgelegt belohnt zu werden. Diese Belohnung erhält er in visueller Form, indem ein Fortschritt angezeigt wird oder indirekt in dem er fühlt, dass seine eigene Erfahrung wächst. Dieser Core Drive ist der meist genutzte in der Gamification Industrie, da er sich auch am einfachsten implementieren lässt.\parencite[88\psqq]{Chou2015}

\glqq Empowerment of Creativity \& Feedback\grqq{} (dt. Förderung der Kreativität \& Feedback) ist der dritte Core Drive. Er spricht besondere Eigenschaften des Menschen an, wie zum Beispiel den Drang Neues zu lernen, seinen Horizont zu erweitern oder neue Dinge zu erfinden. Ein Beispiel hierfür ist das Spiel mit den bunten Bausteinen, Lego. \parencite[123\psqq]{Chou2015}

Das vierte Core Drive \glqq Ownership \& Possession\grqq{} (dt. Eigentum \& Besitz) funktioniert nach einem einfachen Prinzip. Hat ein Mensch von etwas Besitz ergriffen, möchte er davon nicht mehr loslassen, sondern vielmehr es verbessern und daraus das Beste herausholen. Hierbei spielen Elemente wie digitale Währungen oder Besitztümer eine große Rolle. Es besteht eine große Wahrscheinlichkeit, dass Menschen dort automatisch versuchen einen virtuellen Wohlstand aufzubauen. Auch die Möglichkeit der Individualisierung spielen hier eine große Rolle. \parencite[161\psqq]{Chou2015}

\glqq Social Influence \& Relatedness\grqq{} (dt. sozialer Einfluss \& Verbundenheit) als fünftes Core Drive taucht oft im Kontext von Wettbewerben, Neid, Gruppenaufgaben, Allgemeingut und Kameradschaft auf. Dabei wird der Drang des Menschen angesprochen sich mit anderen auszutauschen oder zu messen. Auch bei Dingen zu denen sich ein Mensch emotional verbunden fühlt oder diese mit Kindheitserinnerungen in Verbindung bringt, ist das die treibende Kraft dahinter. \parencite[194\psqq]{Chou2015}

\glqq Scarcity \& Impatience\grqq{} (dt. Knappheit \& Ungeduld) ist der sechste Core Drive. Dieser baut auf dem natürlichen Drang des Menschen auf, Dinge zu wollen, die unerreichbar scheinen. Auch das Gefühl von Exklusivität spielt hierbei eine große Rolle. \parencite[231\psqq]{Chou2015}

Der CoreDrive sieben, \glqq Unpredictability \& Curiosity\grqq{} (dt. Unvorhersehbarkeit \& Neugierde), ist oft bei Glücksspielen anzutreffen. So ist ein Mensch oft stärker motiviert, wenn es anstelle einer Gewinngarantie nur eine Gewinnwahrscheinlichkeit gibt. Mit dem sicheren Wissen einer Belohnung, ist die Motivation nur so groß wie die Belohnung selbst. Besteht aber nur eine bestimmte Chance, dass er diese Belohung erhält, so ist er durch die Spannung stärker motiviert als durch die Belohnung an sich. \parencite[271\psqq]{Chou2015}

Der achte und letzte Core Drive, \glqq  Loss \& Avoidance\grqq{} (dt. Verlust \& Vermeidung), ist durch das Gefühl angeregt etwas Erarbeitetes zu verlieren. Wobei auch das \glqq Sunk Cost\grqq{}-Prinzip eine große Rolle spielt. Dieses Prinzip tritt ein, wenn ein Mensch bereits Bemühungen in die Erreichung eines Ziels investiert hat. Selbst wenn für den Abschluss dieses Ziels noch unrentabel viel Aufwand  zu erbringen ist, werden die bereits investierten Bemühungen als Motivation genommen, trotzdem weiter zu machen. Der User möchte somit das Gefühl vermeiden, die bereits investierten Dinge zu verlieren. \parencite[307\psqq]{Chou2015}

Die acht verschiedenen Core Drives werden in \glqq Left\grqq{}- oder \glqq Right Brain\grqq{} und \glqq White Hat\grqq{} oder \glqq Black Hat\grqq{}-Gamification aufgeteilt. Wie in \autoref{img:core_drives_dir} zu sehen, sind hiermit die drei Core Drives auf der linken und die drei auf der rechten Seite bzw. die drei unteren Core Drives und die drei Oberen gemeint. Die in der Mitte liegenden Core Drives werden für die jeweilige Unterteilung als neutral betrachtet. Die Unterteilung in \glqq Left Brain\grqq{} und \glqq Right Brain\grqq{} kann auch als extrinsisch und intrinsisch bezeichnet werden. \parencite[341\psqq]{Chou2015}

\glqq White Hat\grqq{} Core Drives geben einem Menschen das Gefühl von Stärke, Erfüllung und Befriedigung. Sie geben das Gefühl alles unter Kontrolle zu haben, wohingegen \glqq Black Hat\grqq{} Core Drives genau das Gegenteil bewirken. Diese verbreiten das Gefühl von Angst, Besessenheit und Sucht, wodurch ein User das Gefühl bekommt, die Kontrolle über seine Aktionen verloren zu haben. \parencite[370\psqq]{Chou2015}



\begin{figure}[ht]
\caption{Octalysis Tendenzen \parencite[pp.~29-32]{Chou2015}}
\label{img:core_drives_dir}
\centering
\includegraphics[width=\textwidth]{gamif_oct_left_right_black_white}
\end{figure}

Das Ziel dieser Motivationsfaktoren ist es, eine Aussage darüber zu treffen, wie ausgeglichen die Implementierung von Gamification in einem System ist. Ebenfalls kann eine Aussage darüber getroffen werden, welche Core Drive nachgebessert werden müssen, da sie noch nicht ausgeprägt genug sind. Es ist nicht zwingen notwendig, dass alle Core Drives gleichermaßen ausgeprägt sind. Viel wichtiger ist es, dass das Verhältnis zwischen \glqq Left\grqq{} und \glqq Right Brain\grqq{} als auch von \glqq White Hat\grqq{} und \glqq Black Hat\grqq{}-Gamification ausgeglichen ist.\parencite[342\psqq]{Chou2015}

\subsection{Design Framework }
\label{chap:des_fram}
Mit der Kenntnis der Motivationsfaktoren wird nun deren Anwendung näher erläutert. Benötigt wird hierfür ein \acrfull{gdf}. Es gibt beispielsweise das \acrfull{gmc}, das in acht Schritte unterteilt ist. Es baut auf dem \glqq MDA game design framework\grqq{} und \glqq Business Model Canvas\grqq{} auf. \parencite{Baldeon2016}

In dieser Arbeit wird jedoch auf das \acrfull{osd} gesetzt, da dieses ebenfalls von YouKai Chou entwickelt wurde, um auf den acht Core Drives aufzubauen.  Das Ziel des \acrshort{osd} ist es herauszufinden, welche Gamification-Elemente eingesetzt werden müssen, um einen Menschen zu einem festgelegtem Verhalten zu motivieren. Dabei fokusiert es sich auf die wichtigsten Aspekte und unterteilt diese in einzelne Schritte. Diese Schritte bauen aufeinander auf und sind untereinander abhängig. Das \acrshort{osd}  wird in fünf Schritte unterteilt. Diese fünf Schritte können \autoref{img:gami_oct_strat} entnommen werden. \parencite[467\psqq]{Chou2015}

Zu Beginn werden die genauen Ziele anhand von Business Metriken definiert. Diese Metriken sollten SMART sein. Sie sollen sich im Verlauf der Anwendung ändern und werden im Nachgang mit ihren Ursprungswerten verglichen um die Effektivität der Kampagne zu evaluieren. Bei den Metriken ist ebenfalls darauf zu achten, dass diese nach Priorität sortiert sind um zuerst die wichtigsten Metriken  zu verbessern.

Im nächsten Schritt wird anhand der bereits in \autoref{chap:playertypes} erwähnten Spielertypen die Zielgruppe definiert. Hierbei ist darauf zu achten, dass diese durch ihre unterschiedlichen Motivationsfaktoren kategorisiert werden. Anderenfalls besteht die Gefahr, dass Gruppen definiert werden, welche unterschiedlich erscheinen, jedoch auf die gleiche Weise motiviert werden können. Um die Zielgruppe richtig zu definieren kann auch auf das Konzept der Personas zurückgegriffen werden \parencite{Nielsen2013}. 

Im dritten Schritt werden die \glqq Desired Actions\grqq{} und daraus resultierenden \glqq Win-States\grqq{} definiert. Die \glqq Desired Actions\grqq{} sind kleine Schritte die von der Zielgruppe durchgeführt werden sollen, um die Business Metriken zu verbessern. Solche Schritte können beispielsweise das Besuchen einer Webseite, das Ausfüllen eines Formulars und das anschließende Registrieren, das Anmelden zu einem Newsletter oder das Klicken auf eine Werbeanzeige sein. Der \glqq Win-State\grqq{} beschreibt einen Zustand, in dem die Ziele der Zielgruppe, als auch die Ziele der Business Metriken erreicht werden.

Der vierte Schritt beschreibt die \glqq Feedback Mechanics\grqq{}. Diese helfen der Zielgruppe auf dem Weg zum \glqq Win-State\grqq{} seinen aktuellen Fortschritt im Blick zu behalten. Sie sorgen dafür, dass er die \glqq Desired Actions\grqq{} auch durchführt, indem sie ihn motivieren. Dieser Schritt ist der erste, bei dem konkret auf Gamification-Elemente eingegangen wird. 

Im fünften und letzten Schritt werden die \glqq Incentives\grqq{} (dt. Belohnungen) definiert, welche im \glqq Win-State\grqq{} enthalten sind. Diese Belohnungen werden einer Anwenderin oder einem Anwender nach der Erfüllung der \glqq Desired Actions\grqq{} direkt oder indirekt gegeben, um zu zeigen, dass eine richtige Aktion durchgeführt wurde.


\begin{figure}[ht]
\caption{Octalysis Strategy Dashboard \parencite[p.~467]{Chou2015}}
\label{img:gami_oct_strat}
\centering
\includegraphics[width=0.8\textwidth]{gami_oct_strat_dash}
\end{figure}




%
%
%
%
%
%
%
%
%
%
% 
\chapter{Gamification auf Twitch}

Twitch ist eine Plattform, die ursprünglich speziell für das Livestreamen von Videospielen entwickelt wurde. Somit ist es nicht verwunderlich, dass hier auch aktiv Gamification Anwendung findet. Im folgenden Kapitel werden die Gamification-Elemente in drei Kategorien aufgeteilt. Bei der ersten sind Gamification-Elemente nativ in der Twitch Plattform integriert. In der zweiten Kategorie werden Overlays durch die Streamenden direkt in das Videobild des Streams eingebettet. Die dritte Kategorie verwendet Chatbots, die sich direkt mit dem Twitch-Chat eines Streams verbinden.

Am Ende werden die einzelnen Gamification-Elemente anhand der acht Core Drives aus \autoref{chap:motfak} bewertet. Es wird auch überprüft, wie diese Elemente durch die Twitch API oder andere Schnittstellen beeinflusst werden können. Nur dann, wenn die Gamification-Elemente eine selbst definierte \glqq Desired Action\grqq auslösen können, können diese effektiv für Marketing verwendet werden.

\section{Nativ in Twitch}
\label{chap:nat_twitch}
Da Twitch ein finanzielles Interesse daran hat, Streamende und deren Publikum an sich zu binden, hat die Plattform viele Gamification-Elemente direkt in die Plattform integriert. Sowohl auf der Seite des Publikums als auch auf der von Streamerinnen und Streamern werden Gamification-Elemente eingesetzt. Zunächst wird die Seite der Streamenden beleuchtet. Diese werden motiviert weiter zu streamen, um neuen Content für Twitch zu produzieren. Danach wird auf die Seite der Zuschauerinnen und Zuschauern eingegangen. Diese sollen motiviert werden, den Stream weiter zu konsumieren und  Channels finanziell durch Cheeren und Subscriben zu unterstützen.

Twitch ist eine sich sehr schnell entwickelnde Plattform. So testet die Plattform oft neue Elemente, die nach wenigen Monaten wieder entfernt werden, da sie nicht den gewünschten Langzeiteffekt erzielt haben. Ein Beispiel hierfür sind die \glqq Twitch Crates\grqq{} die durch Spielkäufe direkt auf der Plattform erhalten werden konnten. Da die Möglichkeit des Spielkaufens auf Twitch jedoch entfernt wurde, entfielen auch die Crates. In diesen Crates waren beispielsweise Badges oder Emotes enthalten, die heute trotzdem noch auf der Plattform zu finden sind. Durch fehlende Aktualität und Relevanz werden sie in dieser Arbeit nicht behandelt. \parencite{Santana2018}

\subsection{Achievements \& Meilensteine für Streamer}
\label{chap:gami_achievements}
Der Weg auf Twitch ein erfolgreicher Streamer zu werden ist lang und mit viel Arbeit verbunden. Damit die Content-Creator auf dem Weg dorthin durchgehen motiviert bleiben, hat Twitch ein sogenanntes \glqq Achievement System\grqq{} in die Plattform eingebaut. Dieses System besteht aus vielen kleinen Aufgaben, die es zu erreichen gilt, beispielsweise eine bestimmte Anzahl an Followern zu erreichen, eine Anzahl an Tagen zu streamen und vieles mehr.

Die Achievements und deren aktueller Fortschritt können im Creator-Dashboard in den Twitch Einstellungen angezeigt werden. Dort ist eine Liste mit allen Achievements mit eingebettetem Fortschrittsbalken an der Unterseite jedes Achievements und die aktuellen Zahlen auf der rechten Seite. Zudem gibt es auch Meilensteine die erreicht werden müssen, um die in \autoref{chap:grund_twitch} beschriebenen Status \glqq Affiliate\grqq{} und \glqq Partner\grqq{} zu erreichen. Wie in \autoref{img:milestones} zu sehen, gibt es für beide Status den \glqq Path to Affiliate\grqq{} und \glqq Path to Partner\grqq{} (Core Drive 2 und 4). Da diese Gamification-Elemente aber nur für die Streamenden selbst sichtbar sind können sie in einer Marketing-Kampagne nicht angewandt werden. Somit werden sie für diese Arbeit als nicht relevant definiert.

\begin{figure}[ht]
\caption{Twitch Achievments \& Meilensteine für Streamer \parencite{Twitchb}}
\label{img:milestones}
\centering
\includegraphics[width=\textwidth]{twitch_gamif_achievements}
\end{figure}

Wie in \autoref{chap:grund_twitch} geschrieben sind einige Twitch Mechaniken auf dem Profil eines Streamers erst mit der Erreichung des \glqq Affiliate\grqq{} oder \glqq Partner\grqq{}-Status freigeschaltet. Viele dieser Mechaniken sind Gamification-Elemente. Deshalb wird im weiteren in dieser Arbeit davon ausgegangen, dass mindestens der \glqq Affiliate\grqq{}-Status erreicht wurde. Mit dem Abschließen weiterer Achievements kann die Quantität von Gamification-Elemente erhöht werden. So kann beispielsweise durch Erhöhen der Anzahl durchschnittlicher Chatter die Menge an verfügbaren VIP Badges erhöht werden \parencite{Twitchg}. 

\subsection{Leaderboard}
\label{chap:leaderboard}
Oberhalb des Chats befindet sich eine Liste mit den \glqq Top 10 Sub Giftern\grqq{} sowie den \glqq Top 10 Cheerern\grqq{}. Diese Liste wird auch Leaderboard genannt. Die Streamenden können selbst festlegen, ob die Top 10 des Tages, der Woche, des Monats oder die Top 10 insgesamt angezeigt werden sollen. Durch die zeitliche Begrenzung, sowie die Limitierung auf die Anzahl zehn, sind die Plätze sehr begehrt (Core Drive 6).

Das Leaderboard ist ein einfaches Mittel von Twitch das Publikum dazu zu bewegen, die Streamenden weiter zu unterstützen. Durch das Verschenken von Subscriptions und das Cheeren kann er sich in der Rangliste nach oben kämpfen (Core Drive 2 und 5). Zudem muss auch weiterhin Geld investiert werden, um nicht von anderen überholt zu werden (Core Drive 8).

\begin{figure}[ht]
\caption{Zusammengesetzte Screenshots von Leaderboards von \parencite{LeosMind}}
\centering
\includegraphics[width=0.5\textwidth]{Twitch_Leaderboards}
\end{figure}

\subsection{Badges}
Wie bereits erwähnt sind Badges kleine Symbole, die einen Status oder eine Errungenschaft widerspiegeln (Core Drive 2). Sie werden im Chat links neben dem Usernamen angezeigt und sind für jede Person aus dem Publikum gleichermaßen sichtbar (Core Drive 5). Die beiden Core Drive 2 und 5 finden sich auf Twitch in allen Badges wieder und werden deshalb bei den einzelnen Badge-Typen nicht mehr extra erwähnt.

Zur leichteren Orientierung werden die Badges in dieser Arbeit, wie in \autoref{img:badges_overview} zu sehen, in zwei Gruppen unterteilt. Diese Gruppen sind \glqq Channel-Badges\grqq{} und \glqq Global-Badges\grqq{}. Erstere sind nur in dem Channel aktiv, in dem ein Mitglied diese auch freigeschaltet oder zugewiesen bekommen hat. Ein Mitglied kann diese in beliebig vielen Channels erhalten. Letztere sind auf der ganzen Plattform verfügbare Badges und werden in jedem Channel gleichermaßen angezeigt.

\begin{figure}[ht]
\caption{Twitch Badges Overview (eigene Darstellung)}
\label{img:badges_overview}
\centering
\includegraphics[width=\textwidth]{twitch_badges_overview}
\end{figure}

\subsubsection{User-Type-Badges}
Die User-Type-Badges gibt es sowohl unter den Channel-Badges als auch unter den Global-Badges. Wie ihr Name schon sagt, sind diese an den User-Typen oder an die Rolle eines Mitglieds gebunden. Diese Badges können der \autoref{img:role_badges} entnommen werden.

\begin{figure}[ht]
\caption{User Type Badges, Screenshots zusammengesetzt von \parencite{Twitch2020b}}
\label{img:role_badges}
\centering
\includegraphics[width=\textwidth]{x_twitch_badges}
\end{figure}

Die Rollen können autoritärer Natur sein wie bei \glqq Broadcaster\grqq{}, \glqq Moderator\grqq{}, \glqq Staff\grqq{} und \glqq Admin\grqq{}. So haben Streamenden in ihren eigenen Channels das Broadcaster-Badge. Da sich ein Stream leichter verwalten lässt wenn mehrere User mithelfen gibt es die Moderator-Rolle. Diese Rolle kann von Streamenden an vertrauenswürdige Twitch-Mitglieder vergeben werden. Fällt eine Person im Chat negativ auf, so kann diese von einer Moderatorin oder einem Moderator auf verschiedene Weisen bestraft werden. Ein Moderator erhält das \glqq Chat Moderator\grqq{}-Badge. Die Mitarbeiter von Twitch haben das \glqq Twitch Staff\grqq{}- oder \glqq Admins\grqq{}-Badge. 

Andere Rollen zeigen einen gewissen Status an, wie \glqq VIP\grqq{}, \glqq Verified\grqq{}, \glqq Turbo User\grqq{} und \glqq Prime Gaming User\grqq{}. Die Rolle VIP wird von Streamenden in der Regel an besonders treue und aktive Mitglieder vergeben, die in Livestreams positiv auffallen. Diese Badges sind besonders exklusiv (Core Drive 6). Das Verified-Badged bekommen Streamer, die den Partner-Status erreicht haben. Dieses Badge wird zusätzlich zum Chat auch auf der Profilseite der Streamenden angezeigt. \glqq Turbo User\grqq{} ist ein Premium-Status, den sich Mitglieder durch ein Monatsabo käuflich erwerben können. In diesem Abo sind zusätzliche Vorteile wie zum Beispiel Werbefreiheit auf der ganzen Plattform enthalten. Den \glqq Prime Gaming Status\grqq{} erhalten wie bereits in \autoref{chap:grund_twitch} erwähnt, User, die ihren Amazon Prime Account mit Twitch verknüpft haben.

\subsubsection{Supporter-Badges}
Die \glqq Supporter\grqq{}-Badges sind für Twitch-Mitglieder, die eine Streamerin oder einen Streamer und deren Community mit Bits und verschenkten Subscriptions unterstützen (Core Drive 1). Sie spiegeln die Anzahl an gespendeten Bits und verschenkten Subscriptions wider. Genannt werden sie \glqq Cheer Chat Badges\grqq{} und \glqq Sub Giter Badges\grqq{}. Je nach Anzahl der insgesamt gespendeten Bits oder geschenkten Subs in einem Channel verändern sie sich. Somit ist auch ein Ziel diese Badges im Level steigen zu lassen (Core Drive 4).

Es gibt 18 verschiedene \glqq Cheer Chat Badges\grqq{}. Wie in \autoref{img:supp_badges} zu sehen, erhalten Zuschauerinnen und Zuschauer, die mindestens ein Bit gespendet haben, ein graues Badge. Bei mindestens 100 gespendeten Bits verändert dieses sich zu einem lila Badge. Es können bis zu eine Millionen Bits gespendet werden, um damit das letzte Badge Level zu aktivieren. Es gibt 9 verschiedene \glqq Sub Gifter Badges\grqq{}. Diese erhält ein Mitglied sobald es mindestens eine Subscription verschenkt hat. Um die höchste Stufe zu erreichen müssen mindestens 1.000 Subscriptions gespendet werden.


\begin{figure}[ht]
\caption{Supporter Badges \parencite{Twitch2020b}}
\label{img:supp_badges}
\centering
\includegraphics[width=0.6\textwidth]{twitch_supporter_badges}
\end{figure}

\subsubsection{Subscriber-Badges}
Die sogenannten \glqq Subscriber\grqq{}-Badges bekommt ein Mitglied, das eine Subscription einer Streamerin oder eines Streamers kauft. Hierdurch signalisiert es seine Loyalität gegenüber diesem. Aus diesem Grund werden diese Badges auch oft \glqq Loyalty\grqq{}-Badges genannt werden. Durch sie wird ebenfalls ein starkes Gemeinschafts- bzw. Community-Gefühl bei allen Badge-Besitzerinnen und Besitzern erzeugt. Jedoch muss die Subscription jeden Monat erneuert werden, um das Badge nicht zu verlieren (Core Drive 8). 

Da jeder Channel einen eigenen Stil hat, kann auch das Aussehen der Badges durch die Streamenden selbst definiert werden. So sind die Badges, von Channels auf denen eher Militärsimulationen gespielt werden, im Stil von militärischen Abzeichen. Auf Channels von Musizierenden sind eher  musikrelevante Badges wie Schallplatten oder Noten anzutreffen. Manche Streamerinnen und Streamer entwickeln ihre Badges auch aus einem Insiderwissen der Community. Sind die Streamenden zum Beispiel große Fans von Pizza und die Community hat Kenntnis darüber, können die Badges die Form einer Pizza haben, wie in \autoref{img:sub_badges_eskei} zu sehen.

\begin{figure}[ht]
\caption{Screenshot von Subscriber-Badges von Eskei83 \parencite{Twitch2020b}}
\label{img:sub_badges_eskei}
\centering
\includegraphics[width=\textwidth]{twitch_eskei_badges}
\end{figure}

Wurde mehrere Monate in Folge eine Subscription abgeschlossen verändern diese Badges ihr Aussehen wie in \autoref{img:sub_badges_eskei} zu sehen (Core Drive 4). Das soll die Treue eines Unterstützers signalisieren. Wie bereits in \autoref{chap:grund_twitch} erwähnt, gibt es auch Subscriptions in verschiedenen Preiskategorien. Wurde eine Tier 2 oder Tier 3 Subscription gekauft, so wird dem Badge noch ein sogenannter \glqq Flair\grqq{} verliehen. Ein Flair ist ein zusätzliches Symbol oder eine Grafik, die auf das Badge gelegt wird. Tier 2 und Tier 3 haben unterschiedliche Flairs. Ein besonderes Badge bekommen die ersten Twitch-Mitglieder die einen Channel subscriben, das sogenannte \glqq Founders\grqq{}-Badge.

\subsubsection{Leaderboard-Badges}

\begin{figure}[ht]
\caption{Loaderboard Badges \parencite{Twitch2020b}}
\centering
\includegraphics[width=0.6\textwidth]{twitch_leaderboard_badges}
\end{figure}

Die \glqq Leaderboard\grqq{}-Badges sind, wie der Name schon sagt, eng mit dem Leaderboard aus \autoref{chap:leaderboard} verknüpft. Sie beziehen sich auf den aktuellen Rang im Leaderboard (Core Drive 4). Sie werden nur an die Plätze eins bis drei des Leaderboards vergeben (Core Drive 6). Deshalb müssen die Besitzerinnen und Besitzer eines solchen Badges auch weiterhin die Streamerin oder den Streamer unterstützen, um nicht überholt zu werden und dadurch das Badge wieder zu verlieren (Core Drive 8).

\subsubsection{HypeTrain-Conductor}

\begin{figure}[ht]
\caption{Hypetrain Conductor Badges \parencite{Twitche}}
\centering
\includegraphics[width=0.6\textwidth]{twitch_hypetrain_conductor}
\end{figure}

Ist ein HypeTrain aus \autoref{chap:grund_twitch} abgeschlossen, so erhält nur das Mitglied mit den meisten Spenden den goldenen \glqq HypeTrain Conductor\grqq{}-Badge als Belohnung (Core Drive 1 und 6). Im nächsten HypeTrain kann das goldene Badge aber wieder verloren werden, wenn ein anderes Mitglied diese Rolle übernimmt (Core Drive 8). Hierbei erhält die ehemalige Besitzerin oder der ehemalige Besitzer ein neues, permanentes lila \glqq Former Conductor\grqq{}-Badge, das anzeigt, dass er einmal der \glqq HypeTrain Conductor\grqq{} war. Wird er erneut Conductor in einem späteren HypeTrain so wird dieses wieder durch das Goldene ersetzt und das Spiel beginnt von vorne (Core Drive 4).

\subsubsection{Sonderaktionen Badges}

Hin und wieder vergibt Twitch auch weitere Badges für besondere Aktionen oder Anlässe. Die Badges sind nur limitiert in diesem Zeitraum verfügbar (Core Drive 6). Ein Beispiel hierfür ist die TwitchCon, eine Messe, die von Twitch einmal im Jahr abgehalten wird. Jede Besucherin und jeder Besucher der Messe erhält ein Chat-Badge, das ihn als Teilnehmender der Messe auszeichnet. Auch Charity-Events, wie beispielsweise das \glqq Holiday Season of Giving\grqq{} im Jahr 2018, zählen hier dazu. Pro 100 Bits die im Zeitraum vom 12. - 27. Dezember 2018 gespendent wurden, hat Twitch 20 cent an wohltätige Zwecke gespendet (Core Drive 1). Jede Person die in diesem Zeitraum mindestens 100 Bits mit dem Hashtag \#Charity gespendet hat, wurde mit einem Schneflocken-Badge ausgezeichnet. \parencite{Twitch2018a}

\subsection{Punkte}
\label{chap:tw_gami_punkte}
Twitch hat ein Punktesystem in die Streams implementiert, um die Zuschauerschaft länger in einem Stream zu halten. Es können Punkte gesammelt werden unter anderem durch Zusehen des Streams oder durch Teilnahme an einem Raid (Core Drive 4). So erhalten Zuschauerinnen und Zuschauer wie in \autoref{img:chan_pts} zu sehen, alle fünf Minuten zehn Punkte und durch die Teilnahme an einem Raid 250 Punkte. Ebenfalls erhalten die in unregelmäßigen Abständen kleine Boni mit 50 Punkten (Core Drive 7). Subscriberinnen und Subscriber erhalten die Punkte jeweils multipliziert mit einem definierten Faktor.

\begin{figure}[ht]
\caption{Channelpoints verdienen}
\label{img:chan_pts}
\centering
\includegraphics[width=0.4\textwidth]{twitch_channelpoints_earning}
\end{figure}

Die gesammelten Punkte können eingesetzt werden um bestimmte Aktionen im Stream auszulösen. Somit kann mit den Punkten der Verlauf des Streams beeinflusst werden (Core Drive 3 und 5). Hierfür gibt es einige vorgefertigte Aktionen von Twitch wie zum Beispiel das Hervorheben einer Nachricht im Chat oder das Freischalten eines Subscriber-Emotes. Die Streamenden selbst können aber auch noch andere Aktionen einbauen, beispielsweise, dass sie einen Schluck Wasser trinken oder das aktuelle Spiel nur mit einer Hand spielen müssen. Die Streamerinnen und Streamer können die Anzahl der für die Aktionen nötigen Punkte selbst festlegen. Oft werden für ein paar wenige Aktionen auch eine sehr hohe Anzahl an nötigen Punkten festgelegt, damit die Zuschauerinnen und Zuschauer auf diese sparen müssen (Core Drive 2).

Spezielle Software ermöglichen zudem einen tiefen Eingriff in das Streaming-Programm des Streamers, um unter anderem Szenen zu wechseln oder andere Kameraperspektiven zu aktivieren. Besonders engagierte Streamerinnen und Streamer bieten auch die Möglichkeit Sounds im Stream abzuspielen oder physikalische Elemente im Raum der Streamenden zu beeinflussen, wie zum Beispiel Lichter im Hintergrund.

\subsection{Bits und Cheeren}

Mit den Bits hat Twitch eine eigene \glqq In-Game-Währung\grqq{} etabliert und somit ein eigenes Wirtschaftssystem in ihre Plattform eingebaut. Aus echtem Geld wie Euro oder Dollar werden Bits. So kann der User sein Budget an Bits durch das Kaufen dieser erhöhen (Core Drive 4).

Die Bits sind nach dem Transfer physisch nicht mehr greifbar und nur noch als Zahl einer Fremdwährung erkennbar. Somit ist die Transparenz eingeschränkt. Punkte werden leichter ausgegeben als Echtgeld, da oft vergessen wird, wie viel Geld umgerechnet ausgegeben wird. \parencite{Madigan2019}

Wie bereits in \autoref{chap:grund_twitch} beschrieben, ist Cheeren eine Spende auf Twitch, die durch eine besondere Art zelebriert wird. Vergleichbar mit Crowdfunding werden Streamerinnen und Streamer von den Twitch-Mitgliedern durch das Cheeren unterstützt. Das ist das höhere Ziel aller Beteiligten (Core Drive 1 und 5).

Die Spenderinnen und Spenden haben die Möglichkeit sich aus einer Auswahl verschiedener animierter Emotes die für den Anlass passenden auszusuchen, diese individuell zu kombinieren und im Chat einzusetzen (Core Drive 3).

\subsection{HypeTrain}
Der HypeTrain, ist wie bereits in \autoref{chap:grund_twitch} erwähnt, ein Fortschrittsbalken, welcher durch Cheeren und Sub Giften gefüllt werden kann. Das Publikum arbeiten gemeinsam daran den HypeTrain zu gewinnen und abzuschließen (Core Drive 1, 2 und 5). Da aber Zuschauerinnen und Zuschauer nicht wissen, ob und wann andere den HypeTrain mit Spenden füttern, entsteht eine Angst diesen nicht bezwingen zu können. (Core Drive 8)

Dieser Fortschrittsbalken wird angezeigt sobald eine bestimmte Anzahl an Subscriptions, Bits und Gifted-Subs im Stream eingegangen sind. Es ist oft ein unerwartetes und besonderes Event (Core Drive 6 und 7). Durch einen Countdown wird zusätzlich eine künstliche Knappheit erzeugt. (Core Drive 7).

\subsection{Emotes}

Emotes sind kleine Symbole die im Chat eingesetzt werden können. Sie dienen als eine Art Item die gesammelt werden können um das Repertoire zu vergrößern (Core Drive 4). Die Zuschauerinnen und Zuschauer können aus einer Auswahl an vielen verschiedenen Emotes wählen, um ihre Emotionen im Chat kreativ mit anderen zu teilen (Core Drive 3 und 5). Für diese Arbeit werden sie in die drei Kategorien \glqq Global Emotes\grqq{}, \glqq freischaltbare Emotes\grqq{}  und \glqq Channel Emotes\grqq{} unterteilt. Die Global Emotes sind für alle Twitch Mitglieder gleichermaßen verfügbar. 

Bestimmte Emotes können auch durch verschiedene Aktionen freigeschaltet werden. Beispielsweise durch die Teilnahme am HypeTrain kann bei abgeschlossenem Level ein zufällig ausgewähltes Emote erworben werden. Hin und wieder gibt es auch Kooperationen von Spieleherstellern oder anderen Firmen zusammen mit Twitch. Hierbei wird individuell für einen besonderen Stream, ein Event oder ein Spiel eine Kollektion an Emotes zur Verfügung gestellt. Diese können durch das Schauen eines speziellen Streams, Subscriben oder Cheeren in einem limitierten Zeitraum freigeschalten werden (Core Drive 6).

Channel Emotes sind ähnlich wie bei den Subscriber Badges individuell für jeden Channel und werden durch das Subscriben eines Channels freigeschaltet. Diese können aber auch global in allen anderen Channels verwendet werden und können hierdurch sogar als Werbemittel dienen. Durch sie wird ein starkes Gemeinschaftsgefühl erzeugt (Core Drive 5). Dies, und die Exklusivität ihrer Emotes, nutzen Streamende oft um ihr Publikum zum Kauf einer Subscription anzuregen. Beispielsweise bitten sie in manchen Fällen den Chat ein bestimmtes Emote, das in den Subscriber Emotes enthalten ist, in den Chat zu schreiben. Um nicht ausgeschlossen zu werden, sind die Zuschauerinnen und Zuschauer motiviert eine Subscription zu kaufen (Core Drive 8).

\subsection{Drops}
Drops sind ein Lootbox-System um das Publikum für längere Zeit in einem Stream zu halten. Zuschauerinnen und Zuschauer können durch das Konsumieren eines Streams, in dem diese Funktion aktiviert ist, exklusive Items für das gestreamte Spiel freischalten (Core Drive 2 und 6). Diese Items können Abzeichen, Waffentarnungen oder andere kosmetische Dinge sein. Um sie in das Spiel zu übertragen, muss der jeweilige Spiel-Account mit dem Twitch-Account verknüpft werden.

Es gibt zwei Arten von Drops, die zeitbasierten und eventbasierten. Zeitbasiert bedeutet, das die Zuschauerinnen und Zuschauer für einen vorher definierten Zeitraum dem Stream folgen muss, um den Drop freizuschalten. Bei dem Event basierten Drops müssen sie so lange im Stream bleiben bis die Streamerin oder der Streamer ein bestimmtes Level geschafft oder eine besondere Aktion durchgeführt hat (Core Drive 7).

\subsection{Alerts im Chat}
\label{chap:bllshit}
Wurde eine Subscription gekauft, so wird dies im Chat besonders hervorgehoben, wodurch sich die Käuferinnen und Käufer mehr wertgeschätzt fühlen (Core Drive 2 und 5). Hat ein Mitglied zum erste mal eine Subscription gekauft, wird diese Nachricht automatisch angezeigt, um dem Mitglied diese Funktion näher zu bringen (Core Drive 7). Im Fall einer Erneuerung der Subscription kann selbst bestimmt werden, wann die Mitteilung angezeigt werden soll. Dabei hat es nun auch die Möglichkeit, eine Nachricht zu schreiben, die innerhalb dieses hervorgehobenen Bereiches angezeigt wird (Core Drive 3).

\begin{figure}[ht]
\caption{Alerts im Chat \parencite{Twitch2020d}}
\label{img:achievement_zuschauer}
\centering
\includegraphics[width=0.7\textwidth]{twitch_sub_achievement}
\end{figure}

\section{Stream Overlays}
\label{chap:gami_overlays}
Overlays sind seit Anbeginn von Twitch ein Kernbestandteil von Streams. Um Overlays erklären zu können muss zunächst erklärt werden, wie ein Stream produziert wird. Twitch ist eine sogenannte \glqq Crowdsourced Content\grqq{}-Plattform. Hierbei geht es darum, dass nicht Twitch selbst den Content für ihre Internetseite produziert sondern jeder die Möglichkeit hat einen Stream zu starten. Zum Streamen benötigt man einen Computer auf dem eine spezielle Streaming-Software installiert ist. Die bekannteste ist \acrfull{obs}. \parencite{Zhang2015}

Auch die modernen Videospielkonsolen Playstation 4 und Xbox One bieten mit integrierter Software die Möglichkeit zu Streamen. Diese werden jedoch in dieser Arbeit als nicht relevant definiert.

\acrshort{obs} bietet die Möglichkeit in Echtzeit, Video- und Audiomaterial aus verschiedenen Quellen zu verschmelzen. Diese Quellen können zum Beispiel das Signal von Kameras und Mikrofonen sein. Auch virtuelle Quellen wie Bildschirmaufnahmen, Audio-, Bild- \& Videodateien und Texte können eingebunden werden. Die Quellen werden, wie in \autoref{img:obs_overlays} bildlich dargestellt, wie verschiedene Layer übereinandergelegt und ergeben ein gemeinsames Bild. Das entstandene Bild wird von \acrshort{obs} in ein vorgegebenes Format kodiert. Das fertige Material kann entweder als Videodatei auf der Festplatte abgespeichert werden oder über das \acrfull{rtmp}, an einen Online Service wie Twitch geschickt werden. \parencite{Alam2018}

\begin{figure}[ht]
\caption{Overlay Layers (eigene Darstellung)}
\label{img:obs_overlays}
\centering
\includegraphics[width=\textwidth]{gami_overlay_layers}
\end{figure}

Overlays sind Ebenen welche über das Videobild des Streams gelegt werden. Daher auch der Name Overlay, vom deutschen \glqq Überlagern\grqq{}. Ein Overlay kann ein Bild sein, das zum Beispiel als Rahmen für die Webcam der Streamenden benutzt wird. Auch dynamische Inhalte wie Webseiten können eingebunden werden. Ein großer Vorteil dieser Webseiten ist deren Fähigkeit, sich über einen WebSocket mit der Twitch API zu verbinden. Eine WebSocket Verbindung ist ein Kommunikationskanal zwischen zwei Parteien. Beide Parteien können Pakete senden und empfangen. Somit ist es eine zwei-Wege-Kommunikationsverbindung \parencite{Mozilla}.  Diese Parteien sind in diesem Fall die Webseite und die Twitch API. Über diesen Kanal kann die Webseite Event-basierte Nachrichten erhalten ohne eine neue Anfrage hierfür zu senden. Ein solches Event kann zum Beispiel das Folgen, Subscriben oder Cheeren in einem Channel sein. Beliebte Formen auf diese Events zu reagieren sind Fortschrittsbalken und Alerts. 

Fortschrittsbalken werden dafür verwendet, gemeinsame Ziele für das Publikum zu definieren (Core Drive 1 und 5). Diese Fortschrittsbalken werden in diesem Fall auch \glqq Overlay Goals\grqq{} genannt. Die Ziele können beispielsweise eine bestimme Anzahl an Followern, Subs, Bits oder direkte Spenden sein. Wie in \autoref{img:goal_overlay} zu sehen, wurde als Ziel ein Geldbetrag von 300\$ für einen neuen Gaming-Stuhl definiert. Die Zuschauer haben bereits zusammen 245\$ erreicht. Spendet nun eine weitere Person wird der Fortschittsbalken um den gespendeten Betrag erhöht (Core Drive 2). Für die Beendigung dieser Ziele wird ein Enddatum festgelegt (Core Drive 6). Hierdurch besteht auch die Gefahr, dass dieses Ziel nicht erreicht werden kann (Core Drive 8).


\begin{figure}[ht]
\caption{Goal Overlay \parencite{Streamlabsa}}
\label{img:goal_overlay}
\centering
\includegraphics[width=\textwidth]{twitch_goals}
\end{figure}

Alerts sind kurze Animationen die zusammen mit Hintergrundmusik sehr präsent im Stream angezeigt werden. Sie dienen als positive Rückmeldung bzw. Belohnung (Core Drive 2). Zusätzlich wird, wie in \autoref{img:alert} zu sehen, in der Regel auch der Name der Person angezeigt, die die dargestellte Aktion durchgeführt hat (Core Drive 5). Ebenfalls kann bei einer Subscription auch dieselbe Nachricht wie aus \autoref{chap:bllshit} angezeigt werden (Core Drive 3). Viele Streamerinnen und Streamer nutzen bei den Alerts auch den Zufallsfaktor und zeigen nur in manchen Fällen die Alerts an \parencite{Johnson2019} (Core Drive 7).

\begin{figure}[ht]
\caption{Alert Overlay}
\label{img:alert}
\centering
\includegraphics[width=\textwidth]{twitch_alert}
\end{figure}

Der Kreativität sind bei den Overlays keine Grenzen gesetzt. Alles was mit HTML, CSS und JavaScript implementiert werden kann, kann als Overlay dienen. Damit sich nicht jeder selbst seine Overlays programmieren muss, ist hierfür ein großer Markt an Dienstleistern entstanden. Die zwei bekanntesten sind StreamLabs und Streamelements. Auch Softwareanbieter die mit Videospiel-Streaming eng verknüpft sind, bieten Overlay-Dienste an. So hat Discord, der Anbieter einer Instant-Messaging und Voip-Plattform, das sogenannte \glqq Discord StreamKit\grqq{}. Damit können dynamische Elemente in den Stream eingebunden werden. Sie zeigen z.B. an, wie hoch die Nutzerzahlen gerade auf dem Server sind oder wer gerade in einem definierten Voip-Channel spricht.


\section{Chatbots}
\label{chap:bots}
Neben den Overlays sind auch Chatbots eine beliebte Möglichkeit einen Stream interessant zu gestalten. Diese sind kleine Programme, die dauerhaft mit dem Twitch Chat eines Channels verbunden sind. Sie können Nachrichten lesen und schreiben. Vorher definierte Schlüsselwörter mit einem Ausrufezeichen davor fungieren als Kommandos (!command). Wird ein solches Kommando in den Chat geschrieben, so führt der Bot eine vorher definierte Aktion durch. Diese Aktionen können zum Beispiel das Abspielen verschiedener Sounds oder das Einblenden kurzer Gifs  im Stream sein. (Core Drive 3)

Chatbots ähneln den Kanalpunkten aus \autoref{chap:tw_gami_punkte}. Der Bot erkennt wer aktuell mit dem Chat verbunden ist und gibt diesen in definierten Abständen Punkte. Der aktuelle Punktestand kann zum Beispiel mit dem Kommando \glqq !points\grqq{} ausgegeben werden. Punkte einlösen ist ebenfalls mit einem anderen Kommando möglich.

Die aktuellen Punktestände nutzen viele Bots auch, um ein Rankingsystem zu implementieren. So können Personen mit einem höheren Punktestand auch einen höheren Rang freischalten. Dadurch ist es ihnen möglich Kommandos auszuführen, die einen bestimmten Rang voraussetzen \parencite{Siutila2018} (Core Drive 2 und 5).

Im Gegensatz zu den nativen Kanalpunkten können hier die Punkte auch für verschiedene Aktionen Personen gutgeschrieben werden (Core Drive 4). So gibt es auch einige Minispiele, wie zum Beispiel Roulette, in denen ein User Punkte setzten kann, um diese zu vermehren oder alle zu verlieren (Core Drive 7). Ein weiteres Beispielt sind Affiliate-Links. Hierbei kann durch ein Kommando ein Link generiert werden, der bei Klicken durch Dritte dem Linkersteller Punkte gutschreibt. Die Dritten werden über den Link zum Stream geleitet. \parencite{Browne2018}

\section{Bewertung der Gamification-Elemente}
\label{chap:bewertung_gamif}

In der \autoref{tab:core_twitch} stellen die Spalten 1 bis 8 die Core Drives 1 bis 8 dar. 

\begin{table}[!htp]\centering
\caption{Core Drives Zusammenfassung Twitch}
\label{tab:core_twitch}
\scriptsize
\begin{tabular}{lrrrrrrrrr}\toprule
&\multicolumn{8}{c}{Core Drives} \\
\textbf{} &1 &2 &3 &4 &5 &6 &7 &8 \\
Leaderboard & &x & & &x &x & &x \\
User Type Badges & &x & & &x &x & & \\
Supporter Badges &x &x & &x &x & & & \\
Subscriber Badges & &x & &x &x & & &x \\
Leaderboard Badges & &x & &x &x &x & &x \\
HypeTrain Conductor &x &x & &x &x &x & &x \\
sonderaktionen Badge &x &x & & &x &x & & \\
Channelpoints & &x &x &x &x & &x & \\
Bits \& Cheeren &x & &x &x &x & & & \\
HypeTrain &x &x & & &x &x &x &x \\
Emotes & & &x &x &x &x & &x \\
Drops & &x & & & &x &x & \\
Achievements für Zuschauer & &x &x & &x & &x & \\
\textbf{Insgesamt Native} &5 &11 &4 &7 &12 &8 &4 &6 \\
& & & & & & & & \\
Overlay Goals &x &x & & &x &x & &x \\
Overlay Alerts & &x &x & &x & &x & \\
Chat-Bot Channelpoints & &x &x &x &x & &x & \\
\textbf{Insgesamt Overlay \& Bots} &1 &3 &2 &1 &3 &1 &2 &1 \\
& & & & & & & & \\
\textbf{Insgesamt} &6 &14 &6 &8 &15 &9 &6 &7 \\
\bottomrule
\end{tabular}
\end{table}

Das Octalysis-Framework bietet auch ein Tool zur visuellen Darstellung der Core Drives an. In \autoref{img:oct_twitch} sind die Daten aus \autoref{tab:core_twitch} visualisiert. Der blaue Bereich ist die Ausprägung der jeweiligen Core Drives für die nativen Gamification-Elemente auf Twitch. Es ist erkennbar, dass die \glqq Left Brain\grqq{}-Core Drives stärker ausgeprägt sind. Die rote Linie sind die Ausprägungen wenn für die Erstellung des Graphen auch die Core Drives der Overlays und Chatbots mit einbezogen wird. Es ist erkennbar, dass sich hierbei das Verhältnis leicht verschiebt und die Core Drives horizontal etwas besser ausgeglichen sind.

\begin{figure}[ht]
\caption{Octalysis Core Drives visualisiert (eigene Darstellung)}
\label{img:oct_twitch}
\centering
\includegraphics[width=0.7\textwidth]{oct_twich}
\end{figure}

Damit die Gamification Element auch in einer Kampagne mit definierten Zielen verwendet werden können, muss es eine Möglichkeit geben, diese Elemente zu beeinflussen. Hierfür kann auf die Twitch-Schnittstellen zurückgegriffen werden. Twitch bietet Entwicklern, über verschiedene Programmierschnittstellen, die Möglichkeit Software zu entwickeln. die sich leicht in den Workflow von Twitch integrieren lässt. \parencite{Twitchh}

Für diese Arbeit wurden die Schnittstellen Drops, IRC Chat, PubSub und Twitch API analysiert. Drops können ausgeschlossen werden, da diese lediglich im Zusammenhang mit Videospielen Anwendung finden. Die Schnittstelle IRC Chat bietet die Möglichkeit Bots zu entwickeln und mit dem Chat zu interagieren. Hierbei können mit den Chat-Kommandos \glqq/mod\grqq{} und \glqq /vip\grqq{} die Mod oder VIP-Rollen vergeben und mit \glqq/unmod\grqq{} und \glqq /unvip\grqq{} wieder entfernt werden. 

PubSub ist eine WebSocket Verbindung mit dem Twitch-Server. Diese Verbindung bietet die Möglichkeit sogenannte \glqq Topics\grqq{} zu abonnieren. Diese Topics sind unter anderem Subscriptions oder Cheers, die in einem definierten Channel aktiviert werden. Über die Websocket Verbindung schickt der Twitch-Server eine Nachricht an den verbundenen Client, sobald ein abonniertes Topic getriggert wird. Da diese Mechanik aber lediglich lesend ist können Gamification-Elemente nicht beeinflusst werden.

Die Analyse der Twitch API hat ergeben, dass über die Schnittstellen auf die Gamification-Elemente nur lesend zugegriffen werden kann. Daher ist die Twitch API in dieser Arbeit nicht anwendbar.

Overlays und Chat-Bots bieten, wie bereits erwähnt, grenzenlose Möglichkeiten. Aus diesem Grund wird sich in dieser Arbeit auf die Anwendungsbeispiele aus \autoref{chap:gami_overlays} begrenzt. Diese sind die am häufigsten verwendeten Gamification-Elemente in der Praxis. Alles weitere würde den Rahmen dieser Arbeit sprengen. 

%
%
%
%
%
%
%
%
%
%
% 
\chapter{Marketing auf Twitch}
Online-Marketing ist ein sehr großes Feld und umfasst viele verschiedene Strategien. Diese können einzeln angewandt oder miteinander kombiniert werden. Eine beispielhafte Übersicht über die aktuelle Landschaft der Marketing-Strategien kann \autoref{img:mark_sek} entnommen werden.

\begin{figure}[ht]
\caption{Online Marketing Kompakt im Sektorenmodell \parencite{Volkl-wolf2020}}
\label{img:mark_sek}
\centering
\includegraphics[width=0.8\textwidth]{mark_sek_mod}
\end{figure}

Aufgrund des Aufbaus und der Art und Weise wie Twitch funktioniert, werden nur bestimmte Formen des Online-Marketings dort angewandt. Diese sind Display-Marketing, Video-Marketing, Branded Content-Marketing, Affiliate-Marketing und Influencer-Marketing. Im Folgenden werden die verschiedenen Formen anhand von Beispielen erklärt. Danach wird bewertet und überprüft, wie diese mit Gamification kombiniert werden können.

\section{Display-Marketing}
Bei Display-Marketing werden auf einer Webseite verschiedene Banner eingeblendet, die über Produkte oder Firmen informieren. Aus diesem Grund wird diese Methode alternativ auch Banner-Werbung genannt. Diese Banner können in verschiedenen Formen und Größen auf einer Webseite eingebunden werden. Beispielsweise gibt es sogenannte \glqq Skyscraper\grqq{}, welche am rechten bzw. linken Bildschirmrand platziert sind. Oft werden Banner auch zwischen normalen Content einer Webseite angezeigt, um die Aufmerksamkeit der Betrachtenden in einem konzentrierten Moment auf ein Produkt zu lenken. Banner können auch unerwartet bildschirmfüllend erscheinen, in manchen Fällen sogar als eigenes Fenster oder Tab. Diese Art der Banner wird Pop-Ups genannt.\parencite{Xovi2019}

Firmen können auf Twitch Banner-Werbung schalten, um Produkte zu bewerben oder ihren Livestream hervorzuheben. Diese Banner werden in vier verschiedenen Bereichen der Plattform angezeigt. Beginnend von links nach rechts in \autoref{img:banner_ads} ist die erste Variante, die als \glqq Medium Rectangle\grqq{} bezeichnet wird, ein mittelgroßer Banner für Produkte, welcher in der Explore Seite von Twitch zwischen den Kategorie-Kacheln angezeigt wird. Das \glqq Super Leaderboard\grqq{} wird ebenfalls für Produkte auf der Browse-Seite angezeigt, jedoch am oberen Browser-Rand als langer horizontaler Balken. Auf der Startseite von Twitch wird als erstes Element das sogenannte \glqq Homepage Carousel\grqq{} angezeigt. In diesem werden Livestreams hervorgehoben. Die Kanäle die dort angezeigt werden, haben sich für die dritte Variante der Banner-Werbung entschieden. Die letzte Variante ist ein großer horizontaler Banner, welcher hinter dem Carousel angezeigt wird, um Produkte zu bewerben. \parencite{Twitch2020}

\begin{figure}[ht]
\caption{Bannerplatzierung auf Twitch, Screenshots \parencite{Twitcha}}
\label{img:banner_ads}
\centering
\includegraphics[width=\textwidth]{twitch_ads_banner}
\end{figure}

\section{Video-Marketing}
Video-Marketing ist der Einsatz von audio-visuellen Inhalten in einer Marketingkampagne. Einfach betrachtet ist Video-Marketing das Erstellen und Platzieren eines Videos. Die Ziele dieser Videos können vielfältig sein und reichen vom Bewerben der Firma, Anregen von Verkäufen, Erhöhen der Bekanntheit einer Marke oder Dienstleistung bis hin zum Stärken der Kundenbindung.\parencite{Stringfellow2017}

Twitch als reine Video-Plattform hat Video-Marketing  als einen Kernbestandteil. Der Plattform wurde im Jahr 2019 der \glqq Best Video Marketing and Advertising Platform\grqq{} Award verliehen \parencite{DigidayAwards2019}.

Auf Twitch wird Video-Marketing in Form von kurzen Werbeclips im Videofeed des Streams eingesetzt. Hier werden diese zu verschiedenen Zeiten des Streams eingeblendet. Es wird zwischen pre-roll, mid-roll und post-roll Ads unterschieden. Dabei handelt es sich um klassische Werbevideos wie sie auch im Fernsehen zu sehen sind. Diese werden vor, während oder nach einem Livestream angezeigt. Wird beispielsweise auf einen Live-Channel geklickt, um dessen Stream beizutreten, wird automatisch ein pre-roll Werbeclip gestartet. Wie in \autoref{img:preroll} zu sehen, wird oben rechts in der Ecke die noch verbleibende Dauer angezeigt und der Clip kann nicht übersprungen werden. 

\begin{figure}[ht]
\caption{Twitch Pre-Roll Ad; Screenshot von \parencite{LaraLoft2020}}
\label{img:preroll}
\centering
\includegraphics[width=\textwidth]{twitch_video_preroll_edit}
\end{figure}

Bei den mid-roll Werbeclips haben die Streamerinnen und Streamer die Möglichkeit den Zeitpunkt und die Dauer der Einblendung selbst zu bestimmen. Hierbei können sie zwischen 30, 60, 90, 120, 150 und 180 Sekunden wählen. Diese nutzen sie beispielsweise, um in möglichst spannenden Szenen des Streams die Aufmerksamkeit auf die Werbung zu lenken oder eine kurze Pause einzulegen. Diese Werbeclips werden im sogenannten \glqq Picture-By-Picture\grqq{} Modus angezeigt. Hierbei wird bei Aktivieren des Clips der Livestream verkleinert und im rechten Bildrand angezeigt, während die Werbung den ursprünglichen Platz des Streams einnimmt \parencite{Twitch}. Die post-roll Werbeclips werden nach Beendigung des Streams eingeblendet.

\begin{figure}[ht]
\caption{Twitch Video-Ads Varianten, Screenshots \parencite{Twitcha}}
\label{img:twitch_vid_ads}
\centering
\includegraphics[width=\textwidth]{twitch_ads_video}
\end{figure}

Twitch bietet wie in \autoref{img:twitch_vid_ads} zu sehen drei verschiedene Varianten an, wo die Werbeclips bei den Konsumenten zu sehen sind. Werbende können sich für die gleichzeitige Ausspielung über die Desktop und Mobile Plattform entscheiden. Sie können sich aber auch gezielt für eine der beiden entscheiden, um dadurch spezifischer auf eine Zielgruppe einzugehen. \parencite{Twitcha} Die Werbevideos werden im selben Bereich angezeigt wie auch der Livestream.

Jedoch werden die Werbevideos nicht angezeigt, falls ein AdBlocker (dt. Werbeblocker) im Browser benutzt wird. Das ist eine Browser-Erweiterung, welche das Darstellen von Werbung verhindert. Für Subscriberinnen und Subscriber eines Channels oder für Twitch-Turbo-Mitglieder, werden die Werbeclips auch übersprungen.
Aus diesem Grund hat Twitch, wie in \autoref{img:twitch_vid_ads} zu sehen, eine neue Technologie mit dem Namen SureStream entwickelt, welche es der Plattform ermöglicht die Werbeclips direkt in das Videobild des Livestreams zu integrieren. Dadurch hat ein AdBlocker nicht mehr die Möglichkeit das Abspielen einer Werbung zu unterbinden. \parencite{Twitch2016}

Zudem wird Video-Marketing auch ganz allgemein auf Twitch eingesetzt. Wie bereits eingangs beschrieben umfasst diese Marketingstrategie jegliche Art von audio-visuellem Content. Somit können auch die im folgenden erklärten Marketingstrategien in der Art und Weise wie sie auf Twitch Anwendung finden, unter dem Video-Marketing angesiedelt werden.

\section{Branded Content-Marketing}
Bei Branded Content-Marketing geht es darum relevante Inhalte zu produzieren und diese direkt mit einer Marke zu verknüpfen. Die Inhalte sollen ein positives Gefühl mit der Marke assoziieren und die Gefühle des Publikums ansprechen. Sie haben oft einen unterhaltsamen Charakter und es wird in der Regel Storytelling angewandt.  Die Inhalte werden oft auch in Kooperation mehrerer Firmen oder Influencerinnen und Influencern produziert. Firmen, welche bekannt dafür sind diese Strategien besonders gut einsetzen, sind Red Bull und Coca Cola \parencite{Cardona2020}. Eine solche Werbekampagne ist gut dafür geeignet die Markenbekanntheit zu erhöhen, das Image der Marke zu kommunizieren und ein Gefühl mit dieser Marke zu verbinden.

Auf Twitch startete beispielsweise die Firma Old Spice, ein Hersteller von Pflegeprodukten für Herren, im Jahr 2015 eine interaktive Kampagne auf Twitch. Unter dem Titel \glqq Old Spice Nature Adventure\grqq{} wurde ein Live Stream erstellt, in welchem ein Mann, mit einer Kamera auf dem Rücken gespannt, durch einen Wald läuft. Das Videosignal dieser Kamera wurde im Stream übertragen. Das Publikum konnte Vorschläge in den Twitch Chat schreiben, welche Aktion der Mann als nächstes durchführen soll. Das Produktionsteam suchte aus diesen Vorschlägen etwa alle 10 - 30 Sekunden den Besten aus und der Mann führte diesen durch. Der nächste durchzuführende bzw. aktuelle Vorschlag wurde unten links im Bild (vgl. \autoref{img:oldspice}) zusammen mit dem Namen der Vorschlaggebenden eingeblendet.

\begin{figure}[ht]
\caption{Old Spice Nature Adventure; Screenshot von \parencite{OldSpice2015}}
\label{img:oldspice}
\centering
\includegraphics[width=\textwidth]{twitch_marketing_old_spice}
\end{figure}

Die Kampagne war wie ein virtuelles Rollenspiel aufgebaut. Der Mann war der Spieler-Charakter, welcher durch das Publikum gesteuert wurde. Im Wald, in dem sich der Mann bewegte, waren verschiedene Szenen aufgebaut. Zudem gab es in diesem Wald auch \acrfull{npc}, mit denen der Mann interagieren konnte. Einige dieser \acrshort{npc}s waren als Tiere verkleidet, wodurch einige Szenen entstanden sind, in denen der Mann gegen einen Bären kämpfen musste (vgl. \autoref{img:oldspice}).

Die Werbekampagne von Old Spice erreichte 2,56 Millionen Aufrufe und 29 Jahre an insgesamt angesehenem Video Material. Die exklusiv für diesen Stream freigeschaltenen Chat-Emotes wurden über 105.000 Mal verwendet.  \parencite{Workman2015}

\section{Affiliate-Marketing}
Affiliate-Marketing ist ein provisionsbasierter Ansatz, um ein Produkt direkt an eine Zielgruppe zu bringen. Hierbei können sich Content Creator sogenannte \glqq Affiliate-Links\grqq{} generieren, welche direkt auf die Seite eines Produktes verweisen. In diesem Link ist ein Parameter gesetzt, welcher den Linkersteller eindeutig als solchen identifiziert, sozusagen eine Referenz zu diesem herstellt. Die Links werden deshalb auch \glqq Ref\grqq{}-Links bezeichnet. Sie werden in der Regel bei Produktreviews, Inventarlisten oder in Artikeln über die bestimmte Produktkategorie als Empfehlung eingesetzt. Klicken Interessierte nun auf einen solchen Link gelangen sie direkt auf die Seite des Produktes. Wird dieses Produkt daraufhin gekauft so wird eine Provision an den Linkersteller ausgezahlt, für die Kaufenden entstehen jedoch keine Mehrkosten. \parencite{Gallaugher2001}

\begin{figure}[ht]
\caption{Ref Links in der \glqq About\grqq{} -Sektion auf Twitch \parencite{LeosMind}}
\label{img:aff_about}
\centering
\includegraphics[width=\textwidth]{twitch_affiliate_reflinks}
\end{figure}

Twitch bietet die Möglichkeit in der \glqq About\grqq{} -Sektion eines Profils verschiedene Banner und Texte einzublenden. Ebenso können auch Links eingebunden werden. Dieser Bereich wird von vielen Streamenden dazu genutzt Affiliate-Links zu integrieren. Die \glqq About\grqq{} -Sektion wird auch während eines Streams unterhalb des Videobilds angezeigt. Wie in \autoref{img:aff_about} zu sehen, hat der Streamer (1) eine Liste mit den dazugehörigen Links seiner verwendeten Hardware in seiner \glqq About\grqq{} -Sektion integriert. Jeder dieser Links ist ein \glqq Ref\grqq{}-Link zu Amazon. Auch ein (2) Banner, welcher zur Seite eines Drittanbieters führt, wurde mit einem \glqq Ref\grqq{}-Link versehen. Zudem können Streamenden über einen Bot auch regelmäßig \glqq Ref\grqq{}-Links in den Chat schicken lassen. 

\section{Influencer-Marketing}

Um zu verstehen was Influencer-Marketing bedeutet muss zunächst der Begriff Influencer erklärt werden. Influencerinnen und Influencer sind Personen, die aufgrund ihres Status und ihrer Reichweite in einer Nische einen signifikanten Einfluss auf die Kaufentscheidungen anderer Personen haben \parencite{InfluencerMarketingHub2017}. Sie werden von Firmen bezahlt um deren Produkte zu bewerben. Die Bewerbung der Produkte kann durch gesponserte Inhalte, Empfehlungen oder Produktplatzierungen geschehen. Da das Publikum auf Twitch besonderen Wert auf Authentizität legt, ist bei der Auswahl der Streamerinnen und Streamer darauf zu achten, dass diese mit der Marke harmonieren. \parencite{Reachbird2020}

\begin{figure}[ht]
\caption{Alien Promotion von Lara Loft}
\label{img:laraloftalien}
\centering
\includegraphics[width=\textwidth]{twitch_laraloft_alien_promo_edit}
\end{figure}

Zum Start von \glqq Alien: Convenant\grqq{} im Jahr 2017 führte 20th Century Fox eine Kampagne auf Twitch durch. Diese bestand jedoch nicht aus dem Besprechen des Films oder dem gemeinsamen Schauen des Trailers im Stream, sondern aus der Inszinierung einer Alien-Invasion kurz vor Ende des Streams. Es wurde mit flackernden Lichtern, Bildstörungen oder auch wie in \autoref{img:laraloftalien} zu sehen mit CGI-Effekten wie (1) deutlichen Kratzern an der Wand und (2) einem Alienschwanz, der am Rande des Bildes vorbeistreift zuerst eine gruselige Atmosphäre geschaffen, um Spannung aufzubauen. Nachdem das Licht komplett ausfiel und die Streamerin aus der Kamera trat, um dieses wieder einzuschalten, sprang ein Alien direkt in die Kamera um die Zuschauer zu erschrecken. Daraufhin startet der Trailer zum neuen Film. Dieser erreichte mehr als eine halbe Millionen Aufrufe. \parencite{T3N2018}

\section{Bewertung bzgl. der Einsetzbarkeit von Gamification}
\label{chap:bew_mark_meth}
Für die Bewertung der einzelnen Marketingmethoden auf Twitch wurden diese im Detail über mehrere Monate beobachtet. Hierbei wurde festgestellt, dass die Banner des Display Marketings offensichtlich kaum noch eine Rolle spielen. Bei der Aktivierung eines AdBlockers werden diese zuverlässig ausgeblendet. War der AdBlocker jedoch deaktiviert, so war die Wahrscheinlichkeit, dass ein Banner angezeigt wurde, sehr gering. Während des Schreibens dieser Arbeit wurde gezielt auf die Benutzung eines AdBlockers verzichtet. Jedoch wurde im gesamten Zeitraum von drei Monaten nur einmal ein Banner im Header der Startseite angezeigt.  Deshalb werden Banner für diese Arbeit als nicht relevant definiert.

Da bei Affiliate-Marketing durch einen Link auf die Seite eines Drittanbieters weitergeleitet wird, können die dortigen technischen Freiheiten genutzt werden. So könnte beim Kauf eines Produktes ein Event ausgelöst werden, welches die Overlays in einem Twitch Videobild beeinflusst. Somit bietet diese Marketingstrategie eine Möglichkeit Gamification-Elemente einzusetzen. 

Die neue Technologie der SureStreams kann, wie beworben, auch von einem AdBlocker nicht umgangen werden. Die Gamification-Elemente der Plattform finden jedoch erst ihren Einsatz im Videobild des Streams. Somit können die Video-Ads auch als nicht relevant definiert werden. Allerdings ist Video-Marketing auf Twitch wie bereits beschrieben auch in Branded Content-Marketing und Influencer-Marketing implementiert. 

Als Resultat können drei mögliche Marketingstrategien definiert werden, die der \autoref{tab:mark_gami} entnommen werden können. Zum ersten Branded Content-Marketing in Form eines Livestreams auf dem Channel der Firma. Zweitens wenn Firmen Influencerinnen oder Influencer bezahlen um einen Livestream zu produzieren welcher auf deren Channels ausgestrahlt wird, möglicherweise auch in Form von Branded Content-Marketing. Drittens das Affiliate-Marketing mit den entsprechenden Links, die durch Aktionen wie Kaufen die Overlays beeinflussen. 

\begin{table}[!htp]\centering
\caption{Marketing Methoden Gamification Anwendbarkeit}\label{tab:mark_gami}
\scriptsize
\begin{tabular}{lrr}\toprule
\textbf{Marketing Methode} &\textbf{Gamification anwendbar} \\\midrule
Display-Marketing &nicht anwendbar \\
VideoAds &nicht anwendbar \\
Branded Content-Marketing &anwendbar \\
Affiliate-Marketing &anwendbar \\
Influencer-Marketing &anwendbar \\
\bottomrule
\end{tabular}
\end{table}
%
%
%
%
%
%
%
%
%
%
% 
\chapter{Beispielhafte Konzeption einer Gamification Kampagne}

\section{Anforderungsanalyse durch Experteninterviews}
Am Anfang vieler neuer Entwicklungen steht als erster Prozesspunkt das Requirements Engineering. In diesem wird der Ist-Zustand festgestellt, Ziele definiert und die Anforderungen festgelegt. Um die Anforderungen herauszufinden müssen die Stakeholder des Projektes mit in diesen Prozess einbezogen werden. \parencite{Balzert2009} 

Hierfür gibt es viele verschiedene Ansätze. Für diese Arbeit wurde für die Anforderungsanalyse ein Ansatz mit einer qualitativen Datenerhebung gewählt. In diesem werden semistrukturierte Interviews durchgeführt. Diese haben den Vorteil, dass Verlauf des Gesprächs individuell angepasst werden kann. Es können ebenfalls bei Bedarf Rückfragen gestellt werden, um der Analyse eine nötige Tiefe zu geben. \parencite[107\psqq]{Rupp2014}

Diese Art der Interviews werden auch Leitfadeninterview genannt. In den Interviews werden Fragen gestellt, um dem Interview eine thematische Richtung zu geben, es aber nicht zu sehr einzuschränken \parencite{Wessel2010}. Auch für dieses Interview wurde ein Leitfaden erstellt. In diesem wird der Ablauf genau definiert und dient während des Gespräches als Orientierung.

Zu Beginn des Interviews wird die befragte Person begrüßt und die Rahmenbedingungen für das Interview werden geklärt.

\begin{itemize}
\item Begrüßung der befragten Person und Bedanken für die Teilnahme
\item Kurze Einführung in das Thema
\item Erklären des Interview-Leitfadens
\item Datenschutzvereinbarung
\end{itemize}

Daraufhin wird mit einleitenden Fragen zum Thema hingeführt.

\begin{itemize}
\item Wissen Sie was Gamification ist?
\item Was genau ist Ihre Beziehung zur Musikindustrie und Twitch? Wie lange besteht diese Beziehung schon?
\item Wie und wo sind Ihre täglichen Kontaktpunkte und Tätigkeiten im Bezug auf die Musikindustrie?
\item Wie viele Zuschauer haben Sie im Durchschnitt auf Twitch?
\end{itemize}

Im Hauptteil des Interviews werden dann die Schlüsselfragen gestellt. Hierfür müssen die Interviewfragen systematisch aus der Forschungsfrage der Arbeit abgeleitet werden. Dabei wurde sich am Prozess von Kaiser orientiert. In diesem werden zunächst aus der Forschungsfrage die Analysedimensionen definiert, hieraus im Anschluss Fragenkomplexe gebildet, aus welchen dann wiederum konkrete Fragen für die Interviews entstehen. \parencite{Kaiser2014}

Die Forschungsfrage in dieser Arbeit lautet \textbf{\glqq Wie können die Gamification-Elemente der Plattform Twitch von Firmen in der Musikbranche für Marketingzwecke eingesetzt werden?\grqq{}} Bei der Betrachtung der Forschungsfrage fallen direkt drei Schlagworte auf:

\begin{itemize}
\item Musikbranche
\item Marketingzwecke
\item Gamification
\end{itemize}

Diese sind die drei Kernbestandteile der Forschungsfrage und können somit auch als Analysedimensionen verwendet werden. In der ersten Dimension \glqq Musikbranche\grqq{} ist herauszufinden, wie die Industrie funktioniert bzw. womit Firmen Geld verdienen. Daraus leitet sich direkt die Interviewfrage eins ab um die Hintergründe und Funktionsweise dieser Industrie zu verstehen.

Die zweite Dimension \glqq Marketingzwecke\grqq{} dreht sich um die Frage, was genau die zu implementierende Kampagne erreichen soll. Für eine erfolgreiche Kampagne müssen mehrere Faktoren berücksicktigt werden. Auf der einen Seite muss man sich seiner Zielgruppe im Klaren sein \parencite{Knauer2010}. Daher sollte in den Interviews im Fragekomplex \glqq Zielgruppe\grqq{} auch herausgefunden werden wie genau die Zielgruppe aussieht. Als konkrete Interviewfrage leitet sich hier die Frage zwei nach der wichtigsten Zielgruppe ab. Zudem ist auch in Social Media-Marketing die Tonalität bzw. der Umgangston mit dem Kunden sehr wichtig \parencite{Ballouli2016}, woraus sich die dritte Frage ableitet. In Bezug auf Zielgruppen ist es auch wichtig zu wissen über welche Kanäle mit dieser kommuniziert wird, wodurch die vierte Frage entsteht.

Eine gut geplante Kampagne benötigt vorher definierte Ziele, um im Nachgang den Erfolg dieser Kampagne messen zu können. Deshalb werden, als weiterer Fragenkomplex unter den Marketingzwecken, die Kampagnenziele definiert \parencite{Klein}. Daraus werden Fragen fünf und sechs nach den aktuellen Zielen der Befragten und mögliche gamifizierbare Ziele ausgearbeitet.

Die letzte Dimension ist \glqq relevante Gamification-Elemente\grqq{}. Dies wird benötigt  um herauszufinden welche Elemente von der Plattform Twitch eingesetzt werden können, was mit der siebten Frage erörtert wird. In \autoref{chap:bewertung_gamif} wurden bereits die auf Twitch vorhanden Elemente analysiert. Wie in \autoref{chap:gamif} angesprochen, leiten sich die zu verwendenden Elemente aus den verfolgten Zielen ab. Somit können später die Ergebnisse von Frage sieben verwendet werden.

Eine grafische Darstellung der Herleitung der Interviewfragen kann \autoref{img:herleitung_intfragen} entnommen werden. Die blauen Felder sind die Dimensionen, die grünen Felder die Komplexe und die roten Felder die Interviewfragen.

\begin{figure}[ht]
\caption{Herleitung der Interviewfragen (eigene Darstellung)}
\label{img:herleitung_intfragen}
\centering
\includegraphics[width=\textwidth]{interviewfragenherleitung}
\end{figure}

Nachfolgend die Auflistung aller Interviewfragen.

\begin{enumerate}
\item Frage: Wie funktioniert die Musikindistrie bzw. wie verdienen Unternehmens in der Musikbranche ihr Geld? Wie sind diese mit Twitch verknüpft?
\item Frage: Welche Zielgruppe ist besonders wichtig? Wie ist die demografische Verteilung? Was ist wichtiger, Quantität oder Qualität?
\item Frage: Was sind die aktuellen Wege einen Kunden / Fan zu erreichen? Über welche Kanäle kommunizieren Sie bereits?
\item Frage: Was ist wichtig im Umgang mit einem Kunden / Fan in Ihrer Branche? Welcher Umgangston ist zu verwenden, eher professionell distanziert oder nahbar?
\item Frage: Aktuelle Ziele von Marketingkampagnen? 
\item Frage: Mögliche gamifizierte Ziele einer Marketingkampagne auf Twitch?
\item Frage: Wie könnte man für eine Kampagne auf Twitch auch Gamification-Elemente einsetzen?
\end{enumerate}

Zum Abschluss des Gespräches wurden mit den Befragten noch folgende Punkte durchgeführt.

\begin{itemize}
\item Zusammenfassung der Mitschriften
\item Information über Verwendung der Informationen
\item Danke für die Teilnahme
\item Verabschiedung
\end{itemize}

Für die Erhebung der Daten wurden Experteninterviews digital in der Onlinemeeting Software \glqq Zoom\grqq{} abgehalten. Durch die integrierte Aufnahme-Funktion wurden die Interviews aufgezeichnet. Es wurden insgesamt drei verschiedene Interviews abgehalten. Um einen möglichen Querschnitt der auf Twitch agierenden Musizierenden zu erzielen, wurden zwei Affiliate- und ein Partner-Streamer ausgewählt.

\begin{itemize}
\item (Neuling) DJ und aktiver Twitch Nutzer mit weniger als 100 Followern auf Twitch (Affiliate)
\item (Fortgeschrittener) Musiker \& Produzent mit 5.000 Followern auf Twitch (Affiliate)
\item (Profi) DJ \& Produzent mit 6.000 Followern auf Twitch (Partner)
\end{itemize}

\subsection{Zusammenfassung und Auswertung der Ergebnisse}
\label{chap:zusammenf}

Um die Ergebnisse der drei Interviews analysieren zu können, müssen diese zunächst in eine einheitliche Form gebracht werden. Hierzu wird die Inhaltsanalyse nach \cite{Kuckartz2018} angewandt. Dabei handelt es sich um einen mehrstufigen Prozess, in welchem normalerweise die transkribierten Fließtexte der Interviews in verschiedene Kategorien zusammengefasst werden. Auf eine Transkription des Interviews wurde in dieser Arbeit verzichtet und die Informationen wurden direkt stichpunktartig in tabellarischer Form festgehalten.

Da in dieser Arbeit  die Antworten der Experten in stichpunktartiger Form erhoben wurden, kann auf den ersten Schritt der initiierenden Textarbeit verzichtet werden. Der zweite Schritt ist das Entwickeln von Hauptkategorien. Diese orientieren sich in diesem Fall an den Interviewfragen. In der Analyse mussten keine Unterkategorien gebildet werden. Die Kategorien können der \autoref{tab:int_cat} entnommen werden.

\begin{table}[!htp]\centering
\caption{Interview Kategorien}\label{tab:int_cat}
\scriptsize
\begin{tabular}{lrr}\toprule
\textbf{Code} &\textbf{Beschreibung} \\\midrule
WÜG &wissen über Gamification \\
MIR &Musikindustrie / Revenuestreams \\
WZ &Wichtigste Zielgruppe \\
UT &Umgangston \\
KK &Kommunikationskanäle \\
AZ &allgemeine / aktuelle Kampagnenziele \\
GZ &gamifizierte Ziele \\
GZM &gamifizierte Ziele Mechaniken \\
\bottomrule
\end{tabular}
\end{table}

Die Befragten konnten zunächst mit dem Begriff Gamification wenig anfangen bzw. hatten eine falsche Vorstellung was Gamification ist. Jedoch war ihnen die Mechaniken dahinter sehr wohl bewusst. So ist allen bereits aufgefallen, dass Twitch gezielt Elemente verwendet um das Publikum zu motivieren und an die Plattform zu binden.

Alle Befragten nannten eine sehr ähnliche Liste an möglichen Revenuestreams eines Unternehmens in der Musikbranche. Die wichtigsten Einnahmequellen sind Liveauftritte und der Verkauf von Musik.  Auch Merchandise-Verkäufe und das Livestreamen auf Twitch ist eine lukrative Methode. Der Vertrieb von Musik über Streaming-Dienste wie Spotify rentiert sich erst ab einer bestimmten Anzahl monatlicher Hörer.

Die wichtigste Zielgruppe sind die sogenannten \glqq Die-Hard\grqq{}-Fans. Diese sind besonders stark engagierte Unterstützer. Durch ihre eigene Überzeugung schaffen sie es, ohne weiteres Zutun der Künstlerinnen und Künstler, andere Personen für diese zu akquirieren.

Der Umgangston mit dieser Zielgruppe ist sehr freundschaftlich und nahbar. Auch ist der Begriff von \glqq Fan\grqq{} eher zu vermeiden. Es sollte immer von der \glqq Community\grqq{} gesprochen werden, da dieser Begriff die Zugehörigkeit zum Musizierenden in deren Augen eher unterstreicht.

Die genutzten Kommunikationskanäle sind vor allem die Social Media Plattform Instagram und der Direktnachrichten- \& VoIP-Dienst Discord. In Bezug auf Twitch sollte das Augenmerk besonders auf Discord gelegt werden, da dieser Dienst auch von Videospiel-Streamenden intensiv eingesetzt wird.

Aktuelle Ziele der Befragten sind es auf Twitch mehr Subscriber zu bekommen, in den Musikstreaming-Diensten wie Spotify mehr Follower zu generieren und ihren Vertriebszahlen an Merchandise-Artikeln zu erhöhen. Als mögliche gamifizierte Ziele wurden ebenfalls der Vertrieb von Merchandise genannt, sowie die Erhöhung der Mitgliederzahlen auf einem Discord Server.

Als mögliche Gamification-Mechaniken für diese Ziele könnten laut der Befragten vor allem Alerts dienen. So könnte ein Alert ausgelöst werden sobald jemand dem Discord beitritt oder einen Merchandise-Artikel kauft. Diese können ebenfalls dazu dienen das Publikum über die Existenz der Alerts zu informieren und auch Neid bei diesen auszulösen, da der Nutzername des Alert-Auslösers angezeigt wird.

\subsection{Definition der Kampagne}
Wie bereits durch die Analyse der Marketingmethoden in \autoref{chap:bew_mark_meth} herausgefunden, können für die Kampagne in dieser Arbeit drei Methoden herangezogen werden. Diese sind Video-Marketing in Bezug auf Streams, Branded Content-Marketing, Influencer-Marketing sowie Affiliate-Marketing.

Beispielhaft wird für den Rest dieser Arbeit definiert, dass Video-Marketing eingesetzt wird. Es wird ein eigener Stream produziert, in welchem die Gamification-Elemente eingesetzt werden. Als mögliche Marketingziele werden, wie bereits in \autoref{chap:zusammenf} durch die Befragten vorgeschlagen, die Erhöhung der Merschandise-Käufe, als auch die Erhöhung der Nutzerzahlen auf einem Discord-Server festgelegt.

\subsection{Gamification Konzept}
\label{kap:gami_konzept}

Für die Konzeption der Gamification in der Kampagne müssen die einzelnen Schritte des \acrshort{osd} aus \autoref{chap:des_fram} für jedes Ziel definiert werden. 

\begin{enumerate}
\item Anzahl verkaufter Artikel + Anzahl Discord User
\item /
\item Artikel kaufen + Discord beitreten
\item Jeweils ein Overlay Goal + Jeweils ein Alert + VIP Badge Gambling
\item Goals füllen sich + Teil der Community + Chance auf VIP
\end{enumerate}

Als Metriken werden die Anzahl an verkaufter Merchandise-Arikeln und Discord-Usern definiert. Erhöht sich diese Zahl nach Durchführung der Kampagne, so waren die Maßnahmen erfolgreich.

Da die Gamification-Elemente auf Twitch bereits seit Jahren erfolgreich eingesetzt werden, ist davon auszugehen, dass diese auf die Zuschauerschaft  von Twitch abgestimmt sind. In dieser Arbeit wird auf die gleichen Elemente zurückgegriffen, weshalb eine separate Betrachtung der Spielertypen nicht notwendig ist. Daher kann der zweite Schritt im \acrshort{osd} jeweils übersprungen werden.

Die Desired-Actions sind der Kauf eines Artikels und das Beitreten eines Discord Servers.

Da bereits in \autoref{chap:bewertung_gamif} eine sehr starke Eingrenzung der Gamification-Elemente stattgefunden hat, ist die Anzahl an möglichen Gamification-Elementen gering. 

Um das Publikum konstant über die Möglichkeit der Merchandise-Käufe und dem Discord in Kenntnis zu setzen, wird jeweils eine Overlay Fortschrittsbalken über das Video des Streams gelegt. Für das Durchführen der Aktion wird eine Person mit einem Alert belohnt, wodurch andere Personen aus dem Publikum ebenfalls über die Möglichkeiten in Kenntnis gesetzt werden.

Von den nativen Gamification-Elementen können die zwei Badges VIP und Moderator frei an Zuschauerinnen und Zuschauer vergeben werden. Die Anzahl der verfügbaren VIP-Badges ist hierbei abhängig von dem Fortschritt im dazugehörigen Achievement, wie in \autoref{chap:gami_achievements} beschrieben. Moderator-Rollen können beliebig viele vergeben werden. Da diese Rolle aber viel Verantwortung mit sich bringt und nur an vertrauenswürdige Mitglieder vergeben werden sollte, ist dieses Badge ungeeignet für eine Marketing-Kampagne. Somit ist das VIP-Badge, das einzige native Gamification Element das in Frage kommt. 

Es wird als Belohnung eingesetzt und in einer Art Los-Verfahren vergeben. Unter allen neuen Discord-Usern und Merch-Käufern besteht eine Chance, beim Beitreten des Discords oder Kaufen des Merches, ein VIP Badge zu erhalten. Darüber in Kenntnis setzt ein Schriftzug im Overlay.

Die angesprochenen Core Drives sind die Folgenden: 
\begin{itemize}
\item Epic Meaning \& Calling
\item Development \& Acomplishment
\item Social Influence \& Relatedness
\item Scarcity \& Impatience
\item Unpredictability \& Curiosity
\item Loss \& Avoidance
\end{itemize}

Der Core Drive \glqq Epic Meaning \& Calling\grqq{} zusammen mit \glqq Social Influence \& Relatedness\grqq{} gibt durch die Fortschrittsbalken dem Publikum das Gefühl, gemeinsam den Streamenden zu helfen. Sie arbeiten zusammen an einem höheren Ziel und müssen durch Spenden dieses erreichen. 

Nach Erreichen des gemeinsamen Ziels haben die User das Gefühl von \glqq Development \& Acomplishment\grqq{}. Diese Gefühl wird ebenfalls durch die Alerts nach jedem Kauf und Beitritt erzeugt.

Durch künstliche Zeitverknappung bei den Goals, ist auch \glqq Scarcity \& Impatience\grqq{} hier ein großer Faktor. Durch diese Zeitverknappung besteht auch die Möglichkeit das Ziel nicht zu erreichen, somit spielt auch \glqq Loss \& Avoidance\grqq{} eine Rolle.

Da mit jeder Aktion auch eine Chance besteht ein VIP-Badge zu erhalten haben die Zuschauer durchgehend das Gefühl von \glqq Unpredictability \& Curiosity\grqq{}.


\section{Architekturkonzept}
Das in der \autoref{img:archkonz} dargestellte Architekturkonzept besitzt einen zentralen Server als Herzstück der Anwendung. Auf diesem laufen sämtliche Informationen zusammen und werden in einer Datenbank gespeichert. Diese werden gebündelt über die Overlays und den TwitchIRC-Chat an die Zuschauer ausgegeben. 

Als Streaming-Software wird \acrshort{obs} eingesetzt. Auf dem Server ist das Overlay über eine HTTP-Verbindung abrufbar. Es wird mit HTML, CSS und JavaScript entwickelt. Beim Start von \acrshort{obs} wird dieses vom Server abgerufen und als Layer über das Videobild gelegt. Zudem stellt dieses Overlay eine WebSocket-Verbindung zum Server her. 

Das erste Ziel der Kampagne ist das Erhöhen der Nutzerzahlen auf dem Discord-Server. Da Discord eine Chat-Plattform ist, bietet es genauso wie Twitch die Möglichkeit Chatbots zu integrieren. Diese Bots können zum einen Unterhaltungen mitlesen, auf bestimmte Events wie das Beitreten von neuen Teilnehmerinnen und Teilnehmern reagieren und in den Chat schreiben. 

Für das Lesen der Nutzerzahlen und das Beitreten eines Users wird somit ein Bot entwickelt, der auf diese Events reagieren kann. Sobald jemand dem Discord Server beitritt, bekommt der Bot dieses Event mitgeteilt. Der Bot benachrichtigt daraufhin den Server um im Overlay das Goal zu aktualisieren und ein Alert anzuzeigen.

Für die Merchandise-Verkäufe wird ein Onlineshop eingerichtet. Auf diesem können sich die Zuschauerinnen und Zuschauer die im Stream beworbenen Produkte bestellen. Wird eine Bestellung abgeschlossen, so sendet das Backend des Shops eine Benachrichtigung an den Server. Dieser aktualisiert die Daten in der Datenbank und sendet über die WebSocket-Verbindung den Befehl zur Goal-Aktualisierung und Anzeige der Alerts an das Overlay in \acrshort{obs}. 



\begin{figure}[ht]
\caption{Architekturkonzept (eigene Darstellung)}
\label{img:archkonz}
\centering
\includegraphics[width=\textwidth]{Architekturkonzept_v4}
\end{figure}

%
%
%
%
%
%
%
%
%
%
% 
\chapter{Diskussion}
\label{chap:krit}
Das Thema Gamification ist sehr umfassend. Es ist nicht möglich dieses im Detail in einer Bachelorarbeit umfangreich zu behandeln. Aus diesem Grund wurden unter anderem die Overlays und Bots stark eingegrenzt. Es wurden nur die in der Praxis meist genutzten Beispiele behandelt. Dies führt zu einer einseitigen Betrachtung. Neben diesen gibt es aber auch viele neue Entwicklungen, die zum Beispiel von Streamelements oder Streamlabs zu Verfügung gestellt werden. 

Ebenfalls wurde davon ausgegangen, dass sich Twitch bewusst ist aus welchen Spielertypen sich ihre Zuschauerschaft zusammensetzt. Dies könnte eine negative Auswirkung auf das Ergebnis dieser Arbeit haben. Daher ist dies kritisch zu betrachten.

Bei der Analyse der Gamification-Elemente auf Twitch, wurden diese in Kategorien zusammengefasst. So wurden die VIP-Badges zusammen mit den Prime-Gaming-Badges betrachtet. Diese haben normalerweise unterschiedliche Core Drives. Durch die Zusammenfassung in einer Kategorie wurden diese vermischt. Ebensfalls wurde nur analysiert ob ein Core Drive vorhanden ist. Es sollte allerdings auch zusätzlich die Intensität der Core Drives berücksichtigt werden. Aus diesem Grund gingen in dieser Arbeit Details hinsichtlich der Core Drives verloren. 

In den Experteninterviews wurden lediglich drei Personen befragt. Diese waren zwar bereits auf Twitch aktiv, hatte aber lediglich eine Zuschauerschaft von 10 - 100 durchschnittlichen Zuschauern. Dies führt zu einer einseitigen und dadurch auch ungenauen Betrachtung. Für eine ausführlichere und genauere Analyse sollte Experten mit mehr Zuschauerzahlen herangezogen werden.

Ebenfalls waren die befragten Experten Amateure oder Kleinkünstler. Diese nutzen Twitch lediglich als zweites Standbein oder nur als Hobby. Für eine detaillierte Analyse sollten hier ebenfalls Expertinnen und Experten herangezogen werden, die Twitch als ihr Hauptstandbein ansehen. Denn gerade für diese ist es wichtig, durch effektiven Einsatz von Gamification-Elementen ihre Haupteinnahmequelle zu beeinflussen bzw. zu optimieren.

Da in dieser Arbeit am Ende keine Evaluation durchgeführt wurde, sind die Ergebnisse kritisch zu betrachten. Für aussagekräftige Ergebnisse ist eine Feldtest-Studie empfehlenswert. Mit dieser wäre es möglich statistisch beweisbare Ergebnisse zu erzeugen, um eine Aussage über die Effizienz einer solchen Kampagne zu treffen.

\chapter{Fazit und Ausblick}

Zusammenfassend ist zu sagen, dass Twitch nativ eine große Bandbreite an Gamification-Elementen besitzt. Allerdings finden diese, mit einer Ausnahme, keine Anwendung im Marketingbereich. Lediglich die VIP-Badges können zu Marketingzwecken eingesetzt werden.  Dies ist der fehlenden Möglichkeit, die Gamification-Elemente über die API-Schnittstellen zu beeinflussen, geschuldet.

Da Twitch eine Video-Plattform ist, die zur Übertragung des Bildes einen offenen Standard verwendet, kann mit spezieller Software das Bild des Streams mit Gamifcation Elementen erweitert werden. Hier können über Overlays Elemente wie Goals oder Alerts zum Videobild hinzugefügt werden. Diese können dazu verwendet werden den Zuschauer dazu zu motivieren etwas zu kaufen oder eine vorher definierte Aktion durchzuführen.

So könnten bei einem Kauf eines Artikels in einem externen Shop ein Merchandise-Alert im Stream ausgelöst werden, der den Namen des Käufers und den gekauften Artikel anzeigt. Dies soll den Käufer eine sofortige Befriedigung geben und andere Zuschauer dazu animieren ebenfalls einen Artikel zu kaufen. Mit Goals kann der Community ein Ziel vorgegeben werden. Durch dieses sind sie dazu animiert es gemeinsam zu erreichen.

Wie bereits in \autoref{chap:krit} angesprochen gibt es bei den Overlays, neben denen in dieser Arbeit behandelten, noch viele weitere Gamification-Elemente die von Streamelements oder Streamlabs zur Verfügung gestellt werden. Diese könnten in einer weiteren Studie ebenfalls betrachtet werden. 

In \autoref{chap:bots} wurde bereits erwähnt, dass Chatbots die Möglichkeit bieten auch Channelpoints direkt an User zu vergeben. Diese Funktion könnte Twitch ebenfalls in ihre nativen Kanalpunkte implementieren, um diese dadurch dynamischer zu gestalten und damit mehr Möglichkeiten zu bieten. Allerdings besteht hierbei jedoch die Gefahr von Pointsification. Daher könnte in einer weiteren Forschung überprüft werden, ob das eine sinnvolle Änderung sein könnte.













%
%
%
%
%
%
%
%
%
%
%###################################################################### ENDE
%\appendix
%\chapter{Anhang}
%\section{Transcript der Interviews}
%\includepdf[pages=-]{anhang_transcript.pdf}

\backmatter

\listoffigures
\addcontentsline{toc}{chapter}{Verzeichnisse}

\listoftables

%% create listings list
%  \lstlistoflistings
%  \addcontentsline{toc}{chapter}{Listings}

\cleardoublepage
\phantomsection
\printglossary[type=\acronymtype, title=Abkürzungsverzeichnis,toctitle=Abkürzungsverzeichnis]
\addcontentsline{toc}{chapter}{Literatur}
\printbibliography

\addchap{Eidesstattliche Erklärung}

Hiermit versichere ich, dass ich die vorgelegte Bachelorarbeit selbstständig verfasst und noch nicht anderweitig zu Prüfungszwecken vorgelegt habe. Alle benutzten Quellen und Hilfsmittel sind angegeben, wörtliche und sinngemäße Zitate wurden als solche gekennzeichnet.

\vspace{20pt}
\begin{flushright}
$\overline{~~~~~~~~~~~~~~~~~\mbox{\ShowBaAuthor, am \today}~~~~~~~~~~~~~~~~~}$
\end{flushright}

\addchap{Zustimmung zur Plagiatsüberprüfung}

Hiermit willige ich ein, dass zum Zwecke der Überprüfung auf Plagiate meine vorgelegte Arbeit in digitaler Form an PlagScan (www.plagscan.com) übermittelt und diese vorübergehend (max. 5~Jahre) in der von PlagScan geführten Datenbank gespeichert wird sowie persönliche Daten, die Teil dieser Arbeit sind, dort hinterlegt werden.

\begin{small}
Die Einwilligung ist freiwillig. Ohne diese Einwilligung kann unter Entfernung aller persönlichen Angaben und Wahrung der urheberrechtlichen Vorgaben die Plagiatsüberprüfung nicht verhindert werden. Die Einwilligung zur Speicherung und Verwendung der persönlichen Daten kann jederzeit durch Erklärung gegenüber der Fakultät widerrufen werden.
\end{small}

\vspace{20pt}
\begin{flushright}
$\overline{~~~~~~~~~~~~~~~~~\mbox{\ShowBaAuthor, am \today}~~~~~~~~~~~~~~~~~}$
\end{flushright}

\end{document}
